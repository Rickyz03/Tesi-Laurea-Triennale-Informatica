\chapter{Sistema di raccomandazione}
\label{cap:sistema-raccomandazione}

\intro{In questo capitolo, vengono descritte le teorie e le tecniche utilizzate per il sistema di raccomandazione, con particolare attenzione allo studio dei sistemi reali, delle tecniche di raccomandazione, di combinazione e di valutazione dei risultati. Viene inoltre dato un accenno ai temi della Serendipità e dell'Explainability, oggetto di studio del progetto sebbene non implementati allo stato dell'arte.}

\section{Requisiti per il dataset}

Perchè sia possibile implementare un sistema di raccomandazione per e-commerce, è necessario che il dataset contenga le informazioni necessarie per poter effettuare le raccomandazioni. In particolare, è necessario che il dataset contenga:
\begin{itemize}
    \item \textbf{ID Cliente}: un identificativo univoco per ogni cliente, che permette di distinguere tra i diversi clienti del servizio di e-commerce;
    \item \textbf{SKU Prodotto}: un identificativo univoco per ogni prodotto, che permette di distinguere tra i diversi prodotti del servizio di e-commerce;
    \item \textbf{Nome Cliente}: il nome del cliente, che permette di identificare quest'ultimo in modo più intuitivo; a differenza dell'ID Cliente, il nome non deve necessariamente essere univoco, cioè può essere condiviso da più clienti;
    \item \textbf{Nome Prodotto}: il nome del prodotto, che permette di identificare quest'ultimo in modo più intuitivo; a differenza dell'SKU Prodotto, il nome non deve necessariamente essere univoco, cioè può essere condiviso da più prodotti.
\end{itemize}

Queste informazioni vengono dunque richieste in input come colonne del file CSV che contiene il dataset. L'\gls{llm} ha il compito di riconoscere quali colonne del file CSV ne sono rappresentanti, dopodichè i nomi corrispondenti vengono standardizzati rispettivamente in: \texttt{"Customer ID"}, \texttt{"Product SKU"}, \texttt{"Customer Name"} e \texttt{"Product Name"}. Queste colonne vengono dunque poi utilizzate per implementare le tecniche di raccomandazione, che si basano sull'analisi delle interazioni tra i clienti e i prodotti e sulla loro similarità.

\section{Collaborative filtering}

\section{Similarità}

\section{Rank fusion}

\section{Recbole e Surprise}

\section{Metriche}

\section{Serendipità}

\section{Explainability}
