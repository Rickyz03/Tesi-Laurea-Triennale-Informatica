% \omiss produces '[...]'
\newcommand{\omissis}{[\dots\negthinspace]}

% Itemize symbols
% see: https://tex.stackexchange.com/a/62497
% \renewcommand{\labelitemi}{$\bullet$}
% \renewcommand{\labelitemii}{$\cdot$}
% \renewcommand{\labelitemiii}{$\diamond$}
% \renewcommand{\labelitemiv}{$\ast$}


\let\Chaptermark\chaptermark
% \Chaptername gives current chapter name
\def\chaptermark#1{\def\Chaptername{#1}\Chaptermark{#1}}
\makeatletter
% \currentname gives the current section name
\newcommand*{\currentname}{\@currentlabelname}
\makeatother

% Uncomment the following line for a different header/footer style
% \pagestyle{fancy} \setlength{\headheight}{14.5pt}
\fancyhead[L]{\fontsize{12}{14.5} \selectfont \thechapter. \Chaptername}
\fancyhead[R]{\fontsize{12}{14.5} \selectfont \currentname}
% Page number always in footer
\cfoot{\thepage}


% Custom hyphenation rules
\hyphenation {
    e-sem-pio
    ex-am-ple
}

% Images path, not using \graphicspath because it doesn't properly work with
% latexmk custom dependencies
\NewCommandCopy{\latexincludegraphics}{\includegraphics}
\renewcommand{\includegraphics}[2][]{\latexincludegraphics[#1]{images/#2}}

% Page format settings
% see: http://wwwcdf.pd.infn.it/AppuntiLinux/a2547.htm
%\setlength{\parindent}{14pt}    % first row indentation
%\setlength{\parskip}{0pt}       % paragraphs spacing

% Ho commentato le righe riguardanti i paragrafi, perchè ho \usepackage{parskip} che già si occupa di gestire gli spazi tra i paragrafi: toglie il rientro e aggiunge uno spazio verticale tra i paragrafi


% Load variables
\input{config/variables}

% Acronyms
\newacronym[description={\glslink{api}{Application Program Interface}}]
    {api_acr}{API}{Application Program Interface}

\newacronym[description={\glslink{uml}{Unified Modeling Language}}]
    {uml_acr}{UML}{Unified Modeling Language}

\newacronym[description={\glslink{llm}{Large Language Model}}]
    {llm_acr}{LLM}{Large Language Model}

\newacronym[description={\glslink{ml}{Machine Learning}}]
    {ml_acr}{ML}{Machine Learning}

\newacronym[description={\glslink{ner}{Named Entity Recognition}}]
    {ner_acr}{NER}{Named Entity Recognition}

\newacronym[description={\glslink{rrf}{Reciprocal Rank Fusion}}]
    {rrf_acr}{RRF}{Reciprocal Rank Fusion}

\newacronym[description={\glslink{mapatk}{Mean Average Precision@\emph{k}}}]
    {mapatk_acr}{MAP@\emph{k}}{Mean Average Precision@\emph{k}}

\newacronym[description={\glslink{mrratk}{Mean Reciprocal Rank@\emph{k}}}]
    {mrratk_acr}{MRR@\emph{k}}{Mean Reciprocal Rank@\emph{k}}

\newacronym[description={\glslink{csv}{Comma-Separated Values}}]
    {csv_acr}{CSV}{Comma-Separated Values}

\newacronym[description={\glslink{hdf5}{Hierarchical Data Format 5}}]
    {hdf5_acr}{HDF5}{Hierarchical Data Format 5}

\newacronym[description={\glslink{npy}{NumPy Array}}]
    {npy_acr}{NPY}{NumPy Array}






% Glossary entries
\newglossaryentry{api} {
    name=\glslink{api_acr}{API},
    text=API,
    sort=api,
    description={in informatica con il termine \emph{Application Programming Interface (API)} (ing. interfaccia di programmazione di un'applicazione) si indica ogni insieme di procedure disponibili al programmatore, di solito raggruppate a formare un set di strumenti specifici per l'espletamento di un determinato compito all'interno di un certo programma. La finalità è ottenere un'astrazione, di solito tra l'hardware e il programmatore o tra software a basso e quello ad alto livello semplificando così il lavoro di programmazione}
}

\newglossaryentry{uml} {
    name=\glslink{uml_acr}{UML},
    text=UML,
    sort=uml,
    description={in ingegneria del software \emph{UML, Unified Modeling Language} (ing. linguaggio di modellazione unificato) è un linguaggio di modellazione e specifica basato sul paradigma object-oriented. L'\emph{UML} svolge un'importantissima funzione di ``lingua franca'' nella comunità della progettazione e programmazione a oggetti. Gran parte della letteratura di settore usa tale linguaggio per descrivere soluzioni analitiche e progettuali in modo sintetico e comprensibile a un vasto pubblico}
}

\newglossaryentry{llm} {
    name=\glslink{llm_acr}{LLM},
    text=LLM,
    sort=llm,
    description={\emph{Large Language Model}, modello di linguaggio di grandi dimensioni addestrato su grandi quantità di dati testuali per comprendere e generare testo in linguaggio naturale}
}

\newglossaryentry{agile} {
    name=Agile,
    text=Agile,
    sort=agile,
    description={insieme di metodologie di sviluppo software basate su iterazioni brevi, collaborazione continua e adattamento ai cambiamenti, secondo i principi del Manifesto Agile}
}

\newglossaryentry{scrum} {
    name=Scrum,
    text=Scrum,
    sort=scrum,
    description={framework Agile per la gestione e lo sviluppo di progetti complessi, che suddivide il lavoro in iterazioni chiamate sprint e promuove la collaborazione tra team cross-funzionali}
}

\newglossaryentry{cloudfunctions} {
    name=Cloud Functions,
    text=Cloud Functions,
    sort=cloudfunctions,
    description={servizio di cloud computing che consente di eseguire codice in risposta a eventi senza dover gestire server, tipico dell'architettura serverless}
}

\newglossaryentry{cloudstorage} {
    name=Cloud Storage,
    text=Cloud Storage,
    sort=cloudstorage,
    description={servizio che permette di archiviare e gestire dati su server remoti accessibili tramite Internet, fornendo scalabilità e affidabilità}
}

\newglossaryentry{cloudcomputing} {
    name=Cloud Computing,
    text=Cloud Computing,
    sort=cloudcomputing,
    description={modello di erogazione di risorse informatiche (come server, storage, database, networking, software) tramite Internet, su richiesta e con pagamento a consumo}
}

\newglossaryentry{googlecloudstorage} {
    name=Google Cloud Storage,
    text=Google Cloud Storage,
    sort=googlecloudstorage,
    description={servizio di archiviazione cloud di Google che permette di memorizzare e accedere a dati su scala globale con elevate prestazioni, sicurezza e affidabilità}
}

\newglossaryentry{googlecloudfunctions} {
    name=Google Cloud Functions,
    text=Google Cloud Functions,
    sort=googlecloudfunctions,
    description={servizio serverless di Google Cloud Platform che consente di eseguire codice in risposta a eventi cloud senza dover gestire l'infrastruttura sottostante}
}

\newglossaryentry{preprocessing} {
    name=Preprocessing,
    text=preprocessing,
    sort=preprocessing,
    description={fase preliminare di elaborazione dei dati che comprende operazioni come pulizia, normalizzazione, trasformazione o selezione delle informazioni rilevanti, al fine di preparare i dati per l'analisi o l'addestramento di modelli o per altri scopi}
}

\newglossaryentry{busintel} {
    name=Business Intelligence,
    text=Business Intelligence,
    sort=businessintelligence,
    description={insieme di processi, tecnologie e strumenti che consentono di raccogliere, analizzare e presentare dati aziendali per supportare le decisioni strategiche e operative}
}

\newglossaryentry{ml} {
    name=Machine Learning,
    text=Machine Learning,
    sort=machinelearning,
    description={\glslink{ml_acr}{ML}, ramo dell'intelligenza artificiale che si occupa dello sviluppo di algoritmi e modelli in grado di apprendere automaticamente dai dati e migliorare le proprie prestazioni nel tempo senza essere esplicitamente programmati}
}

\newglossaryentry{ner} {
    name=\glslink{ner_acr}{NER},
    text=NER,
    sort=namedentityrecognition,
    description={tecnica di elaborazione del linguaggio naturale che consiste nell'identificare e classificare automaticamente entità nominate (come persone, organizzazioni, luoghi, date) all'interno di un testo}
}

\newglossaryentry{kmeans} {
    name=K-means,
    text=K-means,
    sort=kmeans,
    description={algoritmo di clustering non supervisionato che suddivide un insieme di dati in \emph{k} gruppi (cluster) in base alla vicinanza tra i punti, minimizzando la distanza intra-cluster}
}

\newglossaryentry{similarita} {
    name=Similarità,
    text=similarità,
    sort=similarita,
    description={misura della somiglianza tra due o più oggetti, spesso utilizzata in informatica per confrontare testi, immagini o altri dati}
}

\newglossaryentry{rrf} {
    name=\glslink{rrf_acr}{RRF},
    text=RRF,
    sort=rrf,
    description={metodo di combinazione dei risultati di più sistemi di recupero dell'informazione, che assegna un punteggio a ciascun elemento in base alla posizione nelle classifiche parziali, favorendo i risultati che appaiono in più classifiche}
}

\newglossaryentry{recallatk} {
    name=Recall@\emph{k},
    text=Recall@\emph{k},
    sort=recallatk,
    description={metrica che misura la frazione di elementi rilevanti recuperati tra i primi \emph{k} risultati restituiti da un sistema di ranking, rispetto al totale degli elementi rilevanti disponibili}
}

\newglossaryentry{precisionatk} {
    name=Precision@\emph{k},
    text=Precision@\emph{k},
    sort=precisionatk,
    description={metrica che misura la frazione di elementi rilevanti tra i primi \emph{k} risultati restituiti da un sistema di ranking}
}

\newglossaryentry{mapatk} {
    name=\glslink{map_acr}{MAP@\emph{k}},
    text=MAP@\emph{k},
    sort=mapatk,
    description={Mean Average Precision@\emph{k}, media delle precisioni calcolate a ogni posizione in cui un elemento rilevante appare tra i primi \emph{k} risultati}
}

\newglossaryentry{mrratk} {
    name=\glslink{mrratk_acr}{MRR@\emph{k}},
    text=MRR@\emph{k},
    sort=mrratk,
    description={Mean Reciprocal Rank@\emph{k}, media dei reciproci della posizione degli elementi rilevanti tra i primi \emph{k} risultati}
}

\newglossaryentry{unserendipityatk} {
    name=Unserendipity@\emph{k},
    text=Unserendipity@\emph{k},
    sort=unserendipityatk,
    description={metrica che misura la mancanza di serendipità tra i primi \emph{k} risultati, ovvero la tendenza a raccomandare elementi già noti o attesi dall'utente}
}

\newglossaryentry{averageitemsimilarity} {
    name=Average Item Similarity,
    text=Average Item Similarity,
    sort=averageitemsimilarity,
    description={media delle similarità tra gli elementi raccomandati e gli elementi rilevanti}
}

\newglossaryentry{meanpopularityatk} {
    name=Mean Popularity@\emph{k},
    text=Mean Popularity@\emph{k},
    sort=meanpopularityatk,
    description={media della popolarità (ad esempio, numero di interazioni o visualizzazioni) degli elementi raccomandati tra i primi \emph{k} risultati}
}

\newglossaryentry{csv} {
    name=\glslink{csv_acr}{CSV},
    text=CSV,
    sort=csv,
    description={Comma-Separated Values, formato di file per la memorizzazione di dati tabulari in cui i valori sono separati da virgole o altri delimitatori, utilizzato per l'importazione ed esportazione di dati tra applicazioni}
}

\newglossaryentry{hdf5} {
    name=\glslink{hdf5_acr}{HDF5},
    text=HDF5,
    sort=hdf5,
    description={Hierarchical Data Format versione 5, formato di file progettato per memorizzare e organizzare grandi quantità di dati in strutture gerarchiche complesse, ampiamente utilizzato in ambito scientifico e di ricerca}
}

\newglossaryentry{npy} {
    name=\glslink{npy_acr}{NPY},
    text=NPY,
    sort=npy,
    description={formato di file binario di NumPy per la memorizzazione efficiente di array multidimensionali, che preserva tipo di dato, forma e ordine dei dati}
}

\newglossaryentry{parquet} {
    name=Parquet,
    text=Parquet,
    sort=parquet,
    description={formato di file colonnare open-source ottimizzato per l'archiviazione e l'elaborazione di grandi dataset, che offre compressione efficiente e prestazioni elevate nelle query analitiche}
}

\newglossaryentry{zarr} {
    name=Zarr,
    text=Zarr,
    sort=zarr,
    description={formato per l'archiviazione di array multidimensionali ottimizzato per il cloud computing, che supporta compressione, chunking e accesso parallelo ai dati}
}

\makeglossaries

\bibliography{appendix/bibliography}

\defbibheading{bibliography} {
    \cleardoublepage
    \phantomsection
    \addcontentsline{toc}{chapter}{\bibname}
    \chapter*{\bibname\markboth{\bibname}{\bibname}}
}

% Spacing between entries
% \setlength\bibitemsep{1.5\itemsep}  % Non funziona: se cambi il valori numerico, non cambia nulla
\AtBeginBibliography{\setlength{\itemsep}{12pt plus 2pt minus 2pt}}

\DeclareBibliographyCategory{libri}
\DeclareBibliographyCategory{articoli}
\DeclareBibliographyCategory{web}

% Libri
\addtocategory{libri}{book:introduction-information-retrieval}

% Articoli
\addtocategory{articoli}{article:serendipity-recommender-systems}
\addtocategory{articoli}{article:item-based-collaborative-filtering}
\addtocategory{articoli}{article:empirical-analysis-predictive-algorithms}
\addtocategory{articoli}{article:reciprocal_rank_fusion}
\addtocategory{articoli}{article:score_combination}
\addtocategory{articoli}{article:weighted_rank_aggregation}
\addtocategory{articoli}{article:rank_biased_precision}
\addtocategory{articoli}{article:mean_popularity}
\addtocategory{articoli}{article:diversity_metric}
\addtocategory{articoli}{article:surprise_metric}
\addtocategory{articoli}{article:confidence_metric}
\addtocategory{articoli}{article:expectedness}

% Siti web
\addtocategory{web}{site:rrf-explained}
\addtocategory{web}{site:collaborative-filtering}
\addtocategory{web}{site:cosine-similarity}
\addtocategory{web}{site:explainable-ai}

% Similarity measures
\addtocategory{web}{site:Euclidean_distance}
\addtocategory{web}{site:Pearson_correlation_coefficient}
\addtocategory{web}{site:Manhattan_distance}
\addtocategory{web}{site:Jaccard_index}
\addtocategory{web}{site:Hamming_distance}
\addtocategory{web}{site:Kullback-Leibler_divergence}

% Rank fusion and aggregation
\addtocategory{web}{site:borda_count}
\addtocategory{web}{site:score_normalization}
\addtocategory{web}{site:median}
\addtocategory{web}{site:geometric_mean}
\addtocategory{web}{site:learning_to_rank}
\addtocategory{web}{site:condorcet_method}

% Recommender systems
\addtocategory{web}{site:recbole}
\addtocategory{web}{site:surprise}

% Metrics and evaluation
\addtocategory{web}{site:metrics-recommender-systems}
\addtocategory{web}{site:recall}
\addtocategory{web}{site:precision_at_k}
\addtocategory{web}{site:mean_average_precision}
\addtocategory{web}{site:mean_reciprocal_rank}
\addtocategory{web}{site:average_item_similarity}
\addtocategory{web}{site:ndcg}
\addtocategory{web}{site:hit_rate}
\addtocategory{web}{site:coverage_metric}
\addtocategory{web}{site:novelty_metric}
\addtocategory{web}{site:serendipity}
\addtocategory{web}{site:serendipity_metric}
\addtocategory{web}{site:distance_measures}
\addtocategory{web}{site:root-mean-square_deviation}

% Python libraries
\addtocategory{web}{site:pandas}
\addtocategory{web}{site:babel}
\addtocategory{web}{site:anthropic}
\addtocategory{web}{site:matplotlib}
\addtocategory{web}{site:pillow}
\addtocategory{web}{site:reportlab}
\addtocategory{web}{site:jinja2}
\addtocategory{web}{site:sentence-transformers}
\addtocategory{web}{site:scikit-learn}
\addtocategory{web}{site:zarr}
\addtocategory{web}{site:google-cloud-storage}
\addtocategory{web}{site:google-cloud-functions}
\addtocategory{web}{site:numpy}
\addtocategory{web}{site:black}
\addtocategory{web}{site:ruff}
\addtocategory{web}{site:mypy}
\addtocategory{web}{site:pytest}

%\defbibheading{opere}{\section*{Riferimenti bibliografici}}
%\defbibheading{web}{\section*{Siti Web consultati}}
% Questi titoli delle sezioni sono ridondanti, perchè sono già presenti come "title" in \printbibliography


\captionsetup{
    tableposition=top,
    figureposition=bottom,
    font=small,
    format=hang,
    labelfont=bf
}

\hypersetup{
    %hyperfootnotes=false,
    %pdfpagelabels,
    colorlinks=true,
    linktocpage=true,
    pdfstartpage=1,
    pdfstartview=,
    breaklinks=true,
    pdfpagemode=UseNone,
    pageanchor=true,
    pdfpagemode=UseOutlines,
    plainpages=false,
    bookmarksnumbered,
    bookmarksopen=true,
    bookmarksopenlevel=1,
    hypertexnames=true,
    pdfhighlight=/O,
    %nesting=true,
    %frenchlinks,
    urlcolor=webbrown,
    linkcolor=RoyalBlue,
    citecolor=webgreen
    %pagecolor=RoyalBlue,
}

% Delete all links and show them in black
\if \isprintable 1
    \hypersetup{draft}
\fi

% Listings setup
\lstset{
    language=[LaTeX]Tex,%C++,
    keywordstyle=\color{RoyalBlue}, %\bfseries,
    basicstyle=\small\ttfamily,
    %identifierstyle=\color{NavyBlue},
    commentstyle=\color{Green}\ttfamily,
    stringstyle=\rmfamily,
    numbers=none, %left,%
    numberstyle=\scriptsize, %\tiny
    stepnumber=5,
    numbersep=8pt,
    showstringspaces=false,
    breaklines=true,
    frameround=ftff,
    frame=single
}

\definecolor{webgreen}{rgb}{0,.5,0}
\definecolor{webbrown}{rgb}{.6,0,0}

\newcommand{\sectionname}{sezione}
\addto\captionsitalian{\renewcommand{\figurename}{Figura}
                       \renewcommand{\tablename}{Tabella}}

% Aggiunge il simbolo \glsfirstoccur dopo ogni occorrenza di \gls
\newcommand{\glsfirstoccur}{\textsubscript{\textbf{\textit{G}}}}
\let\oldgls\gls
\renewcommand{\gls}[1]{\oldgls{#1}\glsfirstoccur}

\newcommand{\intro}[1]{\emph{\textsf{#1}}}

% Risks environment
\newcounter{riskcounter}                % define a counter
\setcounter{riskcounter}{0}             % set the counter to some initial value

%%%% Parameters
% #1: Title
\newenvironment{risk}[1]{
    \refstepcounter{riskcounter}        % increment counter
    \par \noindent                      % start new paragraph
    \textbf{\arabic{riskcounter}. #1}   % display the title before the content of the environment is displayed
}{
    \par\medskip
}

\newcommand{\riskname}{Rischio}

\newcommand{\riskdescription}[1]{\textbf{\\Descrizione:} #1.}

\newcommand{\risksolution}[1]{\textbf{\\Soluzione:} #1.}

% Use case environment
\newcounter{usecasecounter}             % define a counter
\setcounter{usecasecounter}{0}          % set the counter to some initial value

%%%% Parameters
% #1: ID
% #2: Nome
\newenvironment{usecase}[2]{
    \renewcommand{\theusecasecounter}{\usecasename #1}  % this is where the display of
                                                        % the counter is overwritten/modified
    \refstepcounter{usecasecounter}             % increment counter
    \vspace{10pt}
    \par \noindent                              % start new paragraph
    {\large \textbf{\usecasename #1: #2}}       % display the title before the
                                                % content of the environment is displayed
    \medskip
}{
    \medskip
}

\newcommand{\usecasename}{UC}

\newcommand{\usecaseactors}[1]{\textbf{\\Attori Principali:} #1. \vspace{4pt}}
\newcommand{\usecasepre}[1]{\textbf{\\Precondizioni:} #1. \vspace{4pt}}
\newcommand{\usecasedesc}[1]{\textbf{\\Descrizione:} #1. \vspace{4pt}}
\newcommand{\usecasepost}[1]{\textbf{\\Postcondizioni:} #1. \vspace{4pt}}
\newcommand{\usecasealt}[1]{\textbf{\\Scenario Alternativo:} #1. \vspace{4pt}}
\newcommand{\useinclu}[1]{\textbf{\\Inclusioni:} #1. \vspace{4pt}}
\newcommand{\usespecial}[1]{\textbf{\\Specializzazioni:} #1. \vspace{4pt}}

% Namespace description environment
\newenvironment{namespacedesc}{
    \vspace{10pt}
    \par \noindent  % start new paragraph
    \begin{description}
}{
    \end{description}
    \medskip
}

\newcommand{\classdesc}[2]{\item[\textbf{#1:}] #2}




% Requirements table
\newcommand{\RequisitiTable}[3]{%
  \rowcolors{2}{gray!10}{white}
  \renewcommand{\arraystretch}{1.6}

  \begin{longtable}{|p{2cm}|p{8cm}|p{2cm}|}
  \caption{#1} \label{#2} \\

  \hline
  \rowcolor{gray!75}
  #3 \\ \hline
  \endfirsthead

  \hline
  \rowcolor{gray!75}
  #3 \\ \hline
  \endhead

  \multicolumn{3}{r}{\textit{Continua nella prossima pagina...}} \\
  \endfoot

  \endlastfoot
}
