\chapter{Descrizione dello stage}
\label{cap:descrizione-stage}

\intro{In questo capitolo viene presentata una descrizione dell’idea e delle tecnologie utilizzate durante il percorso di stage, e una panoramica del modo di lavorare dell'azienda. Viene inoltre presentata un'analisi dei rischi e delle problematiche riscontrate durante lo sviluppo del progetto.}\\

\section{Introduzione al progetto}

Nel contesto attuale, i report delle vendite e i sistemi di raccomandazione rivestono un ruolo fondamentale per le aziende che operano in mercati competitivi e digitalizzati. I report delle vendite consentono di monitorare l’andamento commerciale, identificare trend, valutare le performance dei prodotti e prendere decisioni strategiche basate su dati concreti. Questi strumenti permettono di individuare rapidamente eventuali criticità o opportunità di crescita, ottimizzando così le strategie di vendita e marketing.\\

Parallelamente, i sistemi di raccomandazione sono diventati essenziali per migliorare l’esperienza utente e incrementare le vendite, soprattutto nelle piattaforme di e-commerce e nei servizi digitali. Attraverso l’analisi dei dati di acquisto e delle preferenze degli utenti, questi sistemi suggeriscono prodotti o servizi personalizzati, aumentando la probabilità di acquisto e la fidelizzazione del cliente. L’integrazione di report delle vendite e sistemi di raccomandazione consente quindi alle aziende di offrire un servizio più mirato ed efficiente, rafforzando la propria posizione sul mercato.\\

In questo contesto, la startup Oribea, che propone soluzioni digitali intelligenti per le aziende, ha ricevuto una commissione da parte di un'azienda di e-commerce per la creazione di strumenti che automatizzino la generazione di report delle vendite e sistemi di raccomandazione. Oribea ha quindi deciso di integrare queste due funzionalità all’interno della propria piattaforma omonima, che consente di usufruire delle cosiddette Task: strumenti che permettono di automatizzare e semplificare specifiche attività aziendali quotidiane.\\
Il progetto di stage si inserisce in questo scenario, con l'obiettivo di sviluppare una Task che consenta di generare un report delle vendite e un sistema di raccomandazione a partire da un file degli ordini, e un’altra Task che permetta di sfruttare il sistema di raccomandazione per generare suggerimenti personalizzati per un determinato cliente o prodotto.\\

Le Task di Oribea sono implementate come funzioni serverless, ovvero funzioni che vengono eseguite in modo autonomo e scalabile, senza la necessità di gestire l'infrastruttura sottostante, e sono caricate ed eseguite su Google \gls{cloudfunctionsg}\textsubscript{\textbf{\textit{G}}}. Lo stato delle funzioni viene salvato in Google \gls{cloudstorageg}\textsubscript{\textbf{\textit{G}}}. Il compito di stage prevede quindi lo sviluppo di due Cloud Functions serverless che implementano le funzionalità richieste, la loro comunicazione tramite Cloud Storage e la loro integrazione con la piattaforma Oribea.\\


\section{Way of working e strumenti utilizzati}
\label{sec:way-of-working}

L’azienda Oribea adotta un modello di sviluppo \gls{agileg}\textsubscript{\textbf{\textit{G}}}, con l’obiettivo di monitorare e controllare il progetto in modo flessibile e continuo, suddividendo le attività in piccoli incrementi e con una collaborazione asincrona e distribuita.\\
In particolare, il modello di sviluppo adottato è \gls{scrumg}\textsubscript{\textbf{\textit{G}}}, che prevede la suddivisione del progetto in sprint, ovvero periodi di tempo di durata fissa, in cui vengono pianificate le attività da svolgere e i relativi obiettivi da raggiungere. All'inizio del progetto è stato scelto di comune accordo la lunghezza per gli sprint di una settimana. Al termine di ogni sprint, è stato effettuato un incontro approfondito con il tutor aziendale, per discutere lo stato di avanzamento del progetto e le attività da svolgere per il successivo sprint.\\
L’obiettivo del modello è dare maggiore importanza al ciclo di vita del software e dei processi correlati, piuttosto che al prodotto finale, con il fine ultimo di migliorare la qualità del prodotto stesso.\\

Gli strumenti principali utilizzati includono:
\begin{itemize}
    \item \textbf{Visual Studio Code}: per la scrittura e la modifica del codice sorgente;
    \item \textbf{Git}: per il versionamento del codice e la gestione delle modifiche;
    \item \textbf{GitHub}: per la gestione del codice sorgente e la collaborazione tra sviluppatori;
    \item \textbf{Slack}: per la comunicazione interna e la gestione dei progetti;
    \item \textbf{Notion}: per la documentazione del progetto;
    \item \textbf{StarUML}: per la creazione dei diagrammi UML.
\end{itemize}

\section{Analisi preventiva dei rischi}

Durante la fase di analisi iniziale sono stati individuati alcuni possibili rischi a cui si potrà andare incontro.
Si è quindi proceduto a elaborare delle possibili soluzioni per far fronte a tali rischi.\\

\begin{risk}{Cambiamento dei requisiti in corso di sviluppo}
    \riskdescription{i requisiti del progetto potrebbero cambiare durante lo sviluppo dello stesso. Alcuni dei motivi potrebbero essere delle modifiche alle esigenze del progetto, un errore di analisi iniziale oppure motivi legati alle tecnologie scelte}
    \risksolution{è utile adottare l’utilizzo di una metodologia agile per tollerare i cambiamenti ai requisiti. È inoltre di fondamentale importanza mantenere una comunicazione costante con il tutor aziendale, in modo da procedere con lo sviluppo del progetto con maggior sicurezza e in modo più organizzato}
    \label{risk:requirements-change} 
\end{risk}

\begin{risk}{Numero e qualità dei dataset di addestramento}
    \riskdescription{per il corretto funzionamento del sistema di raccomandazione è necessario disporre non solo di dataset di qualità, ma anche di una quantità sufficiente di dataset di addestramento che rappresentino diversi scenari e casi d’uso. L’utilizzo di un solo dataset, anche se ben strutturato, potrebbe non essere sufficiente per addestrare il sistema in modo efficace e garantire risultati affidabili e generalizzabili. Oribea ha fornito un dataset di esempio, ma la presenza di un unico dataset limita la capacità del sistema di adattarsi a situazioni diverse e può ridurre la robustezza delle raccomandazioni generate}
    \risksolution{è possibile utilizzare tecniche di data augmentation per aumentare la varietà e la quantità dei dati disponibili, oppure ricorrere a dataset pubblici pertinenti al dominio del progetto. Inoltre, si può valutare la raccolta di ulteriori dataset reali, anche in collaborazione con l’azienda committente, per coprire un maggior numero di casi d’uso e migliorare la generalizzazione del sistema di raccomandazione}
    \label{risk:dataset-absence}
\end{risk}

\begin{risk}{Risposta dell'LLM imprecisa}
    \riskdescription{l'LLM (Large Language Model) utilizzato per generare il report delle vendite potrebbe produrre risposte imprecise o non pertinenti, a causa di una cattiva interpretazione del file degli ordini o di una mancanza di contesto. Questo potrebbe portare a report errati o fuorvianti, che potrebbero influenzare negativamente le decisioni aziendali. Inoltre, nell'ambito del preprocessing dei dati, l'LLM potrebbe non essere in grado di riconoscere correttamente le colonne da etichettare semanticamente, causando errori nella generazione del report e del sistema di raccomandazione}
    \risksolution{è importante testare l'LLM con diversi esempi di file degli ordini e verificare la qualità delle risposte generate. In caso di risposte imprecise, è possibile migliorare il prompt utilizzato per l'LLM, fornendo istruzioni più chiare e dettagliate. Inoltre, è utile implementare un sistema di validazione dei dati in ingresso, per garantire che il file degli ordini sia formattato correttamente e contenga le informazioni necessarie. Infine, si può considerare l'utilizzo di tecniche di post-elaborazione per correggere eventuali errori nelle risposte dell'LLM}
    \label{risk:bad-llm-response}
\end{risk}

\begin{risk}{Ritardi nello sviluppo}
    \riskdescription{durante lo sviluppo del progetto potrebbero verificarsi imprevisti come problemi tecnici o un’eccessiva complessità del tema affrontato, che potrebbero portare a ritardi nello sviluppo del prodotto}
    \risksolution{è importante pianificare il progetto in modo realistico, tenendo conto dei tempi necessari per lo sviluppo e per la risoluzione di eventuali problemi. Inoltre, è utile monitorare costantemente lo stato di avanzamento del progetto e adottare una metodologia Agile per gestire i cambiamenti e le difficoltà che potrebbero sorgere. In caso di ritardi significativi, è possibile rivedere le priorità del progetto e concentrarsi sulle funzionalità essenziali, rimandando eventuali miglioramenti o funzionalità aggiuntive a fasi successive}
    \label{risk:development-delay}
\end{risk}

\begin{risk}{Inesperienza tecnologica}
    \riskdescription{il progetto prevede l’utilizzo di tecnologie di cui non si ha piena esperienza e conoscenza, soprattutto le tecnologie legate al \gls{cloudcomputingg}\textsubscript{\textbf{\textit{G}}}, rendendo più difficoltosa la comprensione e l’applicazione delle stesse in fase di implementazione}
    \risksolution{è utile dedicare del tempo all'apprendimento delle tecnologie utilizzate, consultando la documentazione ufficiale, seguendo corsi online o chiedendo supporto a colleghi più esperti. Inoltre, è importante non esitare a chiedere aiuto al tutor aziendale in caso di difficoltà, in modo da risolvere i problemi in modo tempestivo e ridurre al minimo l'impatto sull'avanzamento del progetto. Infine, si può considerare l'utilizzo di risorse online come forum e community per ottenere supporto e condividere esperienze con altri sviluppatori}
    \label{risk:inexperience}
\end{risk}
