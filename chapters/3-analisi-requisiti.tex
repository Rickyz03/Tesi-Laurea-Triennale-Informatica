\chapter{Analisi dei requisiti}
\label{cap:analisi-requisiti}

\intro{In questo capitolo vengono analizzati gli obiettivi del progetto e ne viene data un’analisi ad alto livello, combinando una visione concettuale con una visione pratica ed implementativa. Vengono inoltre descritti i casi d’uso e i requisiti individuati, con l’obiettivo di fornire una visione generale del sistema e delle sue funzionalità.}\\


\section{Obiettivi dello stage}
\label{sec:obiettivi-stage}

Gli obiettivi fondamentali da raggiungere durante il periodo di tirocinio, stilati
in accordo con il tutor aziendale ed inseriti nel documento \emph{Piano di Lavoro}, sono
identificati dalla seguente notazione:
\begin{itemize}
    \item \textbf{OO}: obiettivi obbligatori, vincolanti in quanto obiettivi primari richiesti dal committente;
    \item \textbf{OD}: obiettivi desiderabili, non vincolanti o strettamente necessari, ma dal riconoscibile valore aggiunto;
    \item \textbf{OZ}: obiettivi opzionali, non vincolanti e non necessari, cioè che potrebbero essere implementati in un secondo momento.
\end{itemize}

Alle sigle precedentemente indicate seguirà un numero progressivo, identificando così
tutti gli obiettivi.

Essi sono i seguenti:
\begin{itemize}
    \item Obbligatori:
    \begin{itemize}
        \item \textbf{OO1}: acquisizione di competenze pratiche su Oribea/DialogSphere;
        \item \textbf{OO2}: connessione a database e gestione dati aziendali o pubblici;
        \item \textbf{OO3}: implementazione di una Task AI che genera un sistema di raccomandazioni e report automatico basato su analisi delle vendite;
        \item \textbf{OO4}: implementazione di una Task AI che permette di raccomandare prodotti ad un cliente in base ai suoi dati di vendita, e viceversa di raccomandare clienti ad un prodotto;
        \item \textbf{OO5}: generazione automatica di report con output coerente, chiaro e adattabile;
        \item \textbf{OO6}: testing e documentazione completa del prototipo.
    \end{itemize}
    \item Desiderabili:
    \begin{itemize}
        \item \textbf{OD1}: ottimizzazione del Task AI per performance e scalabilità;
        \item \textbf{OD2}: personalizzazione dinamica dei prompt per casi d’uso differenti;
        \item \textbf{OD3}: integrazione con strumenti di visualizzazione o interfacce utente.
    \end{itemize}
    \item Opzionali:
    \begin{itemize}
        \item \textbf{OZ1}: sviluppo di un chatbot o di una dashboard interattiva per l'interazione con il sistema di raccomandazioni e report;
        \item \textbf{OZ2}: sperimentazione di tecniche di Explainable AI (XAI) per la trasparenza dei risultati;
        \item \textbf{OZ3}: esportazione automatica dei report in PDF/HTML o invio via e-mail.
    \end{itemize}
\end{itemize}


\newpage

\section{Casi d'uso}

Per lo studio dei casi di utilizzo delle due task sono stati creati dei diagrammi.

I diagrammi dei casi d'uso sono diagrammi di tipo \gls{uml} dedicati alla descrizione delle funzioni o servizi offerti da un sistema, così come sono percepiti e utilizzati dagli attori che interagiscono col sistema stesso. Nel mio caso, l'unico attore che interagisce con le due task è l'utente semplice, che rappresenta un generico utente autenticato nella piattaforma Oribea.

Ciascun caso d’uso riporta gli attori coinvolti, le sue precondizioni, la sua descrizione, le sue postcondizioni ed eventuali sottocasi d’uso, inclusioni, specializzazioni e scenari alternativi.

I casi d’uso che sono stati definiti sono i seguenti:


\hypertarget{UC1}{}
\begin{usecase}{1}{Caricamento file CSV}

\begin{figure}[!h] 
    \centering 
    \includegraphics[width=0.9\columnwidth]{usecase/UC1 - Caricamento file CSV.png}
    \caption{UC1 - Caricamento file CSV}
\end{figure}

\usecaseactors{Utente}
\usecasepre{L'utente ha avviato la task di analisi delle vendite}
\usecasedesc{L'utente carica un file \gls{csv} contenente i dati delle vendite}
\usecasepost{Il sistema ha caricato il file CSV e lo ha memorizzato in un bucket del \gls{googlecloudstorage}}
\label{uc:caricamento-file-csv}
\end{usecase}


\hypertarget{UC2}{}
\begin{usecase}{2}{Selezione lingua per il report}

\begin{figure}[!h] 
    \centering 
    \includegraphics[width=0.9\columnwidth]{usecase/UC2 - Selezione lingua per il report.png}
    \caption{UC2 - Selezione lingua per il report}
\end{figure}

\usecaseactors{Utente}
\usecasepre{L'utente ha avviato la task di analisi delle vendite}
\usecasedesc{L'utente seleziona la lingua in cui desidera generare il report}
\usecasepost{Il sistema ha memorizzato la lingua selezionata}
\usespecial{UC3, UC4, UC5, UC6, UC7}
\label{uc:selezione-lingua-report}
\end{usecase}


\hypertarget{UC3}{}
\begin{usecase}{3}{Selezione lingua italiano per il report}

\usecaseactors{Utente}
\usecasepre{L'utente ha avviato la task di analisi delle vendite}
\usecasedesc{L'utente seleziona la lingua italiano perchè desidera generare il report in italiano}
\usecasepost{Il sistema ha memorizzato la lingua italiano selezionata}
\label{uc:selezione-lingua-italiano-report}
\end{usecase}


\hypertarget{UC4}{}
\begin{usecase}{4}{Selezione lingua inglese per il report}

\usecaseactors{Utente}
\usecasepre{L'utente ha avviato la task di analisi delle vendite}
\usecasedesc{L'utente seleziona la lingua inglese perchè desidera generare il report in inglese}
\usecasepost{Il sistema ha memorizzato la lingua inglese selezionata}
\label{uc:selezione-lingua-inglese-report}
\end{usecase}


\hypertarget{UC5}{}
\begin{usecase}{5}{Selezione lingua francese per il report}

\usecaseactors{Utente}
\usecasepre{L'utente ha avviato la task di analisi delle vendite}
\usecasedesc{L'utente seleziona la lingua francese perchè desidera generare il report in francese}
\usecasepost{Il sistema ha memorizzato la lingua francese selezionata}
\label{uc:selezione-lingua-francese-report}
\end{usecase}


\hypertarget{UC6}{}
\begin{usecase}{6}{Selezione lingua spagnolo per il report}

\usecaseactors{Utente}
\usecasepre{L'utente ha avviato la task di analisi delle vendite}
\usecasedesc{L'utente seleziona la lingua spagnolo perchè desidera generare il report in spagnolo}
\usecasepost{Il sistema ha memorizzato la lingua spagnolo selezionata}
\label{uc:selezione-lingua-spagnolo-report}
\end{usecase}


\hypertarget{UC7}{}
\begin{usecase}{7}{Selezione lingua tedesco per il report}

\usecaseactors{Utente}
\usecasepre{L'utente ha avviato la task di analisi delle vendite}
\usecasedesc{L'utente seleziona la lingua tedesco perchè desidera generare il report in tedesco}
\usecasepost{Il sistema ha memorizzato la lingua tedesco selezionata}
\label{uc:selezione-lingua-tedesco-report}
\end{usecase}


\hypertarget{UC8}{}
\begin{usecase}{8}{Selezione valuta per il report}

\begin{figure}[!h] 
    \centering 
    \includegraphics[width=0.9\columnwidth]{usecase/UC8 - Selezione valuta per il report.png}
    \caption{UC8 - Selezione valuta per il report}
\end{figure}

\usecaseactors{Utente}
\usecasepre{L'utente ha avviato la task di analisi delle vendite}
\usecasedesc{L'utente seleziona la valuta in cui desidera generare il report}
\usecasepost{Il sistema ha memorizzato la valuta selezionata}
\usespecial{UC9, UC10, UC11}
\label{uc:selezione-valuta-report}
\end{usecase}


\hypertarget{UC9}{}
\begin{usecase}{9}{Selezione valuta euro per il report}

\usecaseactors{Utente}
\usecasepre{L'utente ha avviato la task di analisi delle vendite}
\usecasedesc{L'utente seleziona la valuta euro perchè desidera generare il report con i valori monetari espressi in euro}
\usecasepost{Il sistema ha memorizzato la valuta euro selezionata}
\label{uc:selezione-valuta-euro-report}
\end{usecase}


\hypertarget{UC10}{}
\begin{usecase}{10}{Selezione valuta dollaro per il report}

\usecaseactors{Utente}
\usecasepre{L'utente ha avviato la task di analisi delle vendite}
\usecasedesc{L'utente seleziona la valuta dollaro perchè desidera generare il report con i valori monetari espressi in dollari}
\usecasepost{Il sistema ha memorizzato la valuta dollaro selezionata}
\label{uc:selezione-valuta-dollaro-report}
\end{usecase}


\hypertarget{UC11}{}
\begin{usecase}{11}{Selezione valuta sterlina per il report}

\usecaseactors{Utente}
\usecasepre{L'utente ha avviato la task di analisi delle vendite}
\usecasedesc{L'utente seleziona la valuta sterlina perchè desidera generare il report con i valori monetari espressi in sterline}
\usecasepost{Il sistema ha memorizzato la valuta sterlina selezionata}
\label{uc:selezione-valuta-sterlina-report}
\end{usecase}


\hypertarget{UC12}{}
\begin{usecase}{12}{Inserimento indirizzo email a cui inviare il report}

\begin{figure}[!h]
    \centering 
    \includegraphics[width=0.9\columnwidth]{usecase/UC12 - Inserimento indirizzo email a cui inviare il report.png}
    \caption{UC12 - Inserimento indirizzo email a cui inviare il report}
\end{figure}

\usecaseactors{Utente}
\usecasepre{L'utente ha avviato la task di analisi delle vendite}
\usecasedesc{L'utente inserisce l'indirizzo email a cui desidera che venga inviato il report generato}
\usecasepost{Il sistema ha memorizzato l'indirizzo email inserito}

\end{usecase}


\hypertarget{UC13}{}
\begin{usecase}{13}{Visualizzazione esito positivo della task}

\begin{figure}[!h]
    \centering 
    \includegraphics[width=0.9\columnwidth]{usecase/UC13 - Visualizzazione esito positivo della task.png}
    \caption{UC13 - Visualizzazione esito positivo della task}
\end{figure}

\usecaseactors{Utente}
\usecasepre{L'utente ha cliccato sul pulsante di esecuzione della task di analisi delle vendite}
\usecasedesc{L'utente visualizza l'esito positivo della task di analisi delle vendite}
\usecasepost{L'utente ha visualizzato l'esito positivo della task di analisi delle vendite}
\usecasealt{UC14}
\label{uc:visualizzazione-esito-positivo-task}
\end{usecase}


\hypertarget{UC13.1}{}
\begin{usecase}{13.1}{Visualizzazione del token identificativo del file degli ordini}

\usecaseactors{Utente}
\usecasepre{L'utente ha cliccato sul pulsante di esecuzione della task di analisi delle vendite}
\usecasedesc{L'utente visualizza il token identificativo del file degli ordini caricato}
\usecasepost{L'utente ha visualizzato il token identificativo del file degli ordini caricato}
\label{uc:visualizzazione-token-file-ordini}
\end{usecase}


\hypertarget{UC14}{}
\begin{usecase}{14}{Visualizzazione errore nell'elaborazione della task}

\usecaseactors{Utente}
\usecasepre{L'utente ha cliccato sul pulsante di esecuzione della task di analisi delle vendite}
\usecasedesc{L'utente visualizza un errore nell'elaborazione della task di analisi delle vendite}
\usecasepost{L'utente ha visualizzato un errore nell'elaborazione della task di analisi delle vendite}
\label{uc:visualizzazione-errore-task}
\end{usecase}


\hypertarget{UC15}{}
\begin{usecase}{15}{Download report delle vendite}

\begin{figure}[!h] 
    \centering 
    \includegraphics[width=0.9\columnwidth]{usecase/UC15 - Download report delle vendite.png}
    \caption{UC15 - Download report delle vendite}
\end{figure}

\usecaseactors{Utente}
\usecasepre{L'utente ha cliccato sul pulsante di esecuzione della task di analisi delle vendite}
\usecasedesc{L'utente scarica il report delle vendite generato dalla task di analisi delle vendite}
\usecasepost{L'utente ha scaricato il report delle vendite generato dalla task di analisi delle vendite}
\usespecial{UC16, UC17}
\label{uc:download-report-vendite}
\end{usecase}


\hypertarget{UC16}{}
\begin{usecase}{16}{Download report delle vendite in PDF}

\usecaseactors{Utente}
\usecasepre{L'utente ha cliccato sul pulsante di esecuzione della task di analisi delle vendite}
\usecasedesc{L'utente scarica il report delle vendite generato dalla task di analisi delle vendite in formato PDF}
\usecasepost{L'utente ha scaricato il report delle vendite generato dalla task di analisi delle vendite in formato PDF}
\label{uc:download-report-vendite-pdf}
\end{usecase}


\hypertarget{UC17}{}
\begin{usecase}{17}{Download report delle vendite in HTML}

\usecaseactors{Utente}
\usecasepre{L'utente ha cliccato sul pulsante di esecuzione della task di analisi delle vendite}
\usecasedesc{L'utente scarica il report delle vendite generato dalla task di analisi delle vendite in formato HTML}
\usecasepost{L'utente ha scaricato il report delle vendite generato dalla task di analisi delle vendite in formato HTML}
\label{uc:download-report-vendite-html}
\end{usecase}


\hypertarget{UC18}{}
\begin{usecase}{18}{Visualizzazione di un'email contenente il report delle vendite}

\begin{figure}[!h] 
    \centering 
    \includegraphics[width=0.9\columnwidth]{usecase/UC18 - Visualizzazione di un'email contenente il report delle vendite.png}
    \caption{UC18 - Visualizzazione di un'email contenente il report delle vendite}
\end{figure}

\usecaseactors{Utente}
\usecasepre{L'utente ha cliccato sul pulsante di esecuzione della task di analisi delle vendite, e poi ha visitato la propria casella di posta elettronica e cliccato sull'email ricevuta contenente il report delle vendite}
\usecasedesc{L'utente visualizza un'email contenente il report delle vendite generato dalla task di analisi delle vendite}
\usecasepost{L'utente ha visualizzato un'email contenente il report delle vendite generato dalla task di analisi delle vendite}
\useinclu{UC18.1, UC18.2, UC18.3}
\label{uc:visualizzazione-email-report-vendite}
\end{usecase}


\hypertarget{UC18.1}{}
\begin{usecase}{18.1}{Visualizzazione dati numerici}

\begin{figure}[!h] 
    \centering 
    \includegraphics[width=0.9\columnwidth]{usecase/UC18.1 - Visualizzazione dati numerici.png}
    \caption{UC18.1 - Visualizzazione dati numerici}
\end{figure}

\usecaseactors{Utente}
\usecasepre{L'utente ha cliccato sul pulsante di esecuzione della task di analisi delle vendite, e poi ha visitato la propria casella di posta elettronica e cliccato sull'email ricevuta contenente il report delle vendite}
\usecasedesc{L'utente visualizza i dati numerici contenuti nel report delle vendite generato dalla task di analisi delle vendite}
\usecasepost{L'utente ha visualizzato i dati numerici contenuti nel report delle vendite generato dalla task di analisi delle vendite}
\useinclu{18.1.1, 18.1.2, 18.1.3, 18.1.4, 18.1.5}
\label{uc:visualizzazione-dati-numerici}
\end{usecase}


\hypertarget{UC18.1.1}{}
\begin{usecase}{18.1.1}{Visualizzazione fatturato totale}

\usecaseactors{Utente}
\usecasepre{L'utente ha cliccato sul pulsante di esecuzione della task di analisi delle vendite, e poi ha visitato la propria casella di posta elettronica e cliccato sull'email ricevuta contenente il report delle vendite}
\usecasedesc{L'utente visualizza il fatturato totale contenuto nel report delle vendite generato dalla task di analisi delle vendite}
\usecasepost{L'utente ha visualizzato il fatturato totale contenuto nel report delle vendite generato dalla task di analisi delle vendite}
\label{uc:visualizzazione-fatturato-totale}
\end{usecase}


\hypertarget{UC18.1.2}{}
\begin{usecase}{18.1.2}{Visualizzazione ordini unici}

\usecaseactors{Utente}
\usecasepre{L'utente ha cliccato sul pulsante di esecuzione della task di analisi delle vendite, e poi ha visitato la propria casella di posta elettronica e cliccato sull'email ricevuta contenente il report delle vendite}
\usecasedesc{L'utente visualizza il numero di ordini unici contenuti nel report delle vendite generato dalla task di analisi delle vendite}
\usecasepost{L'utente ha visualizzato il numero di ordini unici contenuti nel report delle vendite generato dalla task di analisi delle vendite}
\label{uc:visualizzazione-ordini-unici}
\end{usecase}


\hypertarget{UC18.1.3}{}
\begin{usecase}{18.1.3}{Visualizzazione clienti unici}

\usecaseactors{Utente}
\usecasepre{L'utente ha cliccato sul pulsante di esecuzione della task di analisi delle vendite, e poi ha visitato la propria casella di posta elettronica e cliccato sull'email ricevuta contenente il report delle vendite}
\usecasedesc{L'utente visualizza il numero di clienti unici contenuti nel report delle vendite generato dalla task di analisi delle vendite}
\usecasepost{L'utente ha visualizzato il numero di clienti unici contenuti nel report delle vendite generato dalla task di analisi delle vendite}
\label{uc:visualizzazione-clienti-unici}
\end{usecase}


\hypertarget{UC18.1.4}{}
\begin{usecase}{18.1.4}{Visualizzazione prodotti unici}

\usecaseactors{Utente}
\usecasepre{L'utente ha cliccato sul pulsante di esecuzione della task di analisi delle vendite, e poi ha visitato la propria casella di posta elettronica e cliccato sull'email ricevuta contenente il report delle vendite}
\usecasedesc{L'utente visualizza il numero di prodotti unici contenuti nel report delle vendite generato dalla task di analisi delle vendite}
\usecasepost{L'utente ha visualizzato il numero di prodotti unici contenuti nel report delle vendite generato dalla task di analisi delle vendite}
\label{uc:visualizzazione-prodotti-unici}
\end{usecase}


\hypertarget{UC18.1.5}{}
\begin{usecase}{18.1.5}{Visualizzazione spesa media per ordine}

\usecaseactors{Utente}
\usecasepre{L'utente ha cliccato sul pulsante di esecuzione della task di analisi delle vendite, e poi ha visitato la propria casella di posta elettronica e cliccato sull'email ricevuta contenente il report delle vendite}
\usecasedesc{L'utente visualizza la spesa media per ordine contenuta nel report delle vendite generato dalla task di analisi delle vendite}
\usecasepost{L'utente ha visualizzato la spesa media per ordine contenuta nel report delle vendite generato dalla task di analisi delle vendite}
\label{uc:visualizzazione-spesa-media-per-ordine}
\end{usecase}


\hypertarget{UC18.2}{}
\begin{usecase}{18.2}{Visualizzazione lista di grafici}

\begin{figure}[!h] 
    \centering 
    \includegraphics[width=0.9\columnwidth]{usecase/UC18.2 - Visualizzazione lista di grafici.png}
    \caption{UC18.2 - Visualizzazione lista di grafici}
\end{figure}

\usecaseactors{Utente}
\usecasepre{L'utente ha cliccato sul pulsante di esecuzione della task di analisi delle vendite, e poi ha visitato la propria casella di posta elettronica e cliccato sull'email ricevuta contenente il report delle vendite}
\usecasedesc{L'utente visualizza la lista di grafici contenuti nel report delle vendite generato dalla task di analisi delle vendite}
\usecasepost{L'utente ha visualizzato la lista di grafici contenuti nel report delle vendite generato dalla task di analisi delle vendite}
\useinclu{18.2.1, 18.2.2, 18.2.3, 18.2.4, 18.2.5, 18.2.6, 18.2.7}
\label{uc:visualizzazione-lista-grafici}
\end{usecase}


\hypertarget{UC18.2.1}{}
\begin{usecase}{18.2.1}{Visualizzazione grafico a linea del trend mensile delle vendite}

\usecaseactors{Utente}
\usecasepre{L'utente ha cliccato sul pulsante di esecuzione della task di analisi delle vendite, e poi ha visitato la propria casella di posta elettronica e cliccato sull'email ricevuta contenente il report delle vendite}
\usecasedesc{L'utente visualizza il grafico a linea del trend mensile delle vendite contenuto nel report delle vendite generato dalla task di analisi delle vendite}
\usecasepost{L'utente ha visualizzato il grafico a linea del trend mensile delle vendite contenuto nel report delle vendite generato dalla task di analisi delle vendite}
\label{uc:visualizzazione-grafico-trend-mensile-vendite}
\end{usecase}


\hypertarget{UC18.2.2}{}
\begin{usecase}{18.2.2}{Visualizzazione grafico a barre dei top 5 prodotti per fatturato}

\usecaseactors{Utente}
\usecasepre{L'utente ha cliccato sul pulsante di esecuzione della task di analisi delle vendite, e poi ha visitato la propria casella di posta elettronica e cliccato sull'email ricevuta contenente il report delle vendite}
\usecasedesc{L'utente visualizza il grafico a barre dei top 5 prodotti per fatturato contenuto nel report delle vendite generato dalla task di analisi delle vendite}
\usecasepost{L'utente ha visualizzato il grafico a barre dei top 5 prodotti per fatturato contenuto nel report delle vendite generato dalla task di analisi delle vendite}
\label{uc:visualizzazione-grafico-top-5-prodotti-fatturato}
\end{usecase}


\hypertarget{UC18.2.3}{}
\begin{usecase}{18.2.3}{Visualizzazione grafico a barre dei top 5 clienti per spesa totale}

\usecaseactors{Utente}
\usecasepre{L'utente ha cliccato sul pulsante di esecuzione della task di analisi delle vendite, e poi ha visitato la propria casella di posta elettronica e cliccato sull'email ricevuta contenente il report delle vendite}
\usecasedesc{L'utente visualizza il grafico a barre dei top 5 clienti per spesa totale contenuto nel report delle vendite generato dalla task di analisi delle vendite}
\usecasepost{L'utente ha visualizzato il grafico a barre dei top 5 clienti per spesa totale contenuto nel report delle vendite generato dalla task di analisi delle vendite}
\label{uc:visualizzazione-grafico-top-5-clienti-spesa-totale}
\end{usecase}


\hypertarget{UC18.2.4}{}
\begin{usecase}{18.2.4}{Visualizzazione grafico a linea del trend mensile dei nuovi clienti}

\usecaseactors{Utente}
\usecasepre{L'utente ha cliccato sul pulsante di esecuzione della task di analisi delle vendite, e poi ha visitato la propria casella di posta elettronica e cliccato sull'email ricevuta contenente il report delle vendite}
\usecasedesc{L'utente visualizza il grafico a linea del trend mensile dei nuovi clienti contenuto nel report delle vendite generato dalla task di analisi delle vendite}
\usecasepost{L'utente ha visualizzato il grafico a linea del trend mensile dei nuovi clienti contenuto nel report delle vendite generato dalla task di analisi delle vendite}
\label{uc:visualizzazione-grafico-trend-mensile-nuovi-clienti}
\end{usecase}


\hypertarget{UC18.2.5}{}
\begin{usecase}{18.2.5}{Visualizzazione del grafico a linea del trend mensile della percentuale dei nuovi clienti}

\usecaseactors{Utente}
\usecasepre{L'utente ha cliccato sul pulsante di esecuzione della task di analisi delle vendite, e poi ha visitato la propria casella di posta elettronica e cliccato sull'email ricevuta contenente il report delle vendite}
\usecasedesc{L'utente visualizza il grafico a linea del trend mensile della percentuale dei nuovi clienti contenuto nel report delle vendite generato dalla task di analisi delle vendite}
\usecasepost{L'utente ha visualizzato il grafico a linea del trend mensile della percentuale dei nuovi clienti contenuto nel report delle vendite generato dalla task di analisi delle vendite}
\label{uc:visualizzazione-grafico-trend-mensile-percentuale-nuovi-clienti}
\end{usecase}


\hypertarget{UC18.2.6}{}
\begin{usecase}{18.2.6}{Visualizzazione del grafico a barre delle top 5 date per fatturato}

\usecaseactors{Utente}
\usecasepre{L'utente ha cliccato sul pulsante di esecuzione della task di analisi delle vendite, e poi ha visitato la propria casella di posta elettronica e cliccato sull'email ricevuta contenente il report delle vendite}
\usecasedesc{L'utente visualizza il grafico a barre delle top 5 date per fatturato contenuto nel report delle vendite generato dalla task di analisi delle vendite}
\usecasepost{L'utente ha visualizzato il grafico a barre delle top 5 date per fatturato contenuto nel report delle vendite generato dalla task di analisi delle vendite}
\label{uc:visualizzazione-grafico-top-5-date-fatturato}
\end{usecase}


\hypertarget{UC18.2.7}{}
\begin{usecase}{18.2.7}{Visualizzazione del grafico a barre delle top 5 settimane per fatturato}

\usecaseactors{Utente}
\usecasepre{L'utente ha cliccato sul pulsante di esecuzione della task di analisi delle vendite, e poi ha visitato la propria casella di posta elettronica e cliccato sull'email ricevuta contenente il report delle vendite}
\usecasedesc{L'utente visualizza il grafico a barre delle top 5 settimane per fatturato contenuto nel report delle vendite generato dalla task di analisi delle vendite}
\usecasepost{L'utente ha visualizzato il grafico a barre delle top 5 settimane per fatturato contenuto nel report delle vendite generato dalla task di analisi delle vendite}
\label{uc:visualizzazione-grafico-top-5-settimane-fatturato}
\end{usecase}


\hypertarget{UC18.3}{}
\begin{usecase}{18.3}{Visualizzazione resoconto dell'analisi delle vendite}

\begin{figure}[!h] 
    \centering 
    \includegraphics[width=0.9\columnwidth]{usecase/UC18.3 - Visualizzazione resoconto dell'analisi delle vendite.png}
    \caption{UC18.3 - Visualizzazione resoconto dell'analisi delle vendite}
\end{figure}

\usecaseactors{Utente}
\usecasepre{L'utente ha cliccato sul pulsante di esecuzione della task di analisi delle vendite, e poi ha visitato la propria casella di posta elettronica e cliccato sull'email ricevuta contenente il report delle vendite}
\usecasedesc{L'utente visualizza il resoconto dell'analisi delle vendite contenuto nel report delle vendite generato dalla task di analisi delle vendite}
\usecasepost{L'utente ha visualizzato il resoconto dell'analisi delle vendite contenuto nel report delle vendite generato dalla task di analisi delle vendite}
\label{uc:visualizzazione-resoconto-analisi-vendite}
\end{usecase}


\hypertarget{UC18.4}{}
\begin{usecase}{18.4}{Visualizzazione del token identificativo del file degli ordini}

\begin{figure}[!h] 
    \centering 
    \includegraphics[width=0.9\columnwidth]{usecase/UC18.4 - Visualizzazione del token identificativo del file degli ordini.png}
    \caption{UC18.4 - Visualizzazione del token identificativo del file degli ordini}
\end{figure}

\usecaseactors{Utente}
\usecasepre{L'utente ha cliccato sul pulsante di esecuzione della task di analisi delle vendite, e poi ha visitato la propria casella di posta elettronica e cliccato sull'email ricevuta contenente il report delle vendite}
\usecasedesc{L'utente visualizza il token identificativo del file degli ordini caricato}
\usecasepost{L'utente ha visualizzato il token identificativo del file degli ordini caricato}
\label{uc:visualizzazione-token-file-ordini-mail}
\end{usecase}


\hypertarget{UC19}{}
\begin{usecase}{19}{Inserimento del token identificativo del file degli ordini}

\begin{figure}[!h] 
    \centering 
    \includegraphics[width=0.9\columnwidth]{usecase/UC19 - Inserimento del token identificativo del file degli ordini.png}
    \caption{UC19 - Inserimento del token identificativo del file degli ordini}
\end{figure}

\usecaseactors{Utente}
\usecasepre{L'utente ha eseguito con successo la task di analisi delle vendite e poi ha avviato la task di raccomandazione di prodotti e clienti}
\usecasedesc{L'utente inserisce il token identificativo del file degli ordini, che era stato generato dalla task di analisi delle vendite}
\usecasepost{Il sistema ricerca il bucket corrispondente al token identificativo all'interno del Google Cloud Storage}
\label{uc:inserimento-nome-file-csv}
\end{usecase}


\hypertarget{UC20}{}
\begin{usecase}{20}{Selezione tipo di raccomandazione}

\begin{figure}[!h] 
    \centering 
    \includegraphics[width=0.9\columnwidth]{usecase/UC20 - Selezione tipo di raccomandazione.png}
    \caption{UC20 - Selezione tipo di raccomandazione}
\end{figure}

\usecaseactors{Utente}
\usecasepre{L'utente ha eseguito con successo la task di analisi delle vendite e poi ha avviato la task di raccomandazione di prodotti e clienti}
\usecasedesc{L'utente seleziona la tipologia di raccomandazione che desidera che venga eseguita, tra le seguenti: "Raccomandare prodotti per un cliente" o "Raccomandare clienti per un prodotto"}
\usecasepost{Il sistema ha memorizzato il tipo di raccomandazione selezionato}
\usespecial{UC21, UC22}
\label{uc:selezione-tipo-elemento}
\end{usecase}


\hypertarget{UC21}{}
\begin{usecase}{21}{Selezione di raccomandare prodotti per un cliente}

\usecaseactors{Utente}
\usecasepre{L'utente ha eseguito con successo la task di analisi delle vendite e poi ha avviato la task di raccomandazione di prodotti e clienti}
\usecasedesc{L'utente seleziona il tipo di raccomandazione "Raccomandare prodotti per un cliente"}
\usecasepost{Il sistema ha memorizzato il tipo di raccomandazione "Raccomandare prodotti per un cliente" selezionato}
\label{uc:selezione-raccomandare-prodotti}
\end{usecase}


\hypertarget{UC22}{}
\begin{usecase}{22}{Selezione di raccomandare clienti per un prodotto}

\usecaseactors{Utente}
\usecasepre{L'utente ha eseguito con successo la task di analisi delle vendite e poi ha avviato la task di raccomandazione di prodotti e clienti}
\usecasedesc{L'utente seleziona il tipo di raccomandazione "Raccomandare clienti per un prodotto"}
\usecasepost{Il sistema ha memorizzato il tipo di raccomandazione "Raccomandare clienti per un prodotto" selezionato}
\label{uc:selezione-raccomandare-clienti}
\end{usecase}


\hypertarget{UC23}{}
\begin{usecase}{23}{Inserimento nome elemento a cui raccomandare}

\begin{figure}[!h] 
    \centering 
    \includegraphics[width=0.9\columnwidth]{usecase/UC23 - Inserimento nome elemento a cui raccomandare.png}
    \caption{UC23 - Inserimento nome elemento a cui raccomandare}
\end{figure}

\usecaseactors{Utente}
\usecasepre{L'utente ha eseguito con successo la task di analisi delle vendite e poi ha avviato la task di raccomandazione di prodotti e clienti}
\usecasedesc{L'utente inserisce il nome dell'elemento a cui desidera che vengano raccomandati prodotti o clienti, a seconda del tipo di raccomandazione selezionato}
\usecasepost{Il sistema cerca l'elemento corrispondente al nome inserito all'interno del file salvato nel Google Cloud Storage}
\label{uc:inserimento-nome-elemento}
\end{usecase}


\hypertarget{UC24}{}
\begin{usecase}{24}{Inserimento del numero di raccomandazioni desiderate}

\begin{figure}[!h] 
    \centering 
    \includegraphics[width=0.9\columnwidth]{usecase/UC24 - Inserimento del numero di raccomandazioni desiderate.png}
    \caption{UC24 - Inserimento del numero di raccomandazioni desiderate}
\end{figure}

\usecaseactors{Utente}
\usecasepre{L'utente ha eseguito con successo la task di analisi delle vendite e poi ha avviato la task di raccomandazione di prodotti e clienti}
\usecasedesc{L'utente inserisce il numero di raccomandazioni che desidera che vengano generate}
\usecasepost{Il sistema ha memorizzato il numero di raccomandazioni desiderate}
\label{uc:inserimento-numero-raccomandazioni}
\end{usecase}


\hypertarget{UC25}{}
\begin{usecase}{25}{Selezione lingua per la raccomandazione}

\begin{figure}[!h] 
    \centering 
    \includegraphics[width=0.9\columnwidth]{usecase/UC25 - Selezione lingua per la raccomandazione.png}
    \caption{UC25 - Selezione lingua per la raccomandazione}
\end{figure}

\usecaseactors{Utente}
\usecasepre{L'utente ha eseguito con successo la task di analisi delle vendite e poi ha avviato la task di raccomandazione di prodotti e clienti}
\usecasedesc{L'utente seleziona la lingua in cui desidera generare la raccomandazione}
\usecasepost{Il sistema ha memorizzato la lingua selezionata}
\usespecial{UC26, UC27, UC28, UC29, UC30}
\label{uc:selezione-lingua-raccomandazione}
\end{usecase}


\hypertarget{UC26}{}
\begin{usecase}{26}{Selezione lingua italiano per la raccomandazione}

\usecaseactors{Utente}
\usecasepre{L'utente ha eseguito con successo la task di analisi delle vendite e poi ha avviato la task di raccomandazione di prodotti e clienti}
\usecasedesc{L'utente seleziona la lingua italiano perchè desidera generare la raccomandazione in italiano}
\usecasepost{Il sistema ha memorizzato la lingua italiano selezionata}
\label{uc:selezione-lingua-italiano-raccomandazione}
\end{usecase}


\hypertarget{UC27}{}
\begin{usecase}{27}{Selezione lingua inglese per la raccomandazione}

\usecaseactors{Utente}
\usecasepre{L'utente ha eseguito con successo la task di analisi delle vendite e poi ha avviato la task di raccomandazione di prodotti e clienti}
\usecasedesc{L'utente seleziona la lingua inglese perchè desidera generare la raccomandazione in inglese}
\usecasepost{Il sistema ha memorizzato la lingua inglese selezionata}
\label{uc:selezione-lingua-inglese-raccomandazione}
\end{usecase}


\hypertarget{UC28}{}
\begin{usecase}{28}{Selezione lingua francese per la raccomandazione}

\usecaseactors{Utente}
\usecasepre{L'utente ha eseguito con successo la task di analisi delle vendite e poi ha avviato la task di raccomandazione di prodotti e clienti}
\usecasedesc{L'utente seleziona la lingua francese perchè desidera generare la raccomandazione in francese}
\usecasepost{Il sistema ha memorizzato la lingua francese selezionata}
\label{uc:selezione-lingua-francese-raccomandazione}
\end{usecase}


\hypertarget{UC29}{}
\begin{usecase}{29}{Selezione lingua spagnolo per la raccomandazione}

\usecaseactors{Utente}
\usecasepre{L'utente ha eseguito con successo la task di analisi delle vendite e poi ha avviato la task di raccomandazione di prodotti e clienti}
\usecasedesc{L'utente seleziona la lingua spagnolo perchè desidera generare la raccomandazione in spagnolo}
\usecasepost{Il sistema ha memorizzato la lingua spagnolo selezionata}
\label{uc:selezione-lingua-spagnolo-raccomandazione}
\end{usecase}


\hypertarget{UC30}{}
\begin{usecase}{30}{Selezione lingua tedesco per la raccomandazione}

\usecaseactors{Utente}
\usecasepre{L'utente ha eseguito con successo la task di analisi delle vendite e poi ha avviato la task di raccomandazione di prodotti e clienti}
\usecasedesc{L'utente seleziona la lingua tedesco perchè desidera generare la raccomandazione in tedesco}
\usecasepost{Il sistema ha memorizzato la lingua tedesco selezionata}
\label{uc:selezione-lingua-tedesco-raccomandazione}
\end{usecase}


\hypertarget{UC31}{}
\begin{usecase}{31}{Visualizzazione classifica elementi raccomandati}

\begin{figure}[!h]
    \centering 
    \includegraphics[width=0.9\columnwidth]{usecase/UC31 - Visualizzazione classifica elementi raccomandati.png}
    \caption{UC31 - Visualizzazione classifica elementi raccomandati}
\end{figure}

\usecaseactors{Utente}
\usecasepre{L'utente ha cliccato sul pulsante di esecuzione della task di raccomandazione di prodotti e clienti}
\usecasedesc{L'utente visualizza la classifica degli elementi raccomandati}
\usecasepost{L'utente ha visualizzato la classifica degli elementi raccomandati}
\usecasealt{UC32}
\usespecial{UC33, UC34}
\label{uc:visualizzazione-classifica-elementi-raccomandati}
\end{usecase}


\hypertarget{UC32}{}
\begin{usecase}{32}{Visualizzazione errore di elaborazione della task}

\usecaseactors{Utente}
\usecasepre{L'utente ha cliccato sul pulsante di esecuzione della task di raccomandazione di prodotti e clienti}
\usecasedesc{L'utente visualizza un errore nell'elaborazione della task di raccomandazione di prodotti e clienti}
\usecasepost{L'utente ha visualizzato un errore nell'elaborazione della task di raccomandazione di prodotti e clienti}
\label{uc:visualizzazione-errore-elaborazione-task}
\end{usecase}


\hypertarget{UC33}{}
\begin{usecase}{33}{Visualizzazione classifica prodotti raccomandati}

\usecaseactors{Utente}
\usecasepre{L'utente ha cliccato sul pulsante di esecuzione della task di raccomandazione di prodotti e clienti}
\usecasedesc{L'utente visualizza la classifica dei prodotti raccomandati}
\usecasepost{L'utente ha visualizzato la classifica dei prodotti raccomandati}
\label{uc:visualizzazione-classifica-prodotti-raccomandati}
\end{usecase}


\hypertarget{UC34}{}
\begin{usecase}{34}{Visualizzazione classifica clienti raccomandati}

\usecaseactors{Utente}
\usecasepre{L'utente ha cliccato sul pulsante di esecuzione della task di raccomandazione di prodotti e clienti}
\usecasedesc{L'utente visualizza la classifica dei clienti raccomandati}
\usecasepost{L'utente ha visualizzato la classifica dei clienti raccomandati}
\label{uc:visualizzazione-classifica-clienti-raccomandati}
\end{usecase}





\vspace{2cm}


Sono stati dunque elencati i casi d'uso del sistema individuati in fase di analisi iniziale. Tuttavia, alcuni di essi non rappresentano esattamente per filo e per segno i casi d'uso finali, poichè, nel corso dello stage, sono stati individuati alcuni limiti della piattaforma Oribea, che hanno portato a comportarsi diversamente rispetto a quanto inizialmente previsto per quanto riguarda l'output.

È stata allora presa di comune accordo con l'azienda la decisione di non modificare i diagrammi dei casi d'uso già modellati, per non perdere il lavoro già svolto e per non uscire dal tempo prestabilito per lo stage. I casi d'uso stabiliti in fase d'analisi sono stati comunque molto utili per la comprensione del sistema e per la fase iniziale di sviluppo, dunque nonostante i cambiamenti hanno comunque svolto un ruolo fondamentale nel progetto.


\newpage









\section{Tracciamento dei requisiti}

Da un'attenta analisi dei casi d'uso del progetto è stata stilata la tabella che traccia i requisiti in rapporto ai casi d'uso.

Sono stati individuati diversi tipi di requisiti e si è quindi fatto utilizzo di un codice identificativo per distinguerli.

Il codice dei requisiti è così strutturato:
\begin{center}
    \textbf{R[importanza][tipo]-[numero]}
\end{center}
dove:
\begin{itemize}
	\item L'importanza può essere O per i requisiti obbligatori, D per quelli desiderabili oppure Z per quelli opzionali;
	\item Il tipo può essere F per i requisiti funzionali, Q per quelli qualitativi oppure V per quelli di vincolo;
	\item Il numero è un valore numerico progressivo che identifica univocamente un requisito.
\end{itemize}
Nelle tabelle \ref{tab:requisiti-funzionali}, \ref{tab:requisiti-qualitativi} e \ref{tab:requisiti-vincolo} sono riassunti, rispettivamente, i requisiti funzionali, qualitativi e di vincolo, e l’eventuale loro tracciamento con i casi d'uso delineati in fase di analisi.


\subsection{Requisiti funzionali}

\RequisitiTable
  {Tabella di tracciamento dei requisiti funzionali}
  {tab:requisiti-funzionali}
  {\textbf{Requisito} & \textbf{Descrizione} & \textbf{Fonti}}

ROF-1 & Il sistema deve permettere all'utente di caricare un file CSV contenente gli ordini. & \hyperlink{UC1}{UC1} \\ \hline
ROF-2 & Il sistema deve permettere all'utente di selezionare la lingua da usare per generare il report, tra italiano, inglese, francese, spagnolo e tedesco. & \hyperlink{UC2}{UC2}, \hyperlink{UC3}{UC3}, \hyperlink{UC4}{UC4}, \hyperlink{UC5}{UC5}, \hyperlink{UC6}{UC6} e \hyperlink{UC7}{UC7} \\ \hline
ROF-3 & Il sistema deve permettere all'utente di selezionare la valuta da usare per generare il report, tra euro, dollaro e sterlina. & \hyperlink{UC8}{UC8}, \hyperlink{UC9}{UC9}, \hyperlink{UC10}{UC10} e \hyperlink{UC11}{UC11} \\ \hline
ROF-4 & Il sistema deve permettere all'utente di inserire l'indirizzo email a cui inviare il report generato dalla task di analisi delle vendite. & \hyperlink{UC12}{UC12} \\ \hline
ROF-5 & Nel caso la task di analisi delle vendite sia stata eseguita correttamente, il sistema deve permettere all'utente di visualizzare l'esito positivo della task. & \hyperlink{UC13}{UC13} \\ \hline
ROF-6 & Nel caso la task di analisi delle vendite sia stata eseguita correttamente, il sistema deve permettere all'utente di visualizzare il token identificativo del bucket di Google Cloud Storage in cui è stato salvato il file degli ordini, e di copiarlo negli appunti. & Azienda, \hyperlink{UC13.1}{UC13.1} \\ \hline
ROF-7 & Nel caso avvenga un imprevisto, il sistema deve permettere all'utente di visualizzare un messaggio che segnala un errore di elaborazione della task. & \hyperlink{UC14}{UC14} \\ \hline
ROF-8 & Il sistema deve permettere all'utente di scaricare il report generato dalla task di analisi delle vendite, in formato PDF. & \hyperlink{UC15}{UC15}, \hyperlink{UC16}{UC16}\\ \hline
RDF-9 & Il sistema deve permettere all'utente di scaricare il report generato dalla task di analisi delle vendite, in formato HTML. & \hyperlink{UC15}{UC15}, \hyperlink{UC17}{UC17} \\ \hline
RZF-10 & Il sistema deve inviare il report generato dalla task di analisi delle vendite all'indirizzo email inserito dall'utente. Il testo della mail deve contenere l'HTML del report, e i file PDF e HTML devono essere inviati in allegato. & \hyperlink{UC18}{UC18} \\ \hline
ROF-11 & Nel report delle vendite generato, l'utente deve visualizzare alcuni dati numerici legati alle vendite. & Azienda, \hyperlink{18.1}, \hyperlink{UC18.1.1}{UC18.1.1}, \hyperlink{UC18.1.2}{UC18.1.2}, \hyperlink{UC18.1.3}{UC18.1.3}, \hyperlink{UC18.1.4}{UC18.1.4} e \hyperlink{UC18.1.5}{UC18.1.5} \\ \hline
ROF-12 & Nel report delle vendite generato, l'utente deve visualizzare alcuni grafici generati a partire dai dati delle vendite. & Azienda, \hyperlink{UC18.2}{UC18.2}, \hyperlink{UC18.2.1}{UC18.2.1}, \hyperlink{UC18.2.2}{UC18.2.2}, \hyperlink{UC18.2.3}{UC18.2.3}, \hyperlink{UC18.2.4}{UC18.2.4}, \hyperlink{UC18.2.5}{UC18.2.5}, \hyperlink{UC18.2.6}{UC18.2.6} e \hyperlink{UC18.2.7}{UC18.2.7} \\ \hline
ROF-13 & Nel report delle vendite generato, l'utente deve visualizzare un resoconto dell'analisi delle vendite, generato dall'\gls{llm}. & Azienda, \hyperlink{UC18.3}{UC18.3} \\ \hline
RZF-14 & Nella mail inviata all'utente, l'utente deve visualizzare il token identificativo del bucket di Google Cloud Storage in cui è stato salvato il file degli ordini, e deve poterlo copiare negli appunti. & Azienda, \hyperlink{UC18.4}{UC18.4} \\ \hline
ROF-15 & Il sistema deve permettere all'utente di inserire il token identificativo del bucket di Google Cloud Storage in cui è stato salvato il file degli ordini, il quale era stato generato e mostrato nella task di analisi delle vendite. & \hyperlink{UC19}{UC19} \\ \hline
ROF-16 & Il sistema deve permettere all'utente di selezionare il tipo di raccomandazione da svolgere, tra "Raccomandare prodotti per un cliente" e "Raccomandare clienti per un prodotto". & \hyperlink{UC20}{UC20}, \hyperlink{UC21}{UC21} e \hyperlink{UC22}{UC22} \\ \hline
ROF-17 & Il sistema deve permettere all'utente di inserire il nome dell'elemento a cui desidera che vengano raccomandati prodotti o clienti, a seconda del tipo di raccomandazione selezionato. & \hyperlink{UC23}{UC23} \\ \hline
ROF-18 & Il sistema deve permettere all'utente di inserire il numero di raccomandazioni che desidera che vengano generate. & \hyperlink{UC24}{UC24} \\ \hline
ROF-19 & Il sistema deve permettere all'utente di selezionare la lingua da usare per generare la raccomandazione, tra italiano, inglese, francese, spagnolo e tedesco. & \hyperlink{UC25}{UC25}, \hyperlink{UC26}{UC26}, \hyperlink{UC27}{UC27}, \hyperlink{UC28}{UC28}, \hyperlink{UC29}{UC29} e \hyperlink{UC30}{UC30} \\ \hline
ROF-20 & Nel caso la task di raccomandazione sia stata eseguita correttamente, il sistema deve permettere all'utente di visualizzare la classifica degli elementi raccomandati. & \hyperlink{UC31}{UC31}, \hyperlink{UC33}{UC33} e \hyperlink{UC34}{UC34} \\ \hline
ROF-21 & Nel caso avvenga un imprevisto, il sistema deve permettere all'utente di visualizzare un messaggio che segnala un errore di elaborazione della task. & \hyperlink{UC32}{UC32} \\ \hline

\end{longtable}


\subsection{Requisiti qualitativi}

\RequisitiTable
  {Tabella di tracciamento dei requisiti qualitativi}
  {tab:requisiti-qualitativi}
  {\textbf{Requisito} & \textbf{Descrizione} & \textbf{Fonti}}

ROQ-1 & Il lavoro svolto deve essere opportunamente documentato. & OO6 \\ \hline
ROQ-2 & Il codice prodotto deve essere completamente coperto da test di unità. & OO6 \\ \hline
ROQ-3 & Le risposte prodotte dall'LLM devono essere testate, dopo aver definito un opportuno metodo di test. & Azienda, OO6 \\ \hline
RDQ-4 & La task di raccomandazione di prodotti e clienti deve produrre un risultato in tempi ragionevoli. & Azienda, OD1 \\ \hline
RZQ-5 & Le raccomandazioni devono essere "Explainable", in modo che un tecnico aziendale possa comprendere il motivo per cui sono stati raccomandati determinati prodotti o clienti come output ad un utente. & OZ2 \\ \hline

\end{longtable}


\subsection{Requisiti di vincolo}

\RequisitiTable
  {Tabella di tracciamento dei requisiti di vincolo}
  {tab:requisiti-vincolo}
  {\textbf{Requisito} & \textbf{Descrizione} & \textbf{Fonti}}

ROV-1 & Le funzionalità devono essere espresse mediante due task create usando la piattaforma Oribea. & OO3, OO4 \\ \hline
ROV-2 & I due progetti software delle due tasks devono essere inseriti nel campo function della schermata di creazione task della piattaforma Oribea. & Azienda, OO1 \\ \hline
ROV-3 & I due progetti software delle due tasks devono essere caricati come due repository nel profilo GitHub dell'azienda Oribea. & Azienda \\ \hline
ROV-4 & Le due task devono utilizzare, come archiviazione, il profilo Google Cloud dell'azienda Oribea. & Azienda, OO2 \\ \hline
ROV-5 & La task di analisi delle vendite deve permettere il download del report in formato PDF. & Azienda, OZ3 \\ \hline
RDV-6 & La task di analisi delle vendite deve permettere il download del report in formato HTML. & OZ3 \\ \hline
RZV-7 & La task di analisi delle vendite deve permettere l'invio del report via e-mail. & OZ3 \\ \hline

\end{longtable}
