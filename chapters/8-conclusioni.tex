\chapter{Conclusioni}
\label{cap:conclusioni}

\intro{In questo capitolo, vengono presentate le conclusioni del progetto, con un riepilogo degli obiettivi raggiunti, delle conoscenze acquisite e dei possibili miglioramenti futuri. Si conclude con una valutazione personale dell'esperienza di sviluppo.}

\section{Considerazioni finali}

- Il periodo he avuto l'obiettivo di far questo
- L'ho fatto così
- La maggior parte del tempo l'ho dedicata a questo
- La task è venuta bene, si è rilevata utile (e perchè?) e interessante
- Ricordare i requisiti per il dataset, senza di quelli non si può fare nulla


\section{Raggiungimento degli obiettivi}

- Tutti requisiti sono stati raggiunti
- Non è stato raggiunto obiettivo OZ1 perchè troppo complesso


\section{Conoscenze acquisite}

Il tirocinio svolto ha soddisfatto appieno le aspettative
cioè:
- Approfondimento del linguaggio di programmazione Python;
- Competenze di preparazione e analisi di gradi moli di dati;
- competenze nella stesura di un'analisi delle vendite;
- competenze nello sviluppo di un sistema di raccomandazione;
- competenze nel collegamento a Google Cloud Platform;
- competenze nell'integrazione con software aziendali pre-esistenti;
- Competenze nello sviluppo frontend di form di interazione con l'utente;
- Competenze di ottimizzazione delle performance di un sistema di analisi dati;
- competenze nella stesura della documentazione di un progetto software.


\section{Possibili miglioramenti futuri}

no implementazione per questioni di tempo. I possibili miglioramenti futuri sono:
- chatbot
- sistema di logging


\section{Valutazione personale}

completamento di percorso universitario, in particolare affinamento di Ingegneria del SOftware.
Utile per crescita professionale a livello tecnico.
Purtroppo non è stata utile per crescita professionale a livello di team, in quanto il team era composto da una sola persona, e a livello di azienda, in quanto ho lavorato da remoto e non ho avuto modo di interagire con altri colleghi.
Tuttavia, sono mediamente soddisfatto del risultato finale, in quanto ho raggiunto gli obiettivi prefissati e ho acquisito competenze utili che
rappresentano il punto di partenza per il mio percorso di laurea magistrale.
