% Acronyms
\newacronym[description={\glslink{api}{Application Program Interface}}]
    {api_acr}{API}{Application Program Interface}

\newacronym[description={\glslink{uml}{Unified Modeling Language}}]
    {uml_acr}{UML}{Unified Modeling Language}

\newacronym[description={\glslink{llm}{Large Language Model}}]
    {llm_acr}{LLM}{Large Language Model}

\newacronym[description={\glslink{ml}{Machine Learning}}]
    {ml_acr}{ML}{Machine Learning}

\newacronym[description={\glslink{ner}{Named Entity Recognition}}]
    {ner_acr}{NER}{Named Entity Recognition}

\newacronym[description={\glslink{rrf}{Reciprocal Rank Fusion}}]
    {rrf_acr}{RRF}{Reciprocal Rank Fusion}

\newacronym[description={\glslink{mapatk}{Mean Average Precision@\emph{k}}}]
    {mapatk_acr}{MAP@\emph{k}}{Mean Average Precision@\emph{k}}

\newacronym[description={\glslink{mrratk}{Mean Reciprocal Rank@\emph{k}}}]
    {mrratk_acr}{MRR@\emph{k}}{Mean Reciprocal Rank@\emph{k}}

\newacronym[description={\glslink{csv}{Comma-Separated Values}}]
    {csv_acr}{CSV}{Comma-Separated Values}

\newacronym[description={\glslink{hdf5}{Hierarchical Data Format 5}}]
    {hdf5_acr}{HDF5}{Hierarchical Data Format 5}

\newacronym[description={\glslink{npy}{NumPy Array}}]
    {npy_acr}{NPY}{NumPy Array}

\newacronym[description={\glslink{smtp}{Simple Mail Transfer Protocol}}]
    {smtp_acr}{SMTP}{Simple Mail Transfer Protocol}

\newacronym[description={\glslink{json}{JavaScript Object Notation}}]
    {json_acr}{JSON}{JavaScript Object Notation}

\newacronym[description={\glslink{curl}{Client URL}}]
    {curl_acr}{cURL}{Client URL}

\newacronym[description={\glslink{ndcgatk}{Normalized Discounted Cumulative Gain@\emph{k}}}]
    {ndcgatk_acr}{nDCG@\emph{k}}{Normalized Discounted Cumulative Gain@\emph{k}}

\newacronym[description={\glslink{rmse}{Root Mean Square Error}}]
    {rmse_acr}{RMSE}{Root Mean Square Error}

\newacronym[description={\glslink{bpr}{Bayesian Personalized Ranking}}]
    {bpr_acr}{BPR}{Bayesian Personalized Ranking}

\newacronym[description={\glslink{knnbasic}{K-Nearest Neighbors Basic}}]
    {knnbasic_acr}{KNNBasic}{K-Nearest Neighbors Basic}

\newacronym[description={\glslink{nlp}{Natural Language Processing}}]
    {nlp_acr}{NLP}{Natural Language Processing}




% Glossary entries
\newglossaryentry{api} {
    name=\glslink{api_acr}{API},
    text=API,
    sort=api,
    description={in informatica con il termine \emph{Application Programming Interface (API)} (ing. interfaccia di programmazione di un'applicazione) si indica ogni insieme di procedure disponibili al programmatore, di solito raggruppate a formare un set di strumenti specifici per l'espletamento di un determinato compito all'interno di un certo programma. La finalità è ottenere un'astrazione, di solito tra l'hardware e il programmatore o tra software a basso e quello ad alto livello semplificando così il lavoro di programmazione}
}

\newglossaryentry{uml} {
    name=\glslink{uml_acr}{UML},
    text=UML,
    sort=uml,
    description={in ingegneria del software \emph{UML, Unified Modeling Language} (ing. linguaggio di modellazione unificato) è un linguaggio di modellazione e specifica basato sul paradigma object-oriented. L'\emph{UML} svolge un'importantissima funzione di ``lingua franca'' nella comunità della progettazione e programmazione a oggetti. Gran parte della letteratura di settore usa tale linguaggio per descrivere soluzioni analitiche e progettuali in modo sintetico e comprensibile a un vasto pubblico}
}

\newglossaryentry{llm} {
    name=\glslink{llm_acr}{LLM},
    text=LLM,
    sort=llm,
    description={\emph{Large Language Model}, modello di linguaggio di grandi dimensioni addestrato su grandi quantità di dati testuali per comprendere e generare testo in linguaggio naturale}
}

\newglossaryentry{agile} {
    name=Agile,
    text=Agile,
    sort=agile,
    description={insieme di metodologie di sviluppo software basate su iterazioni brevi, collaborazione continua e adattamento ai cambiamenti, secondo i principi del Manifesto Agile}
}

\newglossaryentry{scrum} {
    name=Scrum,
    text=Scrum,
    sort=scrum,
    description={framework Agile per la gestione e lo sviluppo di progetti complessi, che suddivide il lavoro in iterazioni chiamate sprint e promuove la collaborazione tra team cross-funzionali}
}

\newglossaryentry{cloudfunctions} {
    name=Cloud Functions,
    text=Cloud Functions,
    sort=cloudfunctions,
    description={servizio di cloud computing che consente di eseguire codice in risposta a eventi senza dover gestire server, tipico dell'architettura serverless}
}

\newglossaryentry{cloudstorage} {
    name=Cloud Storage,
    text=Cloud Storage,
    sort=cloudstorage,
    description={servizio che permette di archiviare e gestire dati su server remoti accessibili tramite Internet, fornendo scalabilità e affidabilità}
}

\newglossaryentry{cloudcomputing} {
    name=Cloud Computing,
    text=Cloud Computing,
    sort=cloudcomputing,
    description={modello di erogazione di risorse informatiche (come server, storage, database, networking, software) tramite Internet, su richiesta e con pagamento a consumo}
}

\newglossaryentry{googlecloudstorage} {
    name=Google Cloud Storage,
    text=Google Cloud Storage,
    sort=googlecloudstorage,
    description={servizio di archiviazione cloud di Google che permette di memorizzare e accedere a dati su scala globale con elevate prestazioni, sicurezza e affidabilità}
}

\newglossaryentry{googlecloudfunctions} {
    name=Google Cloud Functions,
    text=Google Cloud Functions,
    sort=googlecloudfunctions,
    description={servizio serverless di Google Cloud Platform che consente di eseguire codice in risposta a eventi cloud senza dover gestire l'infrastruttura sottostante}
}

\newglossaryentry{preprocessing} {
    name=Preprocessing,
    text=preprocessing,
    sort=preprocessing,
    description={fase preliminare di elaborazione dei dati che comprende operazioni come pulizia, normalizzazione, trasformazione o selezione delle informazioni rilevanti, al fine di preparare i dati per l'analisi o l'addestramento di modelli o per altri scopi}
}

\newglossaryentry{busintel} {
    name=Business Intelligence,
    text=Business Intelligence,
    sort=businessintelligence,
    description={insieme di processi, tecnologie e strumenti che consentono di raccogliere, analizzare e presentare dati aziendali per supportare le decisioni strategiche e operative}
}

\newglossaryentry{ml} {
    name=Machine Learning,
    text=Machine Learning,
    sort=machinelearning,
    description={\glslink{ml_acr}{ML}, ramo dell'intelligenza artificiale che si occupa dello sviluppo di algoritmi e modelli in grado di apprendere automaticamente dai dati e migliorare le proprie prestazioni nel tempo senza essere esplicitamente programmati}
}

\newglossaryentry{ner} {
    name=\glslink{ner_acr}{NER},
    text=NER,
    sort=namedentityrecognition,
    description={tecnica di elaborazione del linguaggio naturale che consiste nell'identificare e classificare automaticamente entità nominate (come persone, organizzazioni, luoghi, date) all'interno di un testo}
}

\newglossaryentry{kmeans} {
    name=K-means,
    text=K-means,
    sort=kmeans,
    description={algoritmo di clustering non supervisionato che suddivide un insieme di dati in \emph{k} gruppi (cluster) in base alla vicinanza tra i punti, minimizzando la distanza intra-cluster}
}

\newglossaryentry{collaborativefiltering} {
    name=Collaborative Filtering,
    text=Collaborative Filtering,
    sort=collaborativefiltering,
    description={tecnica di sistema di raccomandazione che produce suggerimenti per un utente basandosi sui comportamenti e preferenze di utenti simili}
}

\newglossaryentry{similarita} {
    name=Similarità,
    text=similarità,
    sort=similarita,
    description={misura della somiglianza tra due o più oggetti, spesso utilizzata in informatica per confrontare testi, immagini o altri dati}
}

\newglossaryentry{rankfusion} {
    name=Rank Fusion,
    text=Rank Fusion,
    sort=rankfusion,
    description={tecnica che combina multiple classifiche di risultati provenienti da diversi algoritmi o sistemi per produrre una classifica finale più accurata e robusta}
}

\newglossaryentry{rrf} {
    name=\glslink{rrf_acr}{RRF},
    text=Reciprocal Rank Fusion (RRF),
    sort=rrf,
    description={metodo di combinazione dei risultati di più sistemi di recupero dell'informazione, che assegna un punteggio a ciascun elemento in base alla posizione nelle classifiche parziali, favorendo i risultati che appaiono in più classifiche}
}

\newglossaryentry{recallatk} {
    name=Recall@\emph{k},
    text=Recall@\emph{k},
    sort=recallatk,
    description={metrica che misura la frazione di elementi rilevanti recuperati tra i primi \emph{k} risultati restituiti da un sistema di ranking, rispetto al totale degli elementi rilevanti disponibili}
}

\newglossaryentry{precisionatk} {
    name=Precision@\emph{k},
    text=Precision@\emph{k},
    sort=precisionatk,
    description={metrica che misura la frazione di elementi rilevanti tra i primi \emph{k} risultati restituiti da un sistema di ranking}
}

\newglossaryentry{mapatk} {
    name=\glslink{mapatk_acr}{MAP@\emph{k}},
    text=Mean Average Precision@\emph{k} (MAP@\emph{k}),
    sort=mapatk,
    description={Mean Average Precision@\emph{k}, media delle precisioni calcolate a ogni posizione in cui un elemento rilevante appare tra i primi \emph{k} risultati}
}

\newglossaryentry{mrratk} {
    name=\glslink{mrratk_acr}{MRR@\emph{k}},
    text=Mean Reciprocal Rank@\emph{k} (MRR@\emph{k}),
    sort=mrratk,
    description={Mean Reciprocal Rank@\emph{k}, media dei reciproci della posizione degli elementi rilevanti tra i primi \emph{k} risultati}
}

\newglossaryentry{unserendipityatk} {
    name=Unserendipity@\emph{k},
    text=Unserendipity@\emph{k},
    sort=unserendipityatk,
    description={metrica che misura la mancanza di serendipità tra i primi \emph{k} risultati, ovvero la tendenza a raccomandare elementi già noti o attesi dall'utente}
}

\newglossaryentry{averageitemsimilarity} {
    name=Average Item Similarity,
    text=Average Item Similarity,
    sort=averageitemsimilarity,
    description={media delle similarità tra gli elementi raccomandati e gli elementi rilevanti}
}

\newglossaryentry{meanpopularityatk} {
    name=Mean Popularity@\emph{k},
    text=Mean Popularity@\emph{k},
    sort=meanpopularityatk,
    description={media della popolarità (ad esempio, numero di interazioni o visualizzazioni) degli elementi raccomandati tra i primi \emph{k} risultati}
}

\newglossaryentry{csv} {
    name=\glslink{csv_acr}{CSV},
    text=CSV,
    sort=csv,
    description={Comma-Separated Values, formato di file per la memorizzazione di dati tabulari in cui i valori sono separati da virgole o altri delimitatori, utilizzato per l'importazione ed esportazione di dati tra applicazioni}
}

\newglossaryentry{hdf5} {
    name=\glslink{hdf5_acr}{HDF5},
    text=HDF5,
    sort=hdf5,
    description={Hierarchical Data Format versione 5, formato di file progettato per memorizzare e organizzare grandi quantità di dati in strutture gerarchiche complesse, ampiamente utilizzato in ambito scientifico e di ricerca}
}

\newglossaryentry{npy} {
    name=\glslink{npy_acr}{NPY},
    text=NPY,
    sort=npy,
    description={formato di file binario di NumPy per la memorizzazione efficiente di array multidimensionali, che preserva tipo di dato, forma e ordine dei dati}
}

\newglossaryentry{parquet} {
    name=Parquet,
    text=Parquet,
    sort=parquet,
    description={formato di file colonnare open-source ottimizzato per l'archiviazione e l'elaborazione di grandi dataset, che offre compressione efficiente e prestazioni elevate nelle query analitiche}
}

\newglossaryentry{zarr} {
    name=Zarr,
    text=Zarr,
    sort=zarr,
    description={formato per l'archiviazione di array multidimensionali ottimizzato per il cloud computing, che supporta compressione, chunking e accesso parallelo ai dati}
}

\newglossaryentry{pandas} {
    name=Pandas,
    text=Pandas,
    sort=pandas,
    description={libreria Python per l'analisi e manipolazione dei dati che fornisce strutture dati ad alte prestazioni e strumenti di analisi, particolarmente utile per lavorare con dati strutturati e serie temporali}
}

\newglossaryentry{babel} {
    name=Babel,
        text=Babel,
        sort=babel,
        description={libreria Python per l'internazionalizzazione e localizzazione che fornisce strumenti per gestire traduzioni, formattazione di date, numeri e valute in diverse lingue e regioni}
}

\newglossaryentry{anthropic} {
    name=Anthropic,
    text=Anthropic,
    sort=anthropic,
    description={azienda di ricerca in intelligenza artificiale che sviluppa sistemi di IA sicuri e interpretabili, nota per aver creato l'assistente Claude}
}

\newglossaryentry{matplotlib} {
    name=Matplotlib,
    text=Matplotlib,
    sort=matplotlib,
    description={libreria Python per la creazione di grafici e visualizzazioni statiche, animate e interattive, ampiamente utilizzata nella comunità scientifica e di data science}
}

\newglossaryentry{pillow} {
    name=Pillow,
    text=Pillow,
    sort=pillow,
    description={libreria Python per l'elaborazione di immagini, fork del Python Imaging Library (PIL), che fornisce supporto per apertura, manipolazione e salvataggio di diversi formati di immagine}
}

\newglossaryentry{reportlab} {
    name=ReportLab,
    text=ReportLab,
    sort=reportlab,
    description={libreria Python per la generazione programmatica di documenti PDF, utilizzata per creare report, fatture e altri documenti formattati}
}

\newglossaryentry{jinja2} {
    name=Jinja2,
    text=Jinja2,
    sort=jinja2,
    description={motore di template moderno e veloce per Python, che permette di separare la logica di presentazione dal codice applicativo utilizzando una sintassi espressiva. Ad esempio, può essere utilizzato per generare HTML dinamico in applicazioni web}
}

\newglossaryentry{smtp} {
    name=\glslink{smtp_acr}{SMTP},
    text=SMTP,
    sort=smtp,
    description={Simple Mail Transfer Protocol, protocollo di comunicazione utilizzato per l'invio di messaggi di posta elettronica attraverso reti IP}
}

\newglossaryentry{sentencetransformers} {
    name=SentenceTransformers,
    text=SentenceTransformers,
    sort=sentencetransformers,
    description={libreria Python che fornisce un modo semplice per utilizzare modelli transformer pre-addestrati per generare embedding semantici di frasi e paragrafi}
}

\newglossaryentry{scikitlearn} {
    name=Scikit-learn,
    text=Scikit-learn,
    sort=scikitlearn,
    description={libreria Python per machine learning che fornisce strumenti semplici ed efficienti per data mining e analisi dei dati, costruita su NumPy, SciPy e Matplotlib}
}

\newglossaryentry{numpy} {
    name=NumPy,
    text=NumPy,
    sort=numpy,
    description={libreria Python fondamentale per il calcolo scientifico che fornisce supporto per array multidimensionali e un'ampia collezione di funzioni matematiche per operare su questi array}
}

\newglossaryentry{black} {
    name=Black,
    text=Black,
    sort=black,
    description={formattatore di codice Python automatico e deterministico che applica uno stile di codice coerente, eliminando discussioni sulla formattazione del codice}
}

\newglossaryentry{ruff} {
    name=Ruff,
    text=Ruff,
    sort=ruff,
    description={linter Python estremamente veloce scritto in Rust, che combina le funzionalità di diversi strumenti come Flake8, isort e pyupgrade}
}

\newglossaryentry{mypy} {
    name=MyPy,
    text=MyPy,
    sort=mypy,
    description={type checker statico per Python che verifica la correttezza dei tipi di dati utilizzando le type annotations, aiutando a individuare errori prima dell'esecuzione}
}

\newglossaryentry{pytest} {
    name=Pytest,
    text=Pytest,
    sort=pytest,
    description={framework per testing in Python che permette di scrivere test semplici e scalabili, con sintassi pulita e funzionalità avanzate come fixture e parametrizzazione}
}

\newglossaryentry{unittest} {
    name=Unittest,
    text=Unittest,
    sort=unittest,
    description={modulo integrato in Python per il testing del codice che fornisce un framework orientato agli oggetti per creare e eseguire test unitari, supportando test fixtures, test suites e test runner}
}

\newglossaryentry{dataframe} {
    name=DataFrame,
    text=DataFrame,
    sort=dataframe,
    description={struttura dati bidimensionale di Pandas con righe e colonne etichettate, che permette di memorizzare e manipolare dati di tipi diversi in formato tabellare, simile a una tabella di database o un foglio di calcolo Excel, con potenti funzionalità per l'analisi e trasformazione dei dati}
}

\newglossaryentry{hardcoded} {
    name=Hardcoded,
    text=hardcoded,
    sort=hardcoded,
    description={termine che indica valori o configurazioni inseriti direttamente nel codice sorgente invece di essere parametrizzati o configurabili esternamente}
}

\newglossaryentry{modelloembedding} {
    name=Modello di embedding,
    text=modello di embedding,
    sort=modelloembedding,
    description={modello di machine learning che trasforma dati (come testo, immagini o altri oggetti) in rappresentazioni vettoriali dense a dimensionalità fissa, catturando relazioni semantiche}
}

\newglossaryentry{json} {
    name=\glslink{json_acr}{JSON},
    text=JSON,
    sort=json,
    description={JavaScript Object Notation, formato di scambio dati leggero e human-readable basato su testo, ampiamente utilizzato per API web e configurazioni}
}

\newglossaryentry{base64} {
    name=Base64,
    text=Base64,
    sort=base64,
    description={schema di codifica che converte dati binari in una stringa di caratteri ASCII utilizzando un alfabeto di 64 caratteri, comunemente usato per trasmettere dati binari su protocolli testuali}
}

\newglossaryentry{curl} {
    name=\glslink{curl_acr}{cURL},
    text=cURL,
    sort=curl,
    description={strumento da riga di comando per trasferire dati da e verso server utilizzando vari protocolli di rete come HTTP, HTTPS, FTP, ampiamente utilizzato per testare API}
}

\newglossaryentry{macchinadev} {
    name=Macchina dev,
    text=macchina dev,
    sort=macchinadev,
    description={ambiente di sviluppo locale o remoto utilizzato dai programmatori per scrivere, testare e debuggare codice prima del deployment in produzione}
}

\newglossaryentry{requests} {
    name=Requests,
    text=Requests,
    sort=requests,
    description={libreria Python per effettuare richieste HTTP in modo semplice e intuitivo, che semplifica l'interazione con servizi web e API REST}
}

\newglossaryentry{angular} {
    name=Angular,
    text=Angular,
    sort=angular,
    description={framework TypeScript open-source sviluppato da Google per la creazione di applicazioni web single-page, che fornisce una struttura completa per lo sviluppo frontend}
}

\newglossaryentry{react} {
    name=React,
    text=React,
    sort=react,
    description={libreria JavaScript sviluppata da Facebook per la costruzione di interfacce utente, basata su componenti riutilizzabili e un approccio dichiarativo}
}

\newglossaryentry{vue} {
    name=Vue,
    text=Vue,
    sort=vue,
    description={framework JavaScript progressivo per la costruzione di interfacce utente, caratterizzato da una curva di apprendimento graduale e grande flessibilità}
}

\newglossaryentry{reacthookform} {
    name=React Hook Form,
    text=React Hook Form,
    sort=reacthookform,
    description={libreria per la gestione di form in React che minimizza i re-render e fornisce validazione performante con un'API semplice basata su hook}
}

\newglossaryentry{zod} {
    name=Zod,
    text=Zod,
    sort=zod,
    description={libreria TypeScript per la validazione e parsing di schemi dati con inferenza automatica dei tipi, che garantisce type safety a runtime. Ideale per la validazione di dati immessi in form o ricevuti da API}
}

\newglossaryentry{shadcnui} {
    name=Shadcn/ui,
    text=Shadcn/ui,
    sort=shadcnui,
    description={collezione di componenti UI riutilizzabili per React, costruiti con Radix UI e Tailwind CSS, che possono essere copiati e personalizzati in un progetto React per creare interfacce utente moderne e accessibili}
}

\newglossaryentry{pandasvectorizedops} {
    name=Pandas Vectorized Operations,
    text=Pandas Vectorized Ops,
    sort=pandasvectorizedops,
    description={operazioni ottimizzate in Pandas che vengono applicate simultaneamente a interi array di dati invece che elemento per elemento, migliorando significativamente le prestazioni}
}

\newglossaryentry{pandasdataframeapply} {
    name=pandas.DataFrame.apply,
    text=pandas.DataFrame.apply,
    sort=pandasdataframeapply,
    description={metodo di Pandas che applica una funzione lungo un asse di un DataFrame, permettendo di trasformare i dati riga per riga o colonna per colonna. È utile per operazioni complesse che non possono essere facilmente eseguite con operazioni vettorializzate, ma, nel caso sia possibile, è consigliabile utilizzare operazioni vettorializzate per migliorare le prestazioni}
}

\newglossaryentry{hook} {
    name=Hook,
    text=hook,
    sort=hook,
    description={funzioni speciali in React che permettono di utilizzare stato e altre funzionalità React, come useState e useEffect, all'interno di componenti funzionali}
}

\newglossaryentry{vitest} {
    name=Vitest,
    text=Vitest,
    sort=vitest,
    description={framework di testing veloce per progetti Vite che fornisce un'esperienza di test moderna con hot module replacement, supporto nativo per TypeScript e compatibilità con l'API di Jest}
}

\newglossaryentry{jest} {
    name=Jest,
    text=Jest,
    sort=jest,
    description={framework di testing JavaScript sviluppato da Facebook, caratterizzato da configurazione zero, snapshot testing integrato, mocking potente e coverage reports, ampiamente utilizzato per testare applicazioni React e Node.js}
}

\newglossaryentry{flowable} {
    name=Flowable,
    text=Flowable,
    sort=flowable,
    description={in ReportLab, oggetto che può essere posizionato e renderizzato su una pagina PDF, come paragrafi, tabelle, immagini o grafici. I Flowable si adattano automaticamente al layout della pagina e possono essere interrotti tra le pagine}
}

\newglossaryentry{canvas} {
    name=Canvas,
    text=Canvas,
    sort=canvas,
    description={in ReportLab, superficie di disegno che rappresenta una pagina PDF su cui è possibile disegnare testo, forme, immagini e altri elementi grafici con controllo preciso sulla posizione e formattazione}
}

\newglossaryentry{bordacount} {
    name=Borda Count (Borda Fusion),
    text=Borda Count (Borda Fusion),
    sort=bordacount,
    description={metodo di rank fusion che assegna punteggi basati sulla posizione in classifica, dove ogni elemento riceve punti inversamente proporzionali alla sua posizione, e i punteggi vengono sommati per produrre la classifica finale}
}

\newglossaryentry{scorebasednorm} {
    name=Score-based Normalization + Aggregation,
    text=Score-based Normalization + Aggregation,
    sort=scorebasednorm,
    description={approccio di rank fusion che normalizza i punteggi di diversi sistemi per renderli comparabili, quindi li aggrega utilizzando operazioni matematiche per produrre punteggi finali}
}

\newglossaryentry{score_combination} {
    name={CombSUM, CombMAX, CombMIN, CombANZ},
    text={CombSUM, CombMAX, CombMIN, CombANZ},
    sort=combsum,
    description={metodi di aggregazione dei punteggi per rank fusion: CombSUM somma i punteggi, CombMAX prende il massimo, CombMIN il minimo, CombANZ somma solo i punteggi non-zero normalizzandoli per il numero di sistemi che hanno assegnato un punteggio}
}

\newglossaryentry{weightedrank} {
    name=Weighted Rank Aggregation,
    text=Weighted Rank Aggregation,
    sort=weightedrank,
    description={tecnica di rank fusion che assegna pesi diversi ai vari sistemi di ranking basandosi sulla loro affidabilità o prestazioni, combinando le classifiche con importanza variabile}
}

\newglossaryentry{medianrank} {
    name=Median Rank,
    text=Median Rank,
    sort=medianrank,
    description={metodo di rank fusion che utilizza la mediana delle posizioni di un elemento nelle diverse classifiche per determinare la posizione finale, fornendo una misura robusta che non è influenzata da valori estremi}
}

\newglossaryentry{geometricmeanrank} {
    name=Geometric Mean Rank,
    text=Geometric Mean Rank,
    sort=geometricmeanrank,
    description={metodo di rank fusion che utilizza la media geometrica delle posizioni di un elemento nelle diverse classifiche per determinare la posizione finale, dando maggior peso ai ranking migliori rispetto alla media aritmetica}
}

\newglossaryentry{learningtorank} {
    name=Learning to Rank a Posteriori (Late Fusion with ML),
    text=Learning to Rank a Posteriori (Late Fusion with ML),
    sort=learningtorank,
    description={approccio di rank fusion che utilizza algoritmi di machine learning per apprendere come combinare ottimalmente i risultati di diversi sistemi di ranking, basandosi su dati di training con classifiche ideali}
}

\newglossaryentry{concordetfusion} {
    name=Condorcet Fusion,
    text=Condorcet Fusion,
    sort=concordetfusion,
    description={metodo di rank fusion basato sul criterio di Condorcet, che determina il ranking finale attraverso confronti a coppie tra elementi, dove un elemento è preferito se vince la maggioranza dei confronti diretti}
}

\newglossaryentry{rbpfusion} {
    name=Rank-biased Precision (RBP) Fusion,
    text=Rank-biased Precision (RBP) Fusion,
    sort=rbpfusion,
    description={tecnica di rank fusion che utilizza la Rank-biased Precision per pesare maggiormente le posizioni iniziali delle classifiche, dando più importanza ai risultati che appaiono in cima alle liste}
}

\newglossaryentry{fpdf} {
    name=FPDF,
    text=FPDF,
    sort=fpdf,
    description={libreria PHP per la generazione di documenti PDF che non richiede estensioni aggiuntive, permettendo di creare PDF con testo, immagini e grafici di base}
}

\newglossaryentry{pypdf2} {
    name=PyPDF2,
    text=PyPDF2,
    sort=pypdf2,
    description={libreria Python per la manipolazione di file PDF esistenti, che permette operazioni come merge, split, crop e rotazione di pagine, ma con funzionalità limitate per la creazione di nuovi contenuti}
}

\newglossaryentry{pdfrw} {
    name=PDFRW,
    text=PDFRW,
    sort=pdfrw,
    description={libreria Python per leggere e scrivere file PDF, che si concentra sulla manipolazione della struttura interna del PDF e supporta operazioni avanzate come la gestione di form e annotazioni}
}

\newglossaryentry{mako} {
    name=Mako,
    text=Mako,
    sort=mako,
    description={motore di template veloce e leggero per Python che supporta codice Python embedded, ereditarietà di template e caching automatico, utilizzato spesso con framework web come Pylons e Pyramid}
}

\newglossaryentry{dominate} {
    name=Dominate,
    text=Dominate,
    sort=dominate,
    description={libreria Python per generare documenti HTML utilizzando un approccio programmatico con sintassi pythonica, che permette di creare markup HTML attraverso oggetti e metodi Python}
}

\newglossaryentry{yattag} {
    name=Yattag,
    text=Yattag,
    sort=yattag,
    description={libreria Python per la generazione di documenti HTML e XML che fornisce un'API semplice e intuitiva per creare markup attraverso context manager e una sintassi dichiarativa}
}

\newglossaryentry{motoretemplate} {
    name=Motore di template,
    text=motore di template,
    sort=motoretemplate,
    description={sistema software che combina template contenenti markup statico con dati dinamici per produrre documenti finali, separando la logica di presentazione dalla logica applicativa}
}

\newglossaryentry{generazioneprogrammatica} {
    name=Generazione programmatica,
    text=generazione programmatica,
    sort=generazioneprogrammatica,
    description={approccio di creazione di documenti o contenuti attraverso codice software, dove la struttura e il contenuto vengono definiti algoritmicamente piuttosto che manualmente, permettendo automazione e personalizzazione dinamica}
}

\newglossaryentry{recbole} {
    name=RecBole,
    text=RecBole,
    sort=recbole,
    description={libreria Python unificata per la ricerca e lo sviluppo di sistemi di raccomandazione, che fornisce implementazioni di numerosi algoritmi di recommendation e strumenti per la valutazione e il confronto delle prestazioni}
}

\newglossaryentry{surprise_metric} {
    name=Surprise,
    text=Surprise,
    sort=surprise,
    description={libreria Python per la costruzione e l'analisi di sistemi di raccomandazione, che fornisce implementazioni di algoritmi di collaborative filtering e strumenti per la valutazione delle prestazioni}
}

\newglossaryentry{ndcgatk} {
    name=\glslink{ndcgatk_acr}{nDCG@\emph{k}},
    text=nDCG@\emph{k},
    sort=ndcgatk,
    description={Normalized Discounted Cumulative Gain@\emph{k}, metrica che calcola il guadagno cumulativo scontato normalizzato tenendo conto sia della rilevanza degli elementi sia della loro posizione nel ranking tra i primi \emph{k} risultati. La normalizzazione permette di confrontare prestazioni su query diverse, mentre lo sconto penalizza elementi rilevanti che appaiono in posizioni più basse}
}

\newglossaryentry{hitrateatk} {
    name=Hit Rate@\emph{k},
    text=Hit Rate@\emph{k},
    sort=hitrateatk,
    description={metrica binaria che vale 1 se almeno un elemento rilevante è presente tra i primi \emph{k} risultati raccomandati, 0 altrimenti. È particolarmente utile per valutare se il sistema riesce a catturare almeno una preferenza dell'utente nelle prime posizioni}
}

\newglossaryentry{coverageatk} {
    name=Coverage@\emph{k},
    text=Coverage@\emph{k},
    sort=coverageatk,
    description={metrica che misura la varietà del sistema calcolando quanti elementi diversi vengono raccomandati complessivamente agli utenti tra i primi \emph{k} risultati. Una coverage elevata indica che il sistema sfrutta un'ampia porzione del catalogo disponibile, evitando di concentrarsi sempre sugli stessi elementi popolari}
}

\newglossaryentry{intralistdiversity} {
    name=Intra-list Diversity,
    text=Intra-list Diversity,
    sort=intralistdiversity,
    description={metrica che misura quanto sono diversi tra loro gli elementi raccomandati all'interno della stessa lista di raccomandazioni per un singolo utente. Una diversità intra-lista elevata garantisce che l'utente riceva suggerimenti variegati piuttosto che elementi molto simili tra loro}
}

\newglossaryentry{novelty} {
    name=Novelty,
    text=Novelty,
    sort=novelty,
    description={metrica che quantifica la novità delle raccomandazioni favorendo elementi poco popolari, che sono potenzialmente più "nuovi" per l'utente. Si basa sull'assunto che elementi con poche interazioni storiche possano rappresentare scoperte interessanti}
}

\newglossaryentry{serendipity} {
    name=Serendipity,
    text=Serendipity,
    sort=serendipity,
    description={metrica che misura la capacità del sistema di raccomandare elementi inaspettati ma potenzialmente interessanti per l'utente. La serendipità rappresenta la qualità di fare scoperte piacevoli e inattese, bilanciando accuratezza e sorpresa nelle raccomandazioni}
}

\newglossaryentry{surprise} {
    name=Surprise,
    text=Surprise,
    sort=surprise,
    description={metrica che quantifica il grado di sorpresa nelle raccomandazioni calcolando la distanza tra le aspettative dell'utente e i risultati effettivi}
}

\newglossaryentry{confidence} {
    name=Confidence,
    text=Confidence,
    sort=confidence,
    description={metrica che valuta il livello di fiducia del sistema nelle proprie raccomandazioni, spesso correlato alla popolarità degli elementi}
}

\newglossaryentry{distance} {
    name=Distance,
    text=Distance,
    sort=distance,
    description={metrica che misura la distanza semantica o comportamentale tra gli elementi raccomandati e quelli già noti all'utente}
}

\newglossaryentry{expectedness} {
    name=Expectedness,
    text=Expectedness,
    sort=expectedness,
    description={metrica che quantifica quanto le raccomandazioni siano prevedibili o attese dall'utente}
}

\newglossaryentry{rmse} {
    name=\glslink{rmse_acr}{RMSE},
    text=Root mean square deviation (RMSE),
    sort=rmse,
    description={Root Mean Square Error, metrica che calcola l'errore quadratico medio per valutare l'accuratezza delle predizioni di rating}
}

\newglossaryentry{explainability} {
    name=Explainability,
    text=Explainability,
    sort=explainability,
    description={capacità di un sistema di fornire spiegazioni comprensibili delle proprie decisioni e raccomandazioni, permettendo agli utenti di comprendere il motivo per cui determinati elementi sono stati suggeriti}
}

\newglossaryentry{logging} {
    name=Logging,
    text=Logging,
    sort=logging,
    description={modulo integrato in Python per la registrazione di messaggi durante l'esecuzione del programma, che fornisce un sistema flessibile per tracciare eventi, errori e informazioni di debug con diversi livelli di severità}
}

\newglossaryentry{bpr} {
    name=\glslink{bpr_acr}{BPR},
    text=BPR,
    sort=bpr,
    description={Bayesian Personalized Ranking, algoritmo di raccomandazione implementato in RecBole che ottimizza il ranking personalizzato utilizzando un approccio bayesiano per apprendere le preferenze degli utenti dai feedback impliciti}
}

\newglossaryentry{knnbasic} {
    name=\glslink{knnbasic_acr}{KNNBasic},
    text=KNNBasic,
    sort=knnbasic,
    description={algoritmo di collaborative filtering basato sui k-nearest neighbors implementato nella libreria Surprise, che predice i rating utilizzando la similarità tra utenti o elementi per identificare i vicini più simili}
}

\newglossaryentry{distanzaeuclidea} {
    name=Distanza Euclidea,
    text=Distanza Euclidea,
    sort=distanzaeuclidea,
    description={misura la distanza geometrica diretta tra due punti nello spazio multidimensionale, calcolata come la radice quadrata della somma dei quadrati delle differenze tra le componenti corrispondenti. È sensibile alla magnitudine dei vettori e può essere influenzata dalla dimensionalità}
}

\newglossaryentry{correlazionepearson} {
    name=Correlazione di Pearson,
    text=Correlazione di Pearson,
    sort=correlazionepearson,
    description={coefficiente che calcola la correlazione lineare tra due variabili, variando da -1 a 1. Richiede che i dati seguano una distribuzione normale e può essere instabile con dati sparsi}
}

\newglossaryentry{distanzamanhattan} {
    name=Distanza di Manhattan,
    text=Distanza di Manhattan,
    sort=distanzamanhattan,
    description={metrica di distanza che calcola la somma delle differenze assolute tra le componenti dei vettori, anche nota come distanza taxi o distanza L1. È semplice da calcolare ma meno efficace nel catturare relazioni complesse}
}

\newglossaryentry{similaritajaccard} {
    name=Similarità di Jaccard,
    text=Similarità di Jaccard,
    sort=similaritajaccard,
    description={indice di similarità particolarmente adatto per dati binari, che misura la sovrapposizione tra insiemi calcolando il rapporto tra l'intersezione e l'unione. Non considera l'intensità delle interazioni ma solo la presenza o assenza}
}

\newglossaryentry{distanzahamming} {
    name=Distanza di Hamming,
    text=Distanza di Hamming,
    sort=distanzahamming,
    description={metrica utilizzata principalmente per confrontare stringhe o sequenze binarie di uguale lunghezza, calcolando il numero di posizioni in cui i simboli corrispondenti sono diversi. Non è adatta per dati numerici continui}
}

\newglossaryentry{divergenzakullbackleibler} {
    name=Divergenza di Kullback-Leibler,
    text=Divergenza di Kullback-Leibler,
    sort=divergenzakullbackleibler,
    description={misura non simmetrica della differenza tra due distribuzioni di probabilità, anche nota come divergenza KL. Richiede che i dati siano interpretabili come distribuzioni e può essere computazionalmente costosa}
}

\newglossaryentry{cosinesimilarity} {
    name=Cosine Similarity,
    text=Cosine Similarity,
    sort=cosinesimilarity,
    description={metrica di similarità che misura l'angolo tra due vettori in uno spazio multidimensionale, calcolando il coseno dell'angolo formato. Varia da -1 a 1, dove 1 indica vettori identici, 0 vettori ortogonali e -1 vettori opposti. È invariante alla magnitudine dei vettori e particolarmente efficace per confrontare documenti e profili utente}
}

\newglossaryentry{nlp} {
    name=\glslink{nlp_acr}{NLP},
    text=Natural Language Processing,
    sort=nlp,
    description={ramo dell'intelligenza artificiale che si occupa dell'interazione tra computer e linguaggio umano, includendo tecniche per l'analisi, comprensione e generazione di testo in linguaggio naturale}
}
