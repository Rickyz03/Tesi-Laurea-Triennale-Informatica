\chapter{Descrizione dello stage}
\label{cap:descrizione-stage}

\intro{In questa sezione viene presentata una descrizione dell’idea e delle tecnologie utilizzate durante il percorso di stage, e una panoramica del modo di lavorare dell'azienda. Viene inoltre presentata un'analisi dei rischi e delle problematiche riscontrate durante lo sviluppo del progetto.}\\

\section{Introduzione al progetto}

Nel contesto attuale, i report delle vendite e i sistemi di raccomandazione rivestono un ruolo fondamentale per le aziende che operano in mercati competitivi e digitalizzati. I report delle vendite consentono di monitorare l’andamento commerciale, identificare trend, valutare le performance dei prodotti e prendere decisioni strategiche basate su dati concreti. Questi strumenti permettono di individuare rapidamente eventuali criticità o opportunità di crescita, ottimizzando così le strategie di vendita e marketing.\\

Parallelamente, i sistemi di raccomandazione sono diventati essenziali per migliorare l’esperienza utente e incrementare le vendite, soprattutto nelle piattaforme di e-commerce e nei servizi digitali. Attraverso l’analisi dei dati di acquisto e delle preferenze degli utenti, questi sistemi suggeriscono prodotti o servizi personalizzati, aumentando la probabilità di acquisto e la fidelizzazione del cliente. L’integrazione di report delle vendite e sistemi di raccomandazione consente quindi alle aziende di offrire un servizio più mirato ed efficiente, rafforzando la propria posizione sul mercato.\\

In questo contesto, la startup Oribea, che propone soluzioni digitali intelligenti per le aziende, ha ricevuto una commissione da parte di un'azienda di e-commerce per la creazione di strumenti che automatizzino la generazione di report delle vendite e sistemi di raccomandazione. Oribea ha quindi deciso di integrare queste due funzionalità all’interno della propria piattaforma omonima, che consente di usufruire delle cosiddette Task: strumenti che permettono di automatizzare e semplificare specifiche attività aziendali quotidiane.\\
Il progetto di stage si inserisce in questo scenario, con l'obiettivo di sviluppare una Task che consenta di generare un report delle vendite e un sistema di raccomandazione a partire da un file degli ordini, e un’altra Task che permetta di sfruttare il sistema di raccomandazione per generare suggerimenti personalizzati per un determinato cliente o prodotto.\\

Le Task di Oribea sono implementate come funzioni serverless, ovvero funzioni che vengono eseguite in modo autonomo e scalabile, senza la necessità di gestire l'infrastruttura sottostante, e sono caricate ed eseguite su Google Cloud Functions. Lo stato delle funzioni viene salvato in Google Cloud Storage. Il compito di stage prevede quindi lo sviluppo di due Cloud Functions serverless che implementano le funzionalità richieste, la loro comunicazione tramite Cloud Storage e la loro integrazione con la piattaforma Oribea.\\


\section{Way of working e strumenti utilizzati}
\label{sec:way-of-working}

L’azienda Oribea adotta un modello di sviluppo Agile, con l’obiettivo di monitorare e controllare il progetto in modo flessibile e continuo, suddividendo le attività in piccoli incrementi e con una collaborazione asincrona e distribuita.\\
In particolare, il modello di sviluppo adottato è Scrum, che prevede la suddivisione del progetto in sprint, ovvero periodi di tempo di durata fissa, in cui vengono pianificate le attività da svolgere e i relativi obiettivi da raggiungere. All'inizio del progetto è stato scelto di comune accordo la lunghezza per gli sprint di una settimana. Al termine di ogni sprint, è stato effettuato un incontro approfondito con il tutor aziendale, per discutere lo stato di avanzamento del progetto e le attività da svolgere per il successivo sprint.\\
L’obiettivo del modello è dare maggiore importanza al ciclo di vita del software e dei processi correlati, piuttosto che al prodotto finale, con il fine ultimo di migliorare la qualità del prodotto stesso.\\

Gli strumenti principali utilizzati includono:
\begin{itemize}
    \item \textbf{Visual Studio Code}: per la scrittura e la modifica del codice sorgente;
    \item \textbf{Git}: per il versionamento del codice e la gestione delle modifiche;
    \item \textbf{GitHub}: per la gestione del codice sorgente e la collaborazione tra sviluppatori;
    \item \textbf{Google Cloud Functions}: per l'hosting delle functions serverless da collegare ai task di Oribea;
    \item \textbf{Google Cloud Storage}: per l'archiviazione dei file e delle matrice di raccomandazione;
    \item \textbf{Slack}: per la comunicazione interna e la gestione dei progetti;
    \item \textbf{Notion}: per la documentazione del progetto;
    \item \textbf{StarUML}: per la creazione dei diagrammi UML.
\end{itemize}

\section{Analisi preventiva dei rischi}

Durante la fase di analisi iniziale sono stati individuati alcuni possibili rischi a cui si potrà andare incontro.
Si è quindi proceduto a elaborare delle possibili soluzioni per far fronte a tali rischi.\\

\begin{risk}{Assenza di dataset di addestramento}
    \riskdescription{s}
    \risksolution{a}
    \label{risk:dataset-absence} 
\end{risk}

\begin{risk}{Risposta dell'LLM imprecisa}
    \riskdescription{s}
    \risksolution{a}
    \label{risk:bad-llm-response} 
\end{risk}
