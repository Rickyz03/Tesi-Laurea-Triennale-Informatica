% Acronyms
\newacronym[description={\glslink{api}{Application Program Interface}}]
    {api_acr}{API}{Application Program Interface}

\newacronym[description={\glslink{uml}{Unified Modeling Language}}]
    {uml_acr}{UML}{Unified Modeling Language}

\newacronym[description={\glslink{llm}{Large Language Model}}]
    {llm_acr}{LLM}{Large Language Model}

\newacronym[description={\glslink{rrf}{Reciprocal Rank Fusion}}]
    {rrf_acr}{RRF}{Reciprocal Rank Fusion}

\newacronym[description={\glslink{mapatk}{Mean Average Precision@\emph{k}}}]
    {mapatk_acr}{MAP@\emph{k}}{Mean Average Precision@\emph{k}}

\newacronym[description={\glslink{mrratk}{Mean Reciprocal Rank@\emph{k}}}]
    {mrratk_acr}{MRR@\emph{k}}{Mean Reciprocal Rank@\emph{k}}

% Glossary entries
\newglossaryentry{api} {
    name=\glslink{api_acr}{API},
    text=API,
    sort=api,
    description={in informatica con il termine \emph{Application Programming Interface (API)} (ing. interfaccia di programmazione di un'applicazione) si indica ogni insieme di procedure disponibili al programmatore, di solito raggruppate a formare un set di strumenti specifici per l'espletamento di un determinato compito all'interno di un certo programma. La finalità è ottenere un'astrazione, di solito tra l'hardware e il programmatore o tra software a basso e quello ad alto livello semplificando così il lavoro di programmazione}
}

\newglossaryentry{uml} {
    name=\glslink{uml_acr}{UML},
    text=UML,
    sort=uml,
    description={in ingegneria del software \emph{UML, Unified Modeling Language} (ing. linguaggio di modellazione unificato) è un linguaggio di modellazione e specifica basato sul paradigma object-oriented. L'\emph{UML} svolge un'importantissima funzione di ``lingua franca'' nella comunità della progettazione e programmazione a oggetti. Gran parte della letteratura di settore usa tale linguaggio per descrivere soluzioni analitiche e progettuali in modo sintetico e comprensibile a un vasto pubblico}
}

\newglossaryentry{llm} {
    name=\glslink{llm_acr}{LLM},
    text=LLM,
    sort=llm,
    description={\emph{Large Language Model}, modello di linguaggio di grandi dimensioni addestrato su grandi quantità di dati testuali per comprendere e generare testo in linguaggio naturale}
}

\newglossaryentry{agile} {
    name=Agile,
    text=Agile,
    sort=agile,
    description={insieme di metodologie di sviluppo software basate su iterazioni brevi, collaborazione continua e adattamento ai cambiamenti, secondo i principi del Manifesto Agile}
}

\newglossaryentry{scrum} {
    name=Scrum,
    text=Scrum,
    sort=scrum,
    description={framework Agile per la gestione e lo sviluppo di progetti complessi, che suddivide il lavoro in iterazioni chiamate sprint e promuove la collaborazione tra team cross-funzionali}
}

\newglossaryentry{cloudfunctions} {
    name=Cloud Functions,
    text=Cloud Functions,
    sort=cloudfunctions,
    description={servizio di cloud computing che consente di eseguire codice in risposta a eventi senza dover gestire server, tipico dell'architettura serverless}
}

\newglossaryentry{cloudstorage} {
    name=Cloud Storage,
    text=Cloud Storage,
    sort=cloudstorage,
    description={servizio che permette di archiviare e gestire dati su server remoti accessibili tramite Internet, fornendo scalabilità e affidabilità}
}

\newglossaryentry{cloudcomputing} {
    name=Cloud Computing,
    text=Cloud Computing,
    sort=cloudcomputing,
    description={modello di erogazione di risorse informatiche (come server, storage, database, networking, software) tramite Internet, su richiesta e con pagamento a consumo}
}

\newglossaryentry{similarita} {
    name=Similarità,
    text=similarità,
    sort=similarita,
    description={misura della somiglianza tra due o più oggetti, spesso utilizzata in informatica per confrontare testi, immagini o altri dati}
}

\newglossaryentry{rrf} {
    name=\glslink{rrf_acr}{RRF},
    text=RRF,
    sort=rrf,
    description={metodo di combinazione dei risultati di più sistemi di recupero dell'informazione, che assegna un punteggio a ciascun elemento in base alla posizione nelle classifiche parziali, favorendo i risultati che appaiono in più classifiche}
}

\newglossaryentry{recallatk} {
    name=Recall@\emph{k},
    text=Recall@\emph{k},
    sort=recallatk,
    description={metrica che misura la frazione di elementi rilevanti recuperati tra i primi \emph{k} risultati restituiti da un sistema di ranking, rispetto al totale degli elementi rilevanti disponibili}
}

\newglossaryentry{precisionatk} {
    name=Precision@\emph{k},
    text=Precision@\emph{k},
    sort=precisionatk,
    description={metrica che misura la frazione di elementi rilevanti tra i primi \emph{k} risultati restituiti da un sistema di ranking}
}

\newglossaryentry{mapatk} {
    name=\glslink{map_acr}{MAP@\emph{k}},
    text=MAP@\emph{k},
    sort=mapatk,
    description={Mean Average Precision@\emph{k}, media delle precisioni calcolate a ogni posizione in cui un elemento rilevante appare tra i primi \emph{k} risultati}
}

\newglossaryentry{mrratk} {
    name=\glslink{mrratk_acr}{MRR@\emph{k}},
    text=MRR@\emph{k},
    sort=mrratk,
    description={Mean Reciprocal Rank@\emph{k}, media dei reciproci della posizione degli elementi rilevanti tra i primi \emph{k} risultati}
}

\newglossaryentry{unserendipityatk} {
    name=Unserendipity@\emph{k},
    text=Unserendipity@\emph{k},
    sort=unserendipityatk,
    description={metrica che misura la mancanza di serendipità tra i primi \emph{k} risultati, ovvero la tendenza a raccomandare elementi già noti o attesi dall'utente}
}

\newglossaryentry{averageitemsimilarity} {
    name=Average Item Similarity,
    text=Average Item Similarity,
    sort=averageitemsimilarity,
    description={media delle similarità tra gli elementi raccomandati e gli elementi rilevanti}
}

\newglossaryentry{meanpopularityatk} {
    name=Mean Popularity@\emph{k},
    text=Mean Popularity@\emph{k},
    sort=meanpopularityatk,
    description={media della popolarità (ad esempio, numero di interazioni o visualizzazioni) degli elementi raccomandati tra i primi \emph{k} risultati}
}
