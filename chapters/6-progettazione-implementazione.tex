\chapter{Progettazione e implementazione}
\label{cap:progettazione-implementazione}

\intro{In questo capitolo, vengono descritte le scelte progettuali e le tecniche implementative utilizzate per realizzare il report delle vendite ed il sistema di raccomandazione. Si inizia con una panoramica del flusso delle attività, seguita da una descrizione delle tecnologie e degli strumenti utilizzati. Successivamente, si approfondiscono i vari componenti del sistema sviluppato.}

\section{Flusso delle attività}


\section{Tecnologie e strumenti}
\label{sec:tecnologie-strumenti}

Di seguito viene data una panoramica delle tecnologie e strumenti utilizzati.

\subsection{Tecnologia 1}
Descrizione Tecnologia 1.

\subsection{Tecnologia 2}
Descrizione Tecnologia 2


\section{Architettura del sistema}



\section{Preprocessing}
\label{sec:preprocessing}

\subsection{Riconoscimento delle marche}
\label{sec:recognition-brands}

Per poter analizzare le vendite in modo più dettagliato, è stato pensato di introdurre una colonna "Marca" nei dataset, che potesse contenere il nome della marca del prodotto presente in tale riga. Per ottenere ciò, era dunque necessario sviluppare un sistema di riconoscimento delle marche. Sono state provate le seguenti strategie:
\begin{itemize}
    \item \textbf{Riconoscimento tramite dizionario}: è stato creato un dizionario contenente le marche più comuni, ma si è rivelato poco efficace, poiché molte marche non erano presenti nel dizionario e il riconoscimento era limitato;
    \item \textbf{Riconoscimento tramite regex}: è stato provato a utilizzare delle espressioni regolari per riconoscere le marche, ma si è rivelato poco efficace, poiché molte marche non seguivano uno schema comune e il riconoscimento era limitato;
    \item \textbf{Riconoscimento tramite modelli di \gls{ml}\glsfirstoccur{}}: è stato pensato di creare un modello di \gls{ml} per riconoscere le marche, ma si è rivelato poco efficace, poiché il dataset non era sufficientemente grande e vario per addestrare un modello affidabile, e soprattuto non erano disponibili le etichette necessarie per un addestramento supervisionato;
    \item \textbf{Riconoscimento tramite modelli di \gls{ner}\glsfirstoccur{}}: è stato provato ad utilizzare un modello di \gls{ner} sulle descrizioni dei prodotti e a selezionare le entità segnalate di tipo \emph{ORG} (organizzazione), ma si è rivelato poco efficace, poiché il modello non era stato addestrato specificamente per questo compito e il riconoscimento era limitato;
    \item \textbf{Riconoscimento tramite \gls{llm}\glsfirstoccur{}}: è stato provato ad utilizzare un modello di linguaggio di grandi dimensioni (\gls{llm}) per riconoscere le marche, e si è rivelato mediamente efficace, ma il riconoscimento di ciascuna marca richiedeva troppo tempo e tantissime chiamate \gls{api}\glsfirstoccur{}; allora, è stato fatto un tentativo di raggruppamento di più descrizioni da inviare assieme per il riconoscimento di più marche contemporaneamente, ma si è rivelato poco efficace, poiché il modello ogni tanto si dimenticava di alcune marche o ne aggiungeva qualcuna in più, totalmente senza motivo (avvenivano cioè le cosidette "allucinazioni").
\end{itemize}

Dunque, si è deciso di non implementare il riconoscimento delle marche nel sistema di analisi automatizzato, poiché non era possibile garantire un riconoscimento affidabile e preciso.


\subsection{Riconoscimento delle categorie}
\label{sec:recognition-categories}

Per poter analizzare le vendite in modo più dettagliato, è stato pensato di introdurre una colonna "Categoria" nei dataset, che potesse contenere il nome della categoria del prodotto presente in tale riga. Per ottenere ciò, era dunque necessario sviluppare un sistema di riconoscimento delle categorie.

Escluse le opzioni già descritte nella sezione \ref{sec:recognition-brands} per il riconoscimento delle marche, è stato pensato di utilizzare un modello \gls{kmeans}\glsfirstoccur{} per raggruppare i prodotti in base alle loro descrizioni, in modo da ottenere delle categorie. Tuttavia, ciò si è rivelato poco efficace, poiché il modello non era in grado di raggruppare i prodotti in categorie in modo affidabile e preciso, e il numero di categorie era troppo elevato per poterle gestire manualmente.

Dunque, si è deciso di non implementare il riconoscimento delle categorie nel sistema di analisi automatizzato, poiché non era possibile garantire un riconoscimento affidabile e preciso.\\



\section{Language processing}
\label{sec:language-processing}

\section{Report}
\subsection{PDF}
\subsection{HTML}

\section{Invio di email}

\section{Le matrici di raccomandazione}
\subsection{Formato di archiviazione delle matrici}

\section{La predizione e rank fusion}

\section{Valutazione delle raccomandazioni}

\section{Collegamento con Google Cloud}
\subsection{Google Cloud Storage}
\subsection{Google Cloud Functions}
