\cleardoublepage
\phantomsection
\pdfbookmark{Sommario}{Sommario}
\begingroup
\let\clearpage\relax
\let\cleardoublepage\relax
\let\cleardoublepage\relax

\chapter*{Sommario}

Questo elaborato descrive l’attività svolta da Stefani Riccardo durante un tirocinio curriculare della durata di 320 ore presso l’azienda Oribea AI S.r.l.

Il progetto si inserisce negli ambiti della Business Intelligence e del Machine Learning, con l'obiettivo di sviluppare un Task AI per l'analisi delle vendite, utilizzando dati provenienti da database aziendali o dataset pubblici. Il sistema realizzato sfrutta un Large Language Model (LLM) per generare analisi automatiche, interpretabili e personalizzabili.

Inoltre, il progetto prevede anche lo sviluppo di un sistema di raccomandazione integrato in un apposito Task AI, che permetta di raccomandare prodotti ai clienti in base alle loro preferenze e comportamenti di acquisto, e viceversa di suggerire possibili clienti a cui proporre i prodotti, ottimizzando le strategie di marketing e vendita.

Prima della fase di sviluppo, è stato condotto uno studio approfondito delle tecnologie impiegate e dei concetti economici fondamentali per garantire la qualità delle analisi delle vendite e delle raccomandazioni prodotte. Le attività e le soluzioni adottate vengono illustrate nei capitoli successivi.

%\vfill

%\selectlanguage{english}
%\pdfbookmark{Abstract}{Abstract}
%\chapter*{Abstract}

%\selectlanguage{italian}

\endgroup

\vfill
