\chapter{Report delle vendite}
\label{cap:report-vendite}

\intro{In questo capitolo, vengono descritte le teorie e le tecniche utilizzate per l'analisi delle vendite, con particolare attenzione allo studio dei dati, delle statistiche utili e dei grafici generati. Viene inoltre discusso il beneficio dell'automazione dell'analisi delle vendite in confronto all'alternativa manuale.}


\section{Benefici dell’automazione dell’analisi delle vendite}

La \gls{busintel}\glsfirstoccur{} è un insieme di tecnologie e pratiche che consentono alle aziende di raccogliere, analizzare e presentare dati per supportare il processo decisionale. L'automazione dell'analisi delle vendite rappresenta un passo avanti significativo rispetto all'analisi manuale, offrendo numerosi vantaggi:
\begin{itemize}
    \item \textbf{Efficienza}: l'automazione consente di elaborare grandi volumi di dati in tempi ridotti, riducendo il tempo necessario per generare report e analisi;
    \item \textbf{Accuratezza}: le operazioni automatizzate riducono il rischio di errori umani, garantendo risultati più precisi e affidabili;
    \item \textbf{Scalabilità}: le soluzioni automatizzate possono gestire facilmente l'aumento dei dati e delle richieste di analisi, adattandosi alle esigenze aziendali in crescita;
    \item \textbf{Accessibilità}: i report automatizzati possono essere facilmente condivisi tra i membri del team e le parti interessate, migliorando la collaborazione e la comunicazione;
    \item \textbf{Personalizzazione}: le soluzioni automatizzate possono essere configurate per generare report specifici in base alle esigenze dell'azienda, consentendo un'analisi mirata.
\end{itemize}

Di conseguenza, compito di questo progetto è stato quello di sviluppare un sistema automatizzato per l'analisi delle vendite, in grado di generare report dettagliati e personalizzati in modo rapido ed efficiente. Questo sistema si basa su un processo di raccolta, elaborazione e visualizzazione dei dati, che consente di ottenere informazioni utili per prendere decisioni strategiche e migliorare le performance aziendali.



\section{Studio delle colonne del dataset}
\label{sec:studio-colonne-dataset}
Per poter analizzare le vendite, è fondamentale comprendere le colonne dei dataset dei quali si dispone. Inizialmente, è stato reso disponibile un unico dataset, che d'ora in poi denominerò \emph{orders-export}, fornito dall'azienda di e-commerce committente, dunque le colonne di questo dataset sono state studiate in dettaglio e prese come riferimento per l'analisi delle vendite e per la selezione dei successivi dataset. Le colonne originali di \emph{orders-export} sono le seguenti:
\begin{itemize}
    \item \textbf{Numero Ordine}: identificativo univoco dell'ordine;
    \item \textbf{Data Ordine}: data dell'ordine, che riporta anche il timestamp esatto;
    \item \textbf{ID Cliente}: identificativo univoco del cliente;
    \item \textbf{Nome Cliente}: nome del cliente;
    \item \textbf{Cognome Cliente}: cognome del cliente;
    \item \textbf{Company}: azienda del cliente, se presente;
    \item \textbf{SKU Prodotto}: codice univoco del prodotto (Stock Keeping Unit);
    \item \textbf{ID Prodotto}: identificativo univoco del prodotto;
    \item \textbf{Descrizione Prodotto}: descrizione del prodotto;
    \item \textbf{Quantità}: quantità di prodotto ordinata;
    \item \textbf{Prezzo Unitario}: prezzo unitario del prodotto;
    \item \textbf{Valore Riga}: importo totale dell'acquisto del prodotto. Ciò si deduce solo dai valori contenuti nella colonna, perchè il nominativo della stessa non è stato attribuito in modo corretto.
\end{itemize}

Dopo un primo sguardo alle colonne, è apparsa chiara la necessità di operare un \gls{preprocessing}\glsfirstoccur{} delle stesse, in modo da poter analizzare le vendite in modo standard ed efficace. Ne è l'esempio la colonna \emph{Valore Riga}, che non fa comprendere il suo contenuto a prima vista, e dunque necessita di un cambio di nominativo.

Dopo aver esaminato le colonne del dataset \emph{orders-export}, è stato sviluppato un sistema di preprocessing, descritto più nel dettaglio a livello implementativo nella sezione \S\ref{sec:preprocessing}. In questa fase, sono state scelte le seguenti colonne per l'analisi delle vendite, e il loro nome è stato tradotto in inglese per uniformità con il resto del progetto:
\begin{itemize}
    \item \textbf{Order ID}: identificativo univoco dell'ordine;
    \item \textbf{Order Timestamp}: data e ora esatta dell'ordine;
    \item \textbf{Customer ID}: identificativo univoco del cliente;
    \item \textbf{Customer Name}: nome e cognome del cliente;
    \item \textbf{Product SKU}: codice univoco del prodotto (Stock Keeping Unit);
    \item \textbf{Product Name}: nome o descrizione del prodotto;
    \item \textbf{Unit Price}: prezzo unitario del prodotto;
    \item \textbf{Quantity}: quantità di prodotto ordinata;
    \item \textbf{Total Price}: importo totale relativo all'acquisto del prodotto.
\end{itemize}

Queste colonne sono state dunque prese come riferimento per cercare ulteriori dataset, che potessero essere utili per l'analisi delle vendite. In particolare, sono stati cercati dataset pubblici nella piattaforma Kaggle che contenessero le stesse colonne o colonne riconducibili a quelle standardizzate di \emph{orders-export}. Sono stati trovati i seguenti dataset, nominati con il nome dell'utente che li ha caricati su Kaggle:
\begin{itemize}
    \item \textbf{Anwer};
    \item \textbf{Cornelius};
    \item \textbf{Dee};
    \item \textbf{Delikkaya};
    \item \textbf{Feroze};
    \item \textbf{Segura};
    \item \textbf{Shaw};
    \item \textbf{Swillm};
    \item \textbf{Vaghasiya}.
\end{itemize}

Ogni dataset è stato dunque trattato per ottenere le stesse colonne standardizzate, in modo da poterle confrontare e analizzare in modo efficace.
Inoltre, sono state successivamente aggiunte le seguenti colonne, ricavate dalla colonna "Order Timestamp", per facilitare alcune operazioni di analisi:
\begin{itemize}
    \item \textbf{Order Day}: giorno dell'ordine (1-31);
    \item \textbf{Order Week}: settimana dell'anno in cui è stato effettuato l'ordine (1-53);
    \item \textbf{Order Month}: mese in cui è stato effettuato l'ordine (1-12);
    \item \textbf{Order Year}: anno in cui è stato effettuato l'ordine (YYYY);
    \item \textbf{ISO Date}: data dell'ordine in formato ISO (YYYY-MM-DD);
    \item \textbf{ISO Month}: mese dell'ordine in formato ISO (YYYY-MM).
\end{itemize}

Fatto ciò, in parallelo con lo studio delle statistiche utili e dei grafici, si è visto necessario introdurre delle ulteriori colonne legate alla data, descrittive invece che numeriche, per allinearsi con il linguaggio naturale del report. Siccome il contenuto di tali colonne dipende dalla lingua del report, la loro creazione è ulteriormente descritta a livello implementativo nella sezione \ref{sec:language-processing}, dedicata alla gestione della lingua. Le colonne aggiuntive sono le seguenti:
\begin{itemize}
    \item \textbf{Date}: data dell'ordine in formato naturale, ad esempio "1 gennaio 2023";
    \item \textbf{Month}: mese dell'ordine in formato naturale, ad esempio "gennaio 2023";
    \item \textbf{Week}: settimana dell'anno in formato naturale, ottenuta unendo assieme le date del lunedì e della domenica di tale settimana, ad esempio "3 febbraio 2025 - 9 febbraio 2025".
\end{itemize}

A questo punto, i dataset preprocessati sono pronti per essere analizzati, e le colonne standardizzate sono pronte per essere utilizzate per ricavare le statistiche utili e i grafici.


\section{Valutazione delle statistiche utili}

Per poter analizzare le vendite, è fondamentale comprendere quali statistiche siano utili per ottenere informazioni significative. L'azienda committente, consapevole di ciò, ha fornito un esempio di report delle vendite, visibile nell'immagine \ref{fig:oribea-report-example}, che è stato utilizzato come base per la valutazione delle statistiche utili.

\begin{figure}[!h] 
    \centering 
    \includegraphics[width=0.9\columnwidth]{Oribea - Esempio di report delle vendite.png}
    \caption{Esempio di report delle vendite fornito da Oribea.}
    \label{fig:oribea-report-example}
\end{figure}

Il report contiene le seguenti statistiche:
\begin{itemize}
    \item \textbf{Fatturato Totale}: il totale delle vendite effettuate in un determinato periodo di tempo;
    \item \textbf{Ordini Unici}: il numero di ordini unici effettuati in un determinato periodo di tempo; questa statistica è necessaria perchè più righe contenenti lo stesso ordine possono essere presenti nel dataset, a causa della presenza di più prodotti nello stesso ordine;
    \item \textbf{Clienti Unici}: il numero di clienti unici che hanno effettuato ordini in un determinato periodo di tempo; questa statistica è necessaria perchè più righe contenenti lo stesso cliente possono essere presenti nel dataset, a causa della presenza di più ordini effettuati dallo stesso cliente.
\end{itemize}

Queste statistiche sono state scelte come base per l'analisi delle vendite, e sono state implementate nel sistema di analisi automatizzato. successivamente, sono state aggiunte altre statistiche utili, che sono state scelte in base alla loro rilevanza per l'analisi delle vendite e alla loro capacità di fornire informazioni significative. Le statistiche aggiuntive sono le seguenti:
\begin{itemize}
    \item \textbf{Prodotti Unici}: il numero di prodotti unici venduti in un determinato periodo di tempo; questa statistica è necessaria perchè più righe contenenti lo stesso prodotto possono essere presenti nel dataset, a causa della presenza di più ordini effettuati dallo stesso cliente;
    \item \textbf{Spesa Media per Ordine}: la spesa media per ordine, calcolata come il fatturato totale diviso per il numero di ordini unici; questa statistica è utile per comprendere quanto i clienti spendono mediamente per ogni ordine.
\end{itemize}

successivamente, sono state pensate delle altre elaborazioni statistiche, che però non sono state implementate nel sistema di analisi automatizzato per le ragioni spiegate di seguito.

\subsection{Riconoscimento delle marche}
\label{sec:recognition-brands}

Per poter analizzare le vendite in modo più dettagliato, è stato pensato di introdurre una colonna "Marca" nei dataset, che potesse contenere il nome della marca del prodotto. Per ottenere ciò, era dunque necessario sviluppare un sistema di riconoscimento delle marche. Sono state provate le seguenti strategie:
\begin{itemize}
    \item \textbf{Riconoscimento tramite dizionario}: è stato creato un dizionario contenente le marche più comuni, ma si è rivelato poco efficace, poiché molte marche non erano presenti nel dizionario e il riconoscimento era limitato;
    \item \textbf{Riconoscimento tramite regex}: è stato provato a utilizzare delle espressioni regolari per riconoscere le marche, ma si è rivelato poco efficace, poiché molte marche non seguivano uno schema comune e il riconoscimento era limitato;
    \item \textbf{Riconoscimento tramite modelli di machine learning}: è stato pensato di creare un modello di machine learning per riconoscere le marche, ma si è rivelato poco efficace, poiché il dataset non era sufficientemente grande e vario per addestrare un modello affidabile, e soprattuto non erano disponibili le etichette necessarie per un addestramento supervisionato;
    \item \textbf{Riconoscimento tramite modelli di Named Entity Recognition (NER)}: è stato provato a utilizzare un modello di NER sulle descrizioni dei prodotti e a selezionare le entità segnalate di tipo \emph{ORG} (organizzazione), ma si è rivelato poco efficace, poiché il modello non era stato addestrato specificamente per questo compito e il riconoscimento era limitato;
    \item \textbf{Riconoscimento tramite LLM}: è stato provato a utilizzare un modello di linguaggio di grandi dimensioni (LLM) per riconoscere le marche, ma si è rivelato poco efficace, non perchè non funzionasse, anzi, si è rivelato molto efficace, ma il riconoscimento di ciascuna marca richiedeva troppo tempo e tantissime chiamate \gls{API}\glsfirstoccur{}; è stato fatto un tentativo di raggruppamento di più descrizioni da inviare assieme per il riconoscimento di più marche contemporaneamente, ma si è rivelato poco efficace, poiché il modello ogni tanto si dimenticava di alcune marche o ne aggiungeva una in più, totalmente senza motivo (avvenivano cioè le cosidette "allucinazioni").
\end{itemize}

Dunque, si è deciso di non implementare il riconoscimento delle marche nel sistema di analisi automatizzato, poiché non era possibile garantire un riconoscimento affidabile e preciso.

\subsection{Riconoscimento delle categorie}
\label{sec:recognition-categories}

Per poter analizzare le vendite in modo più dettagliato, è stato pensato di introdurre una colonna "Categoria" nei dataset, che potesse contenere il nome della categoria del prodotto. Per ottenere ciò, era dunque necessario sviluppare un sistema di riconoscimento delle categorie.

Escluse le opzioni già descritte nella sezione \ref{sec:recognition-brands} per il riconoscimento delle marche, è stato pensato di utilizzare un modello k-means per raggruppare i prodotti in base alle loro descrizioni, in modo da ottenere delle categorie. Tuttavia, si è rivelato poco efficace, poiché il modello non era in grado di raggruppare i prodotti in categorie in modo affidabile e preciso, e il numero di categorie era troppo elevato per poterle gestire manualmente.

Dunque, si è deciso di non implementare il riconoscimento delle categorie nel sistema di analisi automatizzato, poiché non era possibile garantire un riconoscimento affidabile e preciso.\\


A valle di ciò, seppure siano state pensate delle statistiche utili aggiuntive, si è deciso di non implementarle nel sistema di analisi automatizzato. Sono rimaste dunque le cinque statistiche utili descritte in precedenza, che sono state implementate nel sistema di analisi automatizzato e sono state utilizzate per generare il report delle vendite. Queste statistiche sono state scelte in base alla loro rilevanza per l'analisi delle vendite e alla loro capacità di fornire informazioni significative, e sono state ritenute sufficienti per ottenere un'analisi delle vendite efficace e utile.



\section{Valutazione dei grafici utili}
