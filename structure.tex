        %%******************************************%%
        %%                                          %%
        %%        Modello di tesi di laurea         %%
        %%            di Andrea Giraldin            %%
        %%                                          %%
        %%             2 novembre 2012              %%
        %%                                          %%
        %%******************************************%%

\begin{document}
    \frontmatter
    \input{preface/title-page}
    \input{preface/copyright}
    % \input{preface/dedication} % Non ho una persona speciale a cui dedicare la tesi
    \input{preface/acknowledgements}
    \cleardoublepage
\phantomsection
\pdfbookmark{Sommario}{Sommario}
\begingroup
\let\clearpage\relax
\let\cleardoublepage\relax
\let\cleardoublepage\relax

\chapter*{Sommario}

Questo elaborato descrive l’attività svolta da Stefani Riccardo durante un tirocinio curriculare della durata di 320 ore presso l’azienda Oribea AI S.r.l.

Il progetto si inserisce negli ambiti della Business Intelligence e del Machine Learning, con l'obiettivo di sviluppare un Task AI per l'analisi delle vendite, utilizzando dati provenienti da database aziendali o dataset pubblici. Il sistema realizzato sfrutta un Large Language Model (LLM) per generare analisi automatiche, interpretabili e personalizzabili.

Inoltre, il progetto prevede anche lo sviluppo di un sistema di raccomandazione integrato in un apposito Task AI, che permetta di raccomandare prodotti ai clienti in base alle loro preferenze e comportamenti di acquisto, e viceversa di suggerire possibili clienti a cui proporre i prodotti, ottimizzando le strategie di marketing e vendita.

Prima della fase di sviluppo, è stato condotto uno studio approfondito delle tecnologie impiegate e dei concetti economici fondamentali per garantire la qualità delle analisi delle vendite e delle raccomandazioni prodotte. Le attività e le soluzioni adottate vengono illustrate nei capitoli successivi.

%\vfill

%\selectlanguage{english}
%\pdfbookmark{Abstract}{Abstract}
%\chapter*{Abstract}

%\selectlanguage{italian}

\endgroup

\vfill

    \input{preface/table-of-contents}
    \cleardoublepage

    \mainmatter
    \chapter{Introduzione}
\label{cap:introduzione}

\intro{In questo capitolo verrà descritta l’azienda proponente del tirocinio, il way of working, l’organizzazione del testo e delle convenzioni tipografiche impostate.}\\

\noindent Introduzione al contesto applicativo.\\

\noindent Esempio di utilizzo di un termine nel glossario \\
\gls{api}. \\

\noindent Esempio di citazione in linea \\
\cite{site:agile-manifesto}. \\

\noindent Esempio di citazione nel pie' di pagina \\
citazione\footcite{womak:lean-thinking} \\

\section{L'azienda}

Oribea AI S.r.l. è una startup innovativa fondata nel 2024 nella Repubblica di San Marino, in seguito alla separazione dall'azienda di e-commerce ITTweb. La missione di Oribea è fornire soluzioni avanzate di intelligenza artificiale per migliorare l'efficienza e la produttività delle aziende, con un focus particolare sull'implementazione di \gls{llm}\glsfirstoccur{} e agenti intelligenti.
Tra i principali prodotti sviluppati da Oribea vi è l'AI Agent Builder, uno strumento che consente alle imprese di creare e integrare agenti intelligenti personalizzati nei propri processi aziendali. Questi agenti sono progettati per automatizzare attività ripetitive, migliorare la comunicazione interna ed esterna e supportare la presa di decisioni attraverso l'analisi avanzata dei dati. L'AI Agent Builder si distingue per la sua capacità di adattarsi alle specifiche esigenze di ciascuna azienda, offrendo soluzioni su misura che sfruttano le potenzialità degli \gls{llm}.
Inoltre, Oribea sta sviluppando un Sistema Intelligente, concepito per fungere da piattaforma centrale nell'orchestrazione delle attività aziendali. Questo sistema mira a integrare diverse applicazioni e servizi, facilitando la gestione dei processi e migliorando la coerenza e l'efficienza operativa. Attraverso l'uso di tecnologie avanzate di intelligenza artificiale, il Sistema Intelligente di Oribea promette di trasformare il modo in cui le aziende operano, rendendo i processi più fluidi e reattivi alle esigenze del mercato.
La scelta di stabilire la sede a San Marino non è casuale: la Repubblica si sta posizionando come un hub per l'innovazione tecnologica, offrendo un ambiente favorevole allo sviluppo e alla sperimentazione di nuove tecnologie. In questo contesto, Oribea beneficia di un ecosistema dinamico e di una rete di collaborazioni che favoriscono la crescita e l'innovazione.
In sintesi, l'aziemda rappresenta un esempio di come le startup possano contribuire significativamente all'evoluzione del panorama tecnologico, offrendo soluzioni innovative che rispondono alle sfide contemporanee delle aziende. La sua focalizzazione sull'intelligenza artificiale applicata ai processi aziendali la rende un attore rilevante nel contesto della trasformazione digitale.

\begin{figure}
    \centering
    \includegraphics[width=0.5\textwidth]{oribea-logo.png}
    \caption{Logo di Oribea AI S.r.l.}
    \label{fig:oribea-logo}
\end{figure}

\section{L'idea}

L'idea dello stage è nata dalla necessità di sviluppare un sistema che consenta di generare automaticamente un report di analisi delle vendite per un'azienda di e-commerce. Questo report deve essere generato in modo autonomo, senza la necessità di intervento umano, e deve essere in grado di analizzare i dati delle vendite, identificare tendenze e fornire raccomandazioni per migliorare le performance aziendali.

In aggiunta, dallo stesso dataset delle vendite, il sistema deve essere in grado di generare un sistema di raccomandazioni per i clienti, suggerendo prodotti in base alle loro preferenze e comportamenti di acquisto. Questo approccio mira a migliorare l'esperienza del cliente e aumentare le vendite attraverso raccomandazioni personalizzate. Il sistema deve anche suggerire possibili clienti a cui proporre i prodotti, in modo da ottimizzare le strategie di marketing e vendita.
L'obiettivo finale è quello di creare un sistema integrato che possa automatizzare e ottimizzare i processi di analisi delle vendite e raccomandazione dei prodotti, contribuendo così a migliorare l'efficienza operativa dell'azienda e a massimizzare le opportunità di vendita.

Ho scelto questo progetto di stage con il desiderio di approfondire le mie conoscenze nel campo dell'intelligenza artificiale e del machine learning, gli argomenti di cui sono più interessato, in particolare nell'ambito dell'analisi dei dati e delle raccomandazioni personalizzate.

\section{Organizzazione del testo}

\subsection{Struttura del documento}
\label{sec:organizzazione-testo}
Il presente documento è suddiviso in otto capitoli il cui contenuto è brevemente riassunto in seguito:

\begin{description}
    \item[{\hyperref[cap:descrizione-stage]{Il secondo capitolo}}] descrive nel dettaglio il progetto di stage, le tecnologie utilizzate e il modo di lavorare dell'azienda; inoltre, viene fornita un'analisi dei rischi e delle soluzioni adottate per affrontarli;
    
    \item[{\hyperref[cap:analisi-requisiti]{Il terzo capitolo}}] approfondisce l'analisi dei requisiti del sistema, con particolare attenzione alla definizione dei casi d'uso e dei requisiti funzionali e non funzionali, con apposito tracciamento;
    
    \item[{\hyperref[cap:report-vendite]{Il quarto capitolo}}] approfondisce la teoria che sta ala base del report di analisi delle vendite, con particolare attenzione allo studio dei dati, delle tecniche di analisi utilizzate e dei grafici generati;
    
    \item[{\hyperref[cap:sistema-raccomandazione]{Il quinto capitolo}}] approfondisce la teoria che sta alla base del sistema di raccomandazione, con particolare attenzione allo studio di sistemi reali e delle tecniche di raccomandazione e valutazione dei risultati utilizzate;
    
    \item[{\hyperref[cap:progettazione-implementazione]{Il sesto capitolo}}] approfondisce la progettazione e l'implementazione del sistema, con particolare attenzione alla scelta delle tecnologie utilizzate e al loro utilizzo;
    
    \item[{\hyperref[cap:verifica-validazione]{Il settimo capitolo}}] approfondisce le attività di verifica e validazione del sistema, con particolare attenzione ai test di unità sviluppati e all'approccio adottato per i test riguardanti l'LLM;
    
    \item[{\hyperref[cap:conclusioni]{L'ottavo capitolo}}] rappresenta una sintesi finale del lavoro svolto durante il periodo di tirocinio, descrivendo eventuali successi e difficoltà incontrate durante il percorso. Vengono inoltre analizzati i risultati ottenuti rispetto agli obiettivi iniziali così come l’insieme di competenze teoriche e pratiche acquisite nel corso del progetto. Il documento si conclude con una riflessione critica sull’operato e sulla crescita personale e professionale del laureando Riccardo Stefani durante il tirocinio. Infine, viene fornita una panoramica delle prospettive future per il progetto e per l'azienda, evidenziando le opportunità di sviluppo e miglioramento.

\end{description}

\subsection{Convenzioni tipografiche}
\label{sec:convenzioni-tipografiche}

In merito alla redazione del presente documento, sono state adottate le seguenti convenzioni tipografiche:
\begin{itemize}
	\item Gli acronimi, le abbreviazioni e i termini ambigui o di uso non comune menzionati vengono definiti nel glossario, situato alla fine del presente documento;

	\item Per la prima occorrenza dei termini riportati nel glossario viene utilizzato il seguente stile: \gls{api}\glsfirstoccur{};

	\item I termini particolarmente rilevanti in una sezione e quelli in lingua straniera non di uso comune o facenti parti del gergo tecnico sono evidenziati con il carattere \emph{corsivo}, fatta eccezione per le occorrenze presenti nei titoli delle sezioni o nelle didascalie;

	\item I comandi di terminale, i frammenti di codice sorgente e i nomi di file o directory sono evidenziati con il carattere \texttt{monospaziato}.

\end{itemize}

    %\input{chapters/processi} % Processi e metodologie, non consigliato da Zanella. Io lo scriverò nella progettazione
    \chapter{Descrizione dello stage}
\label{cap:descrizione-stage}

\intro{In questo capitolo viene presentata una descrizione dell’idea e delle tecnologie utilizzate durante il percorso di stage, e una panoramica del modo di lavorare dell'azienda. Viene inoltre presentata un'analisi dei rischi e delle problematiche riscontrate durante lo sviluppo del progetto.}\\

\section{Introduzione al progetto}

Nel contesto attuale, i report delle vendite e i sistemi di raccomandazione rivestono un ruolo fondamentale per le aziende che operano in mercati competitivi e digitalizzati. I report delle vendite consentono di monitorare l’andamento commerciale, identificare trend, valutare le performance dei prodotti e prendere decisioni strategiche basate su dati concreti. Questi strumenti permettono di individuare rapidamente eventuali criticità o opportunità di crescita, ottimizzando così le strategie di vendita e marketing.\\

Parallelamente, i sistemi di raccomandazione sono diventati essenziali per migliorare l’esperienza utente e incrementare le vendite, soprattutto nelle piattaforme di e-commerce e nei servizi digitali. Attraverso l’analisi dei dati di acquisto e delle preferenze degli utenti, questi sistemi suggeriscono prodotti o servizi personalizzati, aumentando la probabilità di acquisto e la fidelizzazione del cliente. L’integrazione di report delle vendite e sistemi di raccomandazione consente quindi alle aziende di offrire un servizio più mirato ed efficiente, rafforzando la propria posizione sul mercato.\\

In questo contesto, la startup Oribea, che propone soluzioni digitali intelligenti per le aziende, ha ricevuto una commissione da parte di un'azienda di e-commerce per la creazione di strumenti che automatizzino la generazione di report delle vendite e sistemi di raccomandazione. Oribea ha quindi deciso di integrare queste due funzionalità all’interno della propria piattaforma omonima, che consente di usufruire delle cosiddette Task: strumenti che permettono di automatizzare e semplificare specifiche attività aziendali quotidiane.\\
Il progetto di stage si inserisce in questo scenario, con l'obiettivo di sviluppare una Task che consenta di generare un report delle vendite e un sistema di raccomandazione a partire da un file degli ordini, e un’altra Task che permetta di sfruttare il sistema di raccomandazione per generare suggerimenti personalizzati per un determinato cliente o prodotto.\\

Le Task di Oribea sono implementate come funzioni serverless, ovvero funzioni che vengono eseguite in modo autonomo e scalabile, senza la necessità di gestire l'infrastruttura sottostante, e sono caricate ed eseguite su Google \gls{cloudfunctions}\glsfirstoccur{}. Lo stato delle funzioni viene salvato in Google \gls{cloudstorage}\glsfirstoccur{}. Il compito di stage prevede quindi lo sviluppo di due Cloud Functions serverless che implementano le funzionalità richieste, la loro comunicazione tramite Cloud Storage e la loro integrazione con la piattaforma Oribea.\\


\section{Way of working e strumenti utilizzati}
\label{sec:way-of-working}

L’azienda Oribea adotta un modello di sviluppo \gls{agile}\glsfirstoccur{}, con l’obiettivo di monitorare e controllare il progetto in modo flessibile e continuo, suddividendo le attività in piccoli incrementi e con una collaborazione asincrona e distribuita.\\
In particolare, il modello di sviluppo adottato è \gls{scrum}\glsfirstoccur{}, che prevede la suddivisione del progetto in sprint, ovvero periodi di tempo di durata fissa, in cui vengono pianificate le attività da svolgere e i relativi obiettivi da raggiungere. All'inizio del progetto è stato scelto di comune accordo la lunghezza per gli sprint di una settimana. Al termine di ogni sprint, è stato effettuato un incontro approfondito con il tutor aziendale, per discutere lo stato di avanzamento del progetto e le attività da svolgere per il successivo sprint.\\
L’obiettivo del modello è dare maggiore importanza al ciclo di vita del software e dei processi correlati, piuttosto che al prodotto finale, con il fine ultimo di migliorare la qualità del prodotto stesso.\\

Gli strumenti principali utilizzati includono:
\begin{itemize}
    \item \textbf{Visual Studio Code}: per la scrittura e la modifica del codice sorgente;
    \item \textbf{Git}: per il versionamento del codice e la gestione delle modifiche;
    \item \textbf{GitHub}: per la gestione del codice sorgente e la collaborazione tra sviluppatori;
    \item \textbf{Slack}: per la comunicazione interna e la gestione dei progetti;
    \item \textbf{Notion}: per la documentazione del progetto;
    \item \textbf{StarUML}: per la creazione dei diagrammi UML.
\end{itemize}

\section{Analisi preventiva dei rischi}

Durante la fase di analisi iniziale sono stati individuati alcuni possibili rischi a cui si potrà andare incontro.
Si è quindi proceduto a elaborare delle possibili soluzioni per far fronte a tali rischi.\\

\begin{risk}{Cambiamento dei requisiti in corso di sviluppo}
    \riskdescription{i requisiti del progetto potrebbero cambiare durante lo sviluppo dello stesso. Alcuni dei motivi potrebbero essere delle modifiche alle esigenze del progetto, un errore di analisi iniziale oppure motivi legati alle tecnologie scelte}
    \risksolution{è utile adottare l’utilizzo di una metodologia agile per tollerare i cambiamenti ai requisiti. È inoltre di fondamentale importanza mantenere una comunicazione costante con il tutor aziendale, in modo da procedere con lo sviluppo del progetto con maggior sicurezza e in modo più organizzato}
    \label{risk:requirements-change} 
\end{risk}

\begin{risk}{Numero e qualità dei dataset di addestramento}
    \riskdescription{per il corretto funzionamento del sistema di raccomandazione è necessario disporre non solo di dataset di qualità, ma anche di una quantità sufficiente di dataset di addestramento che rappresentino diversi scenari e casi d’uso. L’utilizzo di un solo dataset, anche se ben strutturato, potrebbe non essere sufficiente per addestrare il sistema in modo efficace e garantire risultati affidabili e generalizzabili. Oribea ha fornito un dataset di esempio, ma la presenza di un unico dataset limita la capacità del sistema di adattarsi a situazioni diverse e può ridurre la robustezza delle raccomandazioni generate}
    \risksolution{è possibile utilizzare tecniche di data augmentation per aumentare la varietà e la quantità dei dati disponibili, oppure ricorrere a dataset pubblici pertinenti al dominio del progetto. Inoltre, si può valutare la raccolta di ulteriori dataset reali, anche in collaborazione con l’azienda committente, per coprire un maggior numero di casi d’uso e migliorare la generalizzazione del sistema di raccomandazione}
    \label{risk:dataset-absence}
\end{risk}

\begin{risk}{Risposta dell'LLM imprecisa}
    \riskdescription{l'LLM (Large Language Model) utilizzato per generare il report delle vendite potrebbe produrre risposte imprecise o non pertinenti, a causa di una cattiva interpretazione del file degli ordini o di una mancanza di contesto. Questo potrebbe portare a report errati o fuorvianti, che potrebbero influenzare negativamente le decisioni aziendali. Inoltre, nell'ambito del preprocessing dei dati, l'LLM potrebbe non essere in grado di riconoscere correttamente le colonne da etichettare semanticamente, causando errori nella generazione del report e del sistema di raccomandazione}
    \risksolution{è importante testare l'LLM con diversi esempi di file degli ordini e verificare la qualità delle risposte generate. In caso di risposte imprecise, è possibile migliorare il prompt utilizzato per l'LLM, fornendo istruzioni più chiare e dettagliate. Inoltre, è utile implementare un sistema di validazione dei dati in ingresso, per garantire che il file degli ordini sia formattato correttamente e contenga le informazioni necessarie. Infine, si può considerare l'utilizzo di tecniche di post-elaborazione per correggere eventuali errori nelle risposte dell'LLM}
    \label{risk:bad-llm-response}
\end{risk}

\begin{risk}{Ritardi nello sviluppo}
    \riskdescription{durante lo sviluppo del progetto potrebbero verificarsi imprevisti come problemi tecnici o un’eccessiva complessità del tema affrontato, che potrebbero portare a ritardi nello sviluppo del prodotto}
    \risksolution{è importante pianificare il progetto in modo realistico, tenendo conto dei tempi necessari per lo sviluppo e per la risoluzione di eventuali problemi. Inoltre, è utile monitorare costantemente lo stato di avanzamento del progetto e adottare una metodologia Agile per gestire i cambiamenti e le difficoltà che potrebbero sorgere. In caso di ritardi significativi, è possibile rivedere le priorità del progetto e concentrarsi sulle funzionalità essenziali, rimandando eventuali miglioramenti o funzionalità aggiuntive a fasi successive}
    \label{risk:development-delay}
\end{risk}

\begin{risk}{Inesperienza tecnologica}
    \riskdescription{il progetto prevede l’utilizzo di tecnologie di cui non si ha piena esperienza e conoscenza, soprattutto le tecnologie legate al \gls{cloudcomputing}\glsfirstoccur{}, rendendo più difficoltosa la comprensione e l’applicazione delle stesse in fase di implementazione}
    \risksolution{è utile dedicare del tempo all'apprendimento delle tecnologie utilizzate, consultando la documentazione ufficiale, seguendo corsi online o chiedendo supporto a colleghi più esperti. Inoltre, è importante non esitare a chiedere aiuto al tutor aziendale in caso di difficoltà, in modo da risolvere i problemi in modo tempestivo e ridurre al minimo l'impatto sull'avanzamento del progetto. Infine, si può considerare l'utilizzo di risorse online come forum e community per ottenere supporto e condividere esperienze con altri sviluppatori}
    \label{risk:inexperience}
\end{risk}

    \chapter{Analisi dei requisiti}
\label{cap:analisi-requisiti}

\intro{In questa sezione vengono analizzati i requisiti del progetto e ne viene data un’analisi ad alto livello, combinando
una visione concettuale con una visione pratica ed implementativa. Vengono inoltre descritti i casi d’uso e i requisiti
individuati, con l’obiettivo di fornire una visione generale del sistema e delle sue funzionalità, in modo semplice e
comprensibile.}\\

\section{Obiettivi dello stage}

Gli obiettivi fondamentali da raggiungere durante il periodo di tirocinio, stilati
in accordo con il tutor aziendale ed inseriti nel documento Piano di Lavoro, sono
identificati dalla seguente notazione:
\begin{itemize}
    \item \textbf{OO}: obiettivi obbligatori, vincolanti in quanto obiettivo primario richiesto dal committente;
    \item \textbf{OD}: obiettivi desiderabili, non vincolanti o strettamente necessari, ma dal riconoscibile valore aggiunto.
    \item \textbf{OZ}: obiettivi opzionali, non vincolanti e non necessari, ma che potrebbero essere implementati in un secondo momento.
\end{itemize}

Alle sigle precedentemente indicate seguirà un numero progressivo, identificando così
tutti gli obiettivi.\\
Essi sono i seguenti:
\begin{itemize}
    \item Obbligatori:
    \begin{itemize}
        \item \textbf{OO1}: acquisizione di competenze pratiche su Oribea/DialogSphere;
        \item \textbf{OO2}: connessione a database e gestione dati aziendali o pubblici;
        \item \textbf{OO3}: implementazione di un Task AI che genera un sistema di raccomandazioni e report automatico basato su analisi delle vendite;
        \item \textbf{OO4}: implementazione di un Task AI che permette di raccomandare prodotti ad un cliente in base ai suoi dati di vendita, e viceversa di raccomandare clienti ad un prodotto;
        \item \textbf{OO5}: generazione automatica di report con output coerente, chiaro e adattabile;
        \item \textbf{OO6}: testing e documentazione completa del prototipo.
    \end{itemize}
    \item Desiderabili:
    \begin{itemize}
        \item \textbf{OD1}: ottimizzazione del Task AI per performance e scalabilità;
        \item \textbf{OD2}: personalizzazione dinamica dei prompt per casi d’uso differenti;
        \item \textbf{OD3}: integrazione con strumenti di visualizzazione o interfacce utente.
    \end{itemize}
    \item Opzionali:
    \begin{itemize}
        \item \textbf{OZ1}: sviluppo di un chatbot o di una dashboard interattiva per l'interazione con il sistema di raccomandazioni e report;
        \item \textbf{OZ2}: sperimentazione di tecniche di Explainable AI (XAI) per la trasparenza dei risultati;
        \item \textbf{OZ3}: esportazione automatica dei report in PDF/HTML o invio via e-mail.
    \end{itemize}
\end{itemize}


\section{Casi d'uso}

Per lo studio dei casi di utilizzo delle due task sono stati creati dei diagrammi.\\
I diagrammi dei casi d'uso (in inglese \emph{Use Case Diagram}) sono diagrammi di tipo \gls{uml} dedicati alla descrizione delle funzioni o servizi offerti da un sistema, così come sono percepiti e utilizzati dagli attori che interagiscono col sistema stesso. Nel mio caso, l'unico attore che interagisce con le due task è l'utente semplice, che rappresenta un generico utente autenticato nella piattaforma Oribea.\\
Ciascun caso d’uso riporta gli attori coinvolti, le sue precondizioni, la sua descrizione, le sue postcondizioni ed eventuali sottocasi d’uso, inclusioni, specializzazioni e
scenari alternativi.\\
I casi d’uso che sono stati definiti sono i seguenti:


\begin{usecase}{1}{Caricamento file CSV}

\begin{figure}[!h] 
    \centering 
    \includegraphics[width=0.9\columnwidth]{usecase/UC1 - Caricamento file CSV.png}
    \caption{UC1 - Caricamento file CSV}
\end{figure}

\usecaseactors{Utente}
\usecasepre{Lo sviluppatore è entrato nel plug-in di simulazione all'interno dell'IDE}
\usecasedesc{La finestra di simulazione mette a disposizione i comandi per configurare, registrare o eseguire un test}
\usecasepost{Il sistema è pronto per permettere una nuova interazione}
\label{uc:caricamento-file-csv}
\end{usecase}


\begin{usecase}{2}{Selezione lingua per il report}

\begin{figure}[!h] 
    \centering 
    \includegraphics[width=0.9\columnwidth]{usecase/UC2 - Selezione lingua per il report.png}
    \caption{UC2 - Selezione lingua per il report}
\end{figure}

\usecaseactors{Utente}
\usecasepre{Lo sviluppatore è entrato nel plug-in di simulazione all'interno dell'IDE}
\usecasedesc{La finestra di simulazione mette a disposizione i comandi per configurare, registrare o eseguire un test}
\usecasepost{Il sistema è pronto per permettere una nuova interazione}
\usespecial{UC3, UC4, UC5, UC6, UC7}
\label{uc:selezione-lingua-report}
\end{usecase}


\begin{usecase}{3}{Selezione lingua italiano per il report}

\usecaseactors{Utente}
\usecasepre{Lo sviluppatore è entrato nel plug-in di simulazione all'interno dell'IDE}
\usecasedesc{La finestra di simulazione mette a disposizione i comandi per configurare, registrare o eseguire un test}
\usecasepost{Il sistema è pronto per permettere una nuova interazione}
\label{uc:selezione-lingua-italiano-report}
\end{usecase}


\begin{usecase}{4}{Selezione lingua inglese per il report}

\usecaseactors{Utente}
\usecasepre{Lo sviluppatore è entrato nel plug-in di simulazione all'interno dell'IDE}
\usecasedesc{La finestra di simulazione mette a disposizione i comandi per configurare, registrare o eseguire un test}
\usecasepost{Il sistema è pronto per permettere una nuova interazione}
\label{uc:selezione-lingua-inglese-report}
\end{usecase}


\begin{usecase}{5}{Selezione lingua francese per il report}

\usecaseactors{Utente}
\usecasepre{Lo sviluppatore è entrato nel plug-in di simulazione all'interno dell'IDE}
\usecasedesc{La finestra di simulazione mette a disposizione i comandi per configurare, registrare o eseguire un test}
\usecasepost{Il sistema è pronto per permettere una nuova interazione}
\label{uc:selezione-lingua-francese-report}
\end{usecase}


\begin{usecase}{6}{Selezione lingua spagnolo per il report}

\usecaseactors{Utente}
\usecasepre{Lo sviluppatore è entrato nel plug-in di simulazione all'interno dell'IDE}
\usecasedesc{La finestra di simulazione mette a disposizione i comandi per configurare, registrare o eseguire un test}
\usecasepost{Il sistema è pronto per permettere una nuova interazione}
\label{uc:selezione-lingua-spagnolo-report}
\end{usecase}


\begin{usecase}{7}{Selezione lingua tedesco per il report}

\usecaseactors{Utente}
\usecasepre{Lo sviluppatore è entrato nel plug-in di simulazione all'interno dell'IDE}
\usecasedesc{La finestra di simulazione mette a disposizione i comandi per configurare, registrare o eseguire un test}
\usecasepost{Il sistema è pronto per permettere una nuova interazione}
\label{uc:selezione-lingua-tedesco-report}
\end{usecase}


\begin{usecase}{8}{Selezione valuta per il report}

\begin{figure}[!h] 
    \centering 
    \includegraphics[width=0.9\columnwidth]{usecase/UC8 - Selezione valuta per il report.png}
    \caption{UC8 - Selezione valuta per il report}
\end{figure}

\usecaseactors{Utente}
\usecasepre{Lo sviluppatore è entrato nel plug-in di simulazione all'interno dell'IDE}
\usecasedesc{La finestra di simulazione mette a disposizione i comandi per configurare, registrare o eseguire un test}
\usecasepost{Il sistema è pronto per permettere una nuova interazione}
\usespecial{UC9, UC10, UC11}
\label{uc:selezione-valuta-report}
\end{usecase}


\begin{usecase}{9}{Selezione valuta euro per il report}

\usecaseactors{Utente}
\usecasepre{Lo sviluppatore è entrato nel plug-in di simulazione all'interno dell'IDE}
\usecasedesc{La finestra di simulazione mette a disposizione i comandi per configurare, registrare o eseguire un test}
\usecasepost{Il sistema è pronto per permettere una nuova interazione}
\label{uc:selezione-valuta-euro-report}
\end{usecase}


\begin{usecase}{10}{Selezione valuta dollaro per il report}

\usecaseactors{Utente}
\usecasepre{Lo sviluppatore è entrato nel plug-in di simulazione all'interno dell'IDE}
\usecasedesc{La finestra di simulazione mette a disposizione i comandi per configurare, registrare o eseguire un test}
\usecasepost{Il sistema è pronto per permettere una nuova interazione}
\label{uc:selezione-valuta-dollaro-report}
\end{usecase}


\begin{usecase}{11}{Selezione valuta sterlina per il report}

\usecaseactors{Utente}
\usecasepre{Lo sviluppatore è entrato nel plug-in di simulazione all'interno dell'IDE}
\usecasedesc{La finestra di simulazione mette a disposizione i comandi per configurare, registrare o eseguire un test}
\usecasepost{Il sistema è pronto per permettere una nuova interazione}
\label{uc:selezione-valuta-sterlina-report}
\end{usecase}


\begin{usecase}{12}{Visualizzazione risultato della task}

\begin{figure}[!h]
    \centering 
    \includegraphics[width=0.9\columnwidth]{usecase/UC12 - Visualizzazione risultato della task.png}
    \caption{UC12 - Visualizzazione risultato della task}
\end{figure}

\usecaseactors{Utente}
\usecasepre{Lo sviluppatore è entrato nel plug-in di simulazione all'interno dell'IDE}
\usecasedesc{La finestra di simulazione mette a disposizione i comandi per configurare, registrare o eseguire un test}
\usecasepost{Il sistema è pronto per permettere una nuova interazione}
\usecasealt{UC13}
\useinclu{UC12.1, UC12.2, UC12.3, UC12.4}  % Ricordati di modificare qua, se dovesse cambiare qualcosa
\label{uc:visualizzazione-risultato-task}
\end{usecase}


% 12.1, 12.2, 12.3 e 12.4 non li metto ancora perchè ho una domanda


\begin{usecase}{13}{Visualizzazione errore nell'elaborazione della task}

\usecaseactors{Utente}
\usecasepre{Lo sviluppatore è entrato nel plug-in di simulazione all'interno dell'IDE}
\usecasedesc{La finestra di simulazione mette a disposizione i comandi per configurare, registrare o eseguire un test}
\usecasepost{Il sistema è pronto per permettere una nuova interazione}
\label{uc:visualizzazione-errore-task}
\end{usecase}


% UC14, UC15, e UC16 non li metto ancora perchè ho una domanda


\begin{usecase}{17}{Inserimento del nome del file CSV degli ordini originale}

\begin{figure}[!h] 
    \centering 
    \includegraphics[width=0.9\columnwidth]{usecase/UC17 - Inserimento nome CSV degli ordini originale.png}
    \caption{UC17 - Inserimento del nome del file CSV degli ordini originale}
\end{figure}

\usecaseactors{Utente}
\usecasepre{Lo sviluppatore è entrato nel plug-in di simulazione all'interno dell'IDE}
\usecasedesc{La finestra di simulazione mette a disposizione i comandi per configurare, registrare o eseguire un test}
\usecasepost{Il sistema è pronto per permettere una nuova interazione}
\label{uc:inserimento-nome-file-csv}
\end{usecase}


\begin{usecase}{18}{Selezione tipo di elemento per cui raccomandare}

\begin{figure}[!h] 
    \centering 
    \includegraphics[width=0.9\columnwidth]{usecase/UC18 - Selezione tipo di elemento per cui raccomandare.png}
    \caption{UC18 - Selezione tipo di elemento per cui raccomandare}
\end{figure}

\usecaseactors{Utente}
\usecasepre{Lo sviluppatore è entrato nel plug-in di simulazione all'interno dell'IDE}
\usecasedesc{La finestra di simulazione mette a disposizione i comandi per configurare, registrare o eseguire un test}
\usecasepost{Il sistema è pronto per permettere una nuova interazione}
\usespecial{UC19, UC20}
\label{uc:selezione-tipo-elemento}
\end{usecase}


\begin{usecase}{19}{Selezione di raccomandare prodotti per un cliente}

\usecaseactors{Utente}
\usecasepre{Lo sviluppatore è entrato nel plug-in di simulazione all'interno dell'IDE}
\usecasedesc{La finestra di simulazione mette a disposizione i comandi per configurare, registrare o eseguire un test}
\usecasepost{Il sistema è pronto per permettere una nuova interazione}
\label{uc:selezione-raccomandare-prodotti}
\end{usecase}


\begin{usecase}{20}{Selezione di raccomandare clienti per un prodotto}

\usecaseactors{Utente}
\usecasepre{Lo sviluppatore è entrato nel plug-in di simulazione all'interno dell'IDE}
\usecasedesc{La finestra di simulazione mette a disposizione i comandi per configurare, registrare o eseguire un test}
\usecasepost{Il sistema è pronto per permettere una nuova interazione}
\label{uc:selezione-raccomandare-clienti}
\end{usecase}


\begin{usecase}{21}{Inserimento nome elemento per cui raccomandare}

\begin{figure}[!h] 
    \centering 
    \includegraphics[width=0.9\columnwidth]{usecase/UC21 - Inserimento nome elemento a cui raccomandare.png}
    \caption{UC21 - Inserimento nome elemento per cui raccomandare}
\end{figure}

\usecaseactors{Utente}
\usecasepre{Lo sviluppatore è entrato nel plug-in di simulazione all'interno dell'IDE}
\usecasedesc{La finestra di simulazione mette a disposizione i comandi per configurare, registrare o eseguire un test}
\usecasepost{Il sistema è pronto per permettere una nuova interazione}
\label{uc:inserimento-nome-elemento}
\end{usecase}


\begin{usecase}{22}{Selezione lingua per la raccomandazione}

\begin{figure}[!h] 
    \centering 
    \includegraphics[width=0.9\columnwidth]{usecase/UC22 - Selezione lingua per la raccomandazione.png}
    \caption{UC22 - Selezione lingua per la raccomandazione}
\end{figure}

\usecaseactors{Utente}
\usecasepre{Lo sviluppatore è entrato nel plug-in di simulazione all'interno dell'IDE}
\usecasedesc{La finestra di simulazione mette a disposizione i comandi per configurare, registrare o eseguire un test}
\usecasepost{Il sistema è pronto per permettere una nuova interazione}
\usespecial{UC23, UC24, UC25, UC26, UC27}
\label{uc:selezione-lingua-raccomandazione}
\end{usecase}


\begin{usecase}{23}{Selezione lingua italiano per la raccomandazione}

\usecaseactors{Utente}
\usecasepre{Lo sviluppatore è entrato nel plug-in di simulazione all'interno dell'IDE}
\usecasedesc{La finestra di simulazione mette a disposizione i comandi per configurare, registrare o eseguire un test}
\usecasepost{Il sistema è pronto per permettere una nuova interazione}
\label{uc:selezione-lingua-italiano-raccomandazione}
\end{usecase}


\begin{usecase}{24}{Selezione lingua inglese per la raccomandazione}

\usecaseactors{Utente}
\usecasepre{Lo sviluppatore è entrato nel plug-in di simulazione all'interno dell'IDE}
\usecasedesc{La finestra di simulazione mette a disposizione i comandi per configurare, registrare o eseguire un test}
\usecasepost{Il sistema è pronto per permettere una nuova interazione}
\label{uc:selezione-lingua-inglese-raccomandazione}
\end{usecase}


\begin{usecase}{25}{Selezione lingua francese per la raccomandazione}

\usecaseactors{Utente}
\usecasepre{Lo sviluppatore è entrato nel plug-in di simulazione all'interno dell'IDE}
\usecasedesc{La finestra di simulazione mette a disposizione i comandi per configurare, registrare o eseguire un test}
\usecasepost{Il sistema è pronto per permettere una nuova interazione}
\label{uc:selezione-lingua-francese-raccomandazione}
\end{usecase}


\begin{usecase}{26}{Selezione lingua spagnolo per la raccomandazione}

\usecaseactors{Utente}
\usecasepre{Lo sviluppatore è entrato nel plug-in di simulazione all'interno dell'IDE}
\usecasedesc{La finestra di simulazione mette a disposizione i comandi per configurare, registrare o eseguire un test}
\usecasepost{Il sistema è pronto per permettere una nuova interazione}
\label{uc:selezione-lingua-spagnolo-raccomandazione}
\end{usecase}


\begin{usecase}{27}{Selezione lingua tedesco per la raccomandazione}

\usecaseactors{Utente}
\usecasepre{Lo sviluppatore è entrato nel plug-in di simulazione all'interno dell'IDE}
\usecasedesc{La finestra di simulazione mette a disposizione i comandi per configurare, registrare o eseguire un test}
\usecasepost{Il sistema è pronto per permettere una nuova interazione}
\label{uc:selezione-lingua-tedesco-raccomandazione}
\end{usecase}


\begin{usecase}{28}{Visualizzazione classifica elementi raccomandati}

\begin{figure}[!h] 
    \centering 
    \includegraphics[width=0.9\columnwidth]{usecase/UC28 - Visualizzazione classifica elementi raccomandati.png}
    \caption{UC28 - Visualizzazione classifica elementi raccomandati}
\end{figure}

\usecaseactors{Utente}
\usecasepre{Lo sviluppatore è entrato nel plug-in di simulazione all'interno dell'IDE}
\usecasedesc{La finestra di simulazione mette a disposizione i comandi per configurare, registrare o eseguire un test}
\usecasepost{Il sistema è pronto per permettere una nuova interazione}
\usecasealt{UC29}
\usespecial{UC30, UC31}
\label{uc:visualizzazione-classifica-elementi-raccomandati}
\end{usecase}


\begin{usecase}{29}{Visualizzazione errore di elaborazione della task}

\usecaseactors{Utente}
\usecasepre{Lo sviluppatore è entrato nel plug-in di simulazione all'interno dell'IDE}
\usecasedesc{La finestra di simulazione mette a disposizione i comandi per configurare, registrare o eseguire un test}
\usecasepost{Il sistema è pronto per permettere una nuova interazione}
\label{uc:visualizzazione-errore-elaborazione-task}
\end{usecase}


\begin{usecase}{30}{Visualizzazione classifica prodotti raccomandati}

\usecaseactors{Utente}
\usecasepre{Lo sviluppatore è entrato nel plug-in di simulazione all'interno dell'IDE}
\usecasedesc{La finestra di simulazione mette a disposizione i comandi per configurare, registrare o eseguire un test}
\usecasepost{Il sistema è pronto per permettere una nuova interazione}
\label{uc:visualizzazione-classifica-prodotti-raccomandati}
\end{usecase}


\begin{usecase}{31}{Visualizzazione classifica clienti raccomandati}

\usecaseactors{Utente}
\usecasepre{Lo sviluppatore è entrato nel plug-in di simulazione all'interno dell'IDE}
\usecasedesc{La finestra di simulazione mette a disposizione i comandi per configurare, registrare o eseguire un test}
\usecasepost{Il sistema è pronto per permettere una nuova interazione}
\label{uc:visualizzazione-classifica-clienti-raccomandati}
\end{usecase}













\section{Tracciamento dei requisiti}

Da un'attenta analisi dei requisiti e degli use case effettuata sul progetto è stata stilata la tabella che traccia i requisiti in rapporto agli use case.\\
Sono stati individuati diversi tipi di requisiti e si è quindi fatto utilizzo di un codice identificativo per distinguerli.\\
Il codice dei requisiti è così strutturato R(F/Q/V)(N/D/O) dove:
\begin{enumerate}
	\item[R =] requisito
    \item[F =] funzionale
    \item[Q =] qualitativo
    \item[V =] di vincolo
    \item[N =] obbligatorio (necessario)
    \item[D =] desiderabile
    \item[Z =] opzionale
\end{enumerate}
Nelle tabelle \ref{tab:requisiti-funzionali}, \ref{tab:requisiti-qualitativi} e \ref{tab:requisiti-vincolo} sono riassunti i requisiti e il loro tracciamento con gli use case delineati in fase di analisi.

\newpage

\begin{table}%
\caption{Tabella del tracciamento dei requisti funzionali}
\label{tab:requisiti-funzionali}
\begin{tabularx}{\textwidth}{lXl}
\hline\hline
\textbf{Requisito} & \textbf{Descrizione} & \textbf{Use Case}\\
\hline
RFN-1     & L'interfaccia permette di configurare il tipo di sonde del test & UC1 \\
\hline
\end{tabularx}
\end{table}%

\begin{table}%
\caption{Tabella del tracciamento dei requisiti qualitativi}
\label{tab:requisiti-qualitativi}
\begin{tabularx}{\textwidth}{lXl}
\hline\hline
\textbf{Requisito} & \textbf{Descrizione} & \textbf{Use Case}\\
\hline
RQD-1    & Le prestazioni del simulatore hardware deve garantire la giusta esecuzione dei test e non la generazione di falsi negativi & - \\
\hline
\end{tabularx}
\end{table}%

\begin{table}%
\caption{Tabella del tracciamento dei requisiti di vincolo}
\label{tab:requisiti-vincolo}
\begin{tabularx}{\textwidth}{lXl}
\hline\hline
\textbf{Requisito} & \textbf{Descrizione} & \textbf{Use Case}\\
\hline
RVO-1    & La libreria per l'esecuzione dei test automatici deve essere riutilizzabile & - \\
\hline
\end{tabularx}
\end{table}%

    \chapter{Report delle vendite}
\label{cap:report-vendite}

\intro{In questo capitolo, vengono descritte le teorie e le tecniche utilizzate per l'analisi delle vendite, con particolare attenzione allo studio dei dati, delle statistiche utili e dei grafici generati. Viene inoltre discusso il beneficio dell'automazione dell'analisi delle vendite in confronto all'alternativa manuale.}

\section{Studio delle colonne del dataset}

\section{Valutazione delle statistiche utili}

\section{Valutazione dei grafici utili}

\section{Benefici dell’automazione dell’analisi delle vendite}

    \chapter{Sistema di raccomandazione}
\label{cap:sistema-raccomandazione}

\intro{In questo capitolo, vengono descritte le teorie e le tecniche utilizzate per il sistema di raccomandazione, con particolare attenzione allo studio dei sistemi reali, delle tecniche di raccomandazione, di combinazione e di valutazione dei risultati. Viene inoltre dato un accenno ai temi della Serendipità e dell'Explainability, oggetto di studio del progetto sebbene non implementati allo stato dell'arte.}

\section{Requisiti per il dataset}

\section{Collaborative filtering}

\section{Similarità}

\section{Rank fusion}

\section{Recbole e Surprise}

\section{Metriche}

\section{Serendipità}

\section{Explainability}

    \chapter{Progettazione e implementazione}
\label{cap:progettazione-implementazione}

\intro{In questo capitolo, vengono descritte le scelte progettuali e le tecniche implementative utilizzate per realizzare il report delle vendite ed il sistema di raccomandazione. Si inizia con una panoramica del flusso delle attività, seguita da una descrizione delle tecnologie e degli strumenti utilizzati. Successivamente, si approfondiscono i vari componenti del sistema sviluppato.}

\section{Flusso delle attività}

\begin{figure}[!h]
    \centering 
    \includegraphics[width=0.9\columnwidth]{activity/Task di analisi delle vendite.png}
    \caption{Flusso delle attività della task di analisi delle vendite}
    \label{fig:activity-sales-analysis}
\end{figure}

Il flusso delle attività per la task di analisi delle vendite è rappresentato nella figura \ref{fig:activity-sales-analysis}. Le attività in sequenza previste dalla task dal lato dell'utente sono:
\begin{itemize}
    \item \textbf{Compilazione dei campi di input}: l'utente deve compilare i campi di input richiesti, cioè deve selezionare il file CSV contenente i dati delle vendite, selezionare la lingua del report e la valuta da utilizzare per i prezzi, e inserire l'indirizzo email a cui inviare il report;
    \item \textbf{Possibili output}: nel caso in cui l'utente abbia inserito correttamente i campi di input e nel caso la task non abbia riscontrato errori, l'utente visualizzerà un messaggio di esito positivo ed un invito a controllare la propria email per il report generato. Viene inoltre segnalata la generazione delle matrici di raccomandazione, e viene visualizzato il token che serve copiare per poter usufruire della task di raccomandazione. In caso la task non vada a buon fine, l'utente visualizzerà un messaggio di errore;
    \item \textbf{I file generati}: la task genera un file PDF ed un file HTML contenenti il report delle vendite, e l'utente ha la possibilità di scaricarli dall'apposita schermata del portale Oribea;
    \item \textbf{La ricezione della mail}: l'utente riceve una mail all'indirizzo email inserito in precedenza, il cui body contiene l'HTML del report, e riporta come allegati il report delle vendite in formato PDF ed HTML; inoltre, viene mostrato anche qui il token che serve copiare per poter usufruire della task di raccomandazione.
\end{itemize}

\begin{figure}[!h]
    \centering 
    \includegraphics[width=0.9\columnwidth]{activity/Task di raccomandazione di prodotti e clienti.png}
    \caption{Flusso delle attività della task di raccomandazione di prodotti e clienti}
    \label{fig:activity-recommendation-products-customers}
\end{figure}

Il flusso delle attività per la task di raccomandazione di prodotti e clienti è rappresentato nella figura \ref{fig:activity-recommendation-products-customers}. Le attività in sequenza previste dalla task dal lato dell'utente sono:
\begin{itemize}
    \item \textbf{Compilazione dei campi di input}: l'utente deve compilare i campi di input richiesti, cioè deve inserire il token ricevuto nella mail della task di analisi delle vendite, selezionare il tipo di raccomandazione tra "Raccomandare prodotti per un cliente" e "Raccomandare clienti per un prodotto", inserire il nome dell'elemento a cui raccomandare, inserire il numero di raccomandazioni desiderate, e selezionare la lingua con cui presentare la classifica di raccomandazioni;
    \item \textbf{Possibili output}: nel caso in cui l'utente abbia inserito correttamente i campi di input e nel caso la task non abbia riscontrato errori, l'utente visualizzerà la classifica di raccomandazioni generata. In caso la task non vada a buon fine, l'utente visualizzerà un messaggio di errore.
\end{itemize}


\section{Tecnologie e strumenti}
\label{sec:tecnologie-strumenti}

Di seguito viene data una panoramica delle tecnologie e strumenti utilizzati per lo sviluppo delle due task di analisi delle vendite e raccomandazione di prodotti e clienti. Le tecnologie sono state scelte in base alle esigenze del progetto, alla facilità d'uso e alla compatibilità con le altre componenti del sistema.

\subsection{Pandas}
Pandas è una libreria Python per l'analisi dei dati, che fornisce strutture dati e funzioni per la manipolazione e l'analisi dei dati. È stata utilizzata per leggere i file CSV contenenti i dati delle vendite e per elaborare i dati in fase di preprocessing.

\subsection{Babel}
Babel è una libreria Python per la gestione della localizzazione e internazionalizzazione delle applicazioni. È stata utilizzata per gestire le date nel report, in modo da poterle visualizzare nel formato corretto in base alla lingua selezionata dall'utente.

\subsection{Anthropic}
Anthropic è una libreria che consente di interagire con i modelli di linguaggio di grandi dimensioni (\gls{llm}) sviluppati dall'azienda Anthropic tramite le loro API. In particolare, è stato usato il modello Claude 3.7 Sonnet per riconoscere le colonne del file CSV contenente i dati delle vendite, e per generare il resoconto finale che fa da ultimo passaggio del report delle vendite.

\subsection{Matplotlib}
Matplotlib è una libreria Python per la creazione di grafici e visualizzazioni dei dati. È stata utilizzata per generare i grafici presenti nel report delle vendite, in modo da rendere i dati più comprensibili e visivamente accattivanti.

\subsection{Pillow}
Pillow è una libreria Python per la manipolazione delle immagini. È stata utilizzata per generare le immagini dei grafici creati con Matplotlib e per la loro conversione in base64, in modo da poterli inserire nel report delle vendite in formato PDF ed HTML.

\subsection{ReportLab}
ReportLab è una libreria Python per la generazione di documenti PDF, che fornisce una disposizione dinamica del contenuto in pagine. È stata utilizzata per creare il report delle vendite in formato PDF, in modo da renderlo scaricabile dal portale Oribea e da poterlo inviare via email all'utente.

\subsection{Jinja2}
Jinja2 è un motore di template per Python, che consente di generare documenti HTML in modo dinamico. È stata utilizzata per creare il report delle vendite in formato HTML, in modo da renderlo visualizzabile direttamente nel browser e da poterlo anche inviare via email all'utente.

\subsection{Server SMTP interno}
Per l'invio di email, è stato utilizzato un server SMTP messo a disposizione da Oribea, che consente di inviare mail personalizzate senza avere la necessità di utilizzare un indirizzo mail esistente o di collegarsi ad un servizio esterno a pagamento. 

\subsection{Sentence Transformers}
SentenceTransformers è una libreria Python per la creazione di modelli di linguaggio basati su trasformatori, che consente di generare rappresentazioni vettoriali di frasi e documenti. È stata utilizzata per calcolare le similarità tra le descrizioni dei prodotti e i nomi dei prodotti, in modo da poter generare le raccomandazioni. In particolare, è stato utilizzato il modello \emph{all-MiniLM-L6-v2} per generare le rappresentazioni vettoriali delle descrizioni dei prodotti e dei nomi dei prodotti, e per calcolare la similarità tra di essi.

\subsection{Scikit-learn}
Scikit-learn è una libreria Python per il machine learning, che fornisce strumenti per la creazione e l'addestramento di modelli di apprendimento automatico. È stata utilizzata per calcolare la similarità tra le descrizioni dei prodotti e i nomi dei prodotti, in modo da poter generare le raccomandazioni. In particolare, è stato utilizzato il metodo \emph{cosine similarity} per calcolare la similarità tra i vettori delle descrizioni dei prodotti e i vettori dei nomi dei prodotti.

\subsection{Zarr}
Zarr è una libreria Python per la gestione di array multidimensionali, che consente di memorizzare e gestire grandi quantità di dati in modo efficiente. È stata utilizzata per memorizzare su cloud le matrici di raccomandazione generate dalla prima task, in modo da poterle utilizzare successivamente per le raccomandazioni nella seconda task.

\subsection{Google Cloud Storage}
Google Cloud Storage è un servizio di archiviazione di oggetti su cloud fornito da Google. È stato utilizzato per memorizzare i file generati dalla task di analisi delle vendite, come il report in formato PDF ed HTML, e le matrici di raccomandazione generate dalla task di raccomandazione di prodotti e clienti. In particolare, sono stati utilizzati i pacchetti \emph{google.cloud.storage} e \emph{gcsfs} per interagire con il servizio: \emph{google.cloud.storage} è stato utilizzato per scrivere i file, mentre \emph{gcsfs} è stato utilizzato per leggerli.

\subsection{Google Cloud Functions}
Google Cloud Functions è un servizio di calcolo serverless fornito da Google, che consente di eseguire codice in risposta a eventi. È stato utilizzato per caricarvi il codice delle due task di analisi delle vendite e raccomandazione di prodotti e clienti, in modo da poterle eseguire in modo scalabile e senza doversi preoccupare della gestione dei server. Il portale Oribea è configurato per invocare le funzioni di Google Cloud Functions quando l'utente richiede l'esecuzione delle task.

\subsection{Numpy}
Numpy è una libreria Python per il calcolo scientifico, che fornisce strutture dati e funzioni per la manipolazione di array multidimensionali. È stata utilizzata per calcolare le metriche di valutazione delle raccomandazioni.

\subsection{Black}
Black è un formattatore di codice Python che consente di mantenere uno stile di codifica coerente e leggibile. È stato utilizzato per formattare il codice delle due task, in modo da renderlo più leggibile e mantenere uno stile di codifica uniforme.

\subsection{Ruff}
Ruff è un linter per Python che consente di rilevare errori di sintassi e problemi di stile nel codice. È stato utilizzato per verificare la correttezza del codice delle due task, in modo da garantire che fosse privo di errori e conforme agli standard di codifica.

\subsection{MyPy}
MyPy è un tipo di controllo per Python che consente di verificare la correttezza dei tipi di dati nel codice. È stato utilizzato per verificare la correttezza dei tipi di dati delle due task, in modo da garantire che il codice fosse privo di errori di tipo e conforme agli standard di codifica.

\subsection{Pytest}
Pytest è un framework di testing per Python che consente di scrivere e eseguire test automatizzati. È stato utilizzato per testare le due task, in modo da garantire che funzionassero correttamente e fossero prive di errori. I test sono stati scritti in modo da coprire i casi d'uso principali delle due task, e sono stati eseguiti automaticamente durante lo sviluppo per garantire la qualità del codice.


\section{Architettura del sistema}

Prima di cominciare a sviluppare le due task, è stata svolta un'analisi preliminare per definire l'architettura del sistema e le interazioni tra le varie componenti. Siccome le task rappresentano due funzionalità distinte, è stato deciso di svilupparle come due funzioni separate su Google Cloud Functions, e quindi come due diversi progetti il più separati possibile, in modo da poterle gestire in modo indipendente e poterle testare separatamente.
Per entrambi i progetti, è stato deciso di utilizzare il linguaggio Python, in modo da poter sfruttare le librerie e gli strumenti già descritti nella sezione \ref{sec:tecnologie-strumenti}.
Inoltre, per entrambi è stato scelto un approccio procedurale a discapito di un approccio orientato agli oggetti, poiché le task sono relativamente semplici e non richiedono una struttura complessa. Tuttavia, è stato deciso di utilizzare i moduli Python per organizzare il codice in modo modulare e rendere più facile la lettura e la manutenzione del codice.
I moduli Python sono rappresentati da cartelle e da file con estensione \emph{.py}, e sono stati utilizzati per raggruppare le funzioni correlate e per separare le responsabilità del codice. In particolare, sono stati creati i moduli descritti di seguito.

Per quanto riguarda la task di analisi delle vendite, sono stati creati i seguenti moduli:
\begin{itemize}
    \item \textbf{preparation}: questo modulo, rappresentato da una cartella, contiene tutti i file necessari per l'inizializzazione del sistema e per l'elaborazione dei dati prima dell'analisi delle vendite. In particolare, contiene i seguenti sottomoduli:
    \begin{itemize}
        \item \textbf{Dependency Injection}: questo modulo, rappresentato da un file \emph{.py}, contiene le funzioni per l'iniezione delle dipendenze necessarie per l'esecuzione della task, come i parametri di configurazione delle API verso l'\gls{llm}\glsfirstoccur{} Anthropic e verso il server SMTP interno di Oribea;
        \item \textbf{Preprocessing}: questo modulo, rappresentato da un file \emph{.py}, contiene le funzioni per il preprocessing dei dati delle vendite, in particolare la funzione che va a chiamare l'\gls{llm} Anthropic per riconoscere le colonne del file CSV contenente i dati delle vendite e le funzioni di standardizzazione dei nomi e di sanificazione dei dati;
        \item \textbf{Language Processing}: questo modulo, rappresentato da un file \emph{.py}, contiene le funzioni per il processamento del linguaggio naturale, in particolare le funzioni per la formattazione delle date e per la selezione dei testi corretti in base alla lingua selezionata dall'utente.
    \end{itemize}
    \item \textbf{report}: questo modulo, rappresentato da una cartella, contiene tutti i file necessari per la generazione del report delle vendite. In particolare, contiene i seguenti sottomoduli:
    \begin{itemize}
        \item \textbf{calculations_charts}: questo modulo, rappresentato da un file \emph{.py}, contiene le funzioni per il calcolo delle statistiche e dei trend di vendita e per la generazione dei grafici da inserire nel report delle vendite;
        \item \textbf{generate_report}: questo modulo, rappresentato da un file \emph{.py}, contiene le funzioni per la generazione del report delle vendite in formato PDF ed HTML, utilizzando le librerie ReportLab e Jinja2. Contiene inoltre le funzioni di supporto per la generazione del report, come la funzione di conversione delle immagini dei grafici in base64, la funzione per chiedere all'\gls{llm} Anthropic di generare il resoconto finale del report, e le funzioni per la conversione della risposta ottenuta dall'\gls{llm} da Markdown a flowables (per il PDF) e a HTML;
        \item \tetxbf{send_email}: questo modulo, rappresentato da un file \emph{.py}, contiene la funzione per l'invio dell'email all'utente con il report delle vendite in formato PDF ed HTML, che utilizza il server SMTP interno di Oribea, e una funzione di supporto per la creazione del body della mail.
    \end{itemize}
    \item \textbf{matrices}: questo modulo, rappresentato da un file \emph{.py}, contiene le funzioni per la generazione delle matrici di raccomandazione, che vengono salvate in oggetti di tipo Zarr. In particolare, contiene le funzioni per la generazione delle matrici di raccomandazione dei prodotti e dei clienti;
    \item \textbf{main}: questo modulo, rappresentato da un file \emph{.py}, contiene la funzione principale che viene eseguita quando la task viene invocata, e che coordina l'esecuzione delle altre funzioni dei moduli scritti in precedenza. Inoltre, contiene le funzioni per l'upload in Google Cloud Storage dei file generati dalla task, come il report delle vendite in formato PDF ed HTML e le matrici di raccomandazione.
\end{itemize}

Per quanto riguarda la task di raccomandazione di prodotti e clienti, sono stati creati i seguenti moduli:
\begin{itemize}
    \item \textbf{prediction}: questo modulo, rappresentato da una cartella, contiene tutti i file necessari per la predizione delle raccomandazioni. In particolare, contiene i seguenti sottomoduli:
    \begin{itemize}
        \item \textbf{predictor}: questo modulo, rappresentato da un file \emph{.py}, contiene le funzioni dirette per la predizione delle raccomandazioni, che utilizzano le matrici di raccomandazione generate dalla task di analisi delle vendite. Esso contiene anche l'algoritmo di rank fusion, che combina assieme le raccomandazioni ottenute dalle matrici;
        \item \textbf{filter}: questo modulo, rappresentato da un file \emph{.py}, contiene le funzioni per il filtraggio delle raccomandazioni, in modo da rimuovere, a seconda del caso, i prodotti già acquistati dall'utente ricevuto in input oppure gli utenti che hanno già acquistato il prodotto ricevuto in input;
        \item \textbf{make_prediction}: questo modulo, rappresentato da un file \emph{.py}, contiene le funzioni per la creazione della predizione delle raccomandazioni, che utilizzano le funzioni dei moduli \emph{predictor} e \emph{filter} per generare la classifica di raccomandazioni. Contiene inoltre una funzione di explanation che stampa su console i passaggi intermedi della predizione, per facilitare il debug e la comprensione del funzionamento della task;
        \item \textbf{build_output}: questo modulo, rappresentato da un file \emph{.py}, contiene una funzione che costruisce l'output della task, cioè inserisce un'introduzione ed una conclusione, selezionate in base alla lingua scelta dall'utente, attorno alla classifica di raccomandazioni.
    \end{itemize}
    \item \textbf{evaluation}: questo modulo, rappresentato da una cartella, contiene tutti i file necessari per la valutazione delle raccomandazioni. In particolare, contiene i seguenti sottomoduli:
    \begin{itemize}
        \item \textbf{metrics}: questo modulo, rappresentato da un file \emph{.py}, contiene le funzioni per il calcolo delle metriche di valutazione delle raccomandazioni. Queste metriche vengono calcolate confrontando le raccomandazioni generate dalla task con le vendite passate dei prodotti o dei clienti;
        \item \textbf{check_metrics}: questo modulo, rappresentato da un file \emph{.py}, contiene le funzioni per il controllo delle metriche di valutazione delle raccomandazioni, in modo da verificare se le raccomandazioni generate sono valide e soddisfacenti. In particolare, vengono richiamate le funzioni del modulo \emph{metrics} per calcolare le metriche di valutazione, che vengono stampate su console, assieme al loro significato, per facilitare il debug e la comprensione del funzionamento della task.
    \end{itemize}
    \item \textbf{main}: questo modulo, rappresentato da un file \emph{.py}, contiene la funzione principale che viene eseguita quando la task viene invocata, e che coordina l'esecuzione delle altre funzioni dei moduli scritti in precedenza. Inizialmente, essa si occupa di cercare in Google Cloud Storage il bucket identificato dal token ricevuto in input, e di caricare le matrici di raccomandazione da esso. Successivamente, essa richiama le funzioni del modulo \emph{prediction/make_prediction} per generare la classifica di raccomandazioni, e quelle del modulo \emph{evaluation/check_metrics} per calcolare le metriche di valutazione delle raccomandazioni. Infine, essa restituisce l'output della task, cioè la classifica di raccomandazioni.
\end{itemize}

Sono stati appena descritti i moduli contenuti dentro la cartella \emph{src} di ciascuna task, che rappresentano il cuore del codice dei due progetti. Attorno a tale cartella, è stata creata una struttura di cartelle e file che rappresentano la configurazione del progetto, le dipendenze necessarie per l'esecuzione delle task, i test automatizzati e un file main per collegarsi al servizio Cloud Functions. In particolare, sono stati creati i seguenti file e cartelle:
\begin{itemize}
    \item \textbf{requirements.txt}: questo file contiene le dipendenze necessarie per l'esecuzione delle task, che vengono installate automaticamente quando si carica il codice su Google Cloud Functions;
    \item \textbf{.gcloudignore}: questo file contiene le regole per ignorare i file e le cartelle che non devono essere caricati su Google Cloud Functions, come i file di test e i file di configurazione locali;
    \item \textbf{tests}: questa cartella contiene i test automatizzati per le due task, che vengono eseguiti automaticamente durante lo sviluppo per garantire la qualità del codice. I test sono scritti utilizzando il framework Pytest, e sono organizzati in moduli separati per ciascuna task;
    \item \textbf{main.py}: questo file contiene il codice principale per collegarsi al servizio Cloud Functions, e richiama la funzione principale della task corrispondente;
    \item \textbf{pyproject.toml}: questo file contiene la configurazione del progetto, come il nome del progetto, la versione, la descrizione e le dipendenze necessarie per l'esecuzione delle task. Inoltre, contiene le configurazioni per i tool di formattazione e linting del codice, come Black, Ruff e MyPy;
    \item \textbf{pytest.ini}: questo file contiene la configurazione per Pytest, il framework di testing utilizzato per i test automatizzati delle due task. In particolare, contiene la configurazione della coverage a 75\% per i test, e la configurazione per eseguire i test per l'\gls{llm} separatamente rispetto ai test "classici";
    \item \textbf{file .bat}: script per l'esecuzione dell'analisi statica del codice, che eseguono i tool di formattazione e linting Black, Ruff e MyPy. Questi script sono stati creati per facilitare l'esecuzione di questi ultimi tool, in modo da poterli consultare facilmente durante lo sviluppo.
\end{itemize}



\section{Preprocessing}
\label{sec:preprocessing}

\subsection{Riconoscimento delle marche}
\label{sec:recognition-brands}

Per poter analizzare le vendite in modo più dettagliato, è stato pensato di introdurre una colonna "Marca" nei dataset, che potesse contenere il nome della marca del prodotto presente in tale riga. Per ottenere ciò, era dunque necessario sviluppare un sistema di riconoscimento delle marche. Sono state provate le seguenti strategie:
\begin{itemize}
    \item \textbf{Riconoscimento tramite dizionario}: è stato creato un dizionario contenente le marche più comuni, ma si è rivelato poco efficace, poiché molte marche non erano presenti nel dizionario e il riconoscimento era limitato;
    \item \textbf{Riconoscimento tramite regex}: è stato provato a utilizzare delle espressioni regolari per riconoscere le marche, ma si è rivelato poco efficace, poiché molte marche non seguivano uno schema comune e il riconoscimento era limitato;
    \item \textbf{Riconoscimento tramite modelli di \gls{ml}\glsfirstoccur{}}: è stato pensato di creare un modello di \gls{ml} per riconoscere le marche, ma si è rivelato poco efficace, poiché il dataset non era sufficientemente grande e vario per addestrare un modello affidabile, e soprattuto non erano disponibili le etichette necessarie per un addestramento supervisionato;
    \item \textbf{Riconoscimento tramite modelli di \gls{ner}\glsfirstoccur{}}: è stato provato ad utilizzare un modello di \gls{ner} sulle descrizioni dei prodotti e a selezionare le entità segnalate di tipo \emph{ORG} (organizzazione), ma si è rivelato poco efficace, poiché il modello non era stato addestrato specificamente per questo compito e il riconoscimento era limitato;
    \item \textbf{Riconoscimento tramite \gls{llm}\glsfirstoccur{}}: è stato provato ad utilizzare un modello di linguaggio di grandi dimensioni (\gls{llm}) per riconoscere le marche, e si è rivelato mediamente efficace, ma il riconoscimento di ciascuna marca richiedeva troppo tempo e tantissime chiamate \gls{api}\glsfirstoccur{}; allora, è stato fatto un tentativo di raggruppamento di più descrizioni da inviare assieme per il riconoscimento di più marche contemporaneamente, ma si è rivelato poco efficace, poiché il modello ogni tanto si dimenticava di alcune marche o ne aggiungeva qualcuna in più, totalmente senza motivo (avvenivano cioè le cosidette "allucinazioni").
\end{itemize}

Dunque, si è deciso di non implementare il riconoscimento delle marche nel sistema di analisi automatizzato, poiché non era possibile garantire un riconoscimento affidabile e preciso.


\subsection{Riconoscimento delle categorie}
\label{sec:recognition-categories}

Per poter analizzare le vendite in modo più dettagliato, è stato pensato di introdurre una colonna "Categoria" nei dataset, che potesse contenere il nome della categoria del prodotto presente in tale riga. Per ottenere ciò, era dunque necessario sviluppare un sistema di riconoscimento delle categorie.

Escluse le opzioni già descritte nella sezione \ref{sec:recognition-brands} per il riconoscimento delle marche, è stato pensato di utilizzare un modello \gls{kmeans}\glsfirstoccur{} per raggruppare i prodotti in base alle loro descrizioni, in modo da ottenere delle categorie. Tuttavia, ciò si è rivelato poco efficace, poiché il modello non era in grado di raggruppare i prodotti in categorie in modo affidabile e preciso, e il numero di categorie era troppo elevato per poterle gestire manualmente.

Dunque, si è deciso di non implementare il riconoscimento delle categorie nel sistema di analisi automatizzato, poiché non era possibile garantire un riconoscimento affidabile e preciso.\\



\section{Language processing}
\label{sec:language-processing}

\section{Report}
\subsection{PDF}
\subsection{HTML}

\section{Invio di email}

\section{Le matrici di raccomandazione}
\subsection{Formato di archiviazione delle matrici}

\section{La predizione e rank fusion}

\section{Valutazione delle raccomandazioni}

\section{Collegamento con Google Cloud}
\subsection{Google Cloud Storage}
\subsection{Google Cloud Functions}
 % Progettazione e Implementazione, non consigliato da Zanella ma lo metto lo stesso
    \chapter{Verifica e validazione}
\label{cap:verifica-validazione}

\intro{Breve introduzione al capitolo}\\

\section{Test di unità}

\section{Test dell’LLM}
 % Verifica e Validazione, non consigliato da Zanella ma lo metto lo stesso
    \chapter{Conclusioni}
\label{cap:conclusioni}

\intro{In questo capitolo, vengono presentate le conclusioni del progetto, con un riepilogo degli obiettivi raggiunti, delle conoscenze acquisite e dei possibili miglioramenti futuri. Si conclude con una valutazione personale dell'esperienza di sviluppo.}

\section{Considerazioni finali}

Il periodo di tirocinio del laureando Stefani Riccardo ha avuto l'obiettivo di sviluppare un sistema di analisi vendite ed un sistema di raccomandazione per un'azienda di e-commerce, con l'intento di migliorare l'esperienza utente e incrementare le vendite attraverso suggerimenti personalizzati. Il progetto ha richiesto l'analisi dei dati di vendita esistenti, la preparazione dei dati, lo sviluppo di algoritmi di raccomandazione e l'integrazione con l'infrastruttura aziendale. L'implementazione è stata realizzata utilizzando tecnologie moderne come Python, machine learning e integrazione con \gls{googlecloudplatform}.

La maggior parte del tempo è stata dedicata all'analisi e preparazione dei dati di vendita, fase cruciale per garantire l'efficacia del sistema di raccomandazione. Il sistema sviluppato si è rivelato utile per l'azienda, fornendo sia informazioni sui trend di vendita utili per le decisioni strategiche, sia raccomandazioni personalizzate per gli utenti. Questo ha permesso di migliorare l'esperienza di acquisto, aumentando la soddisfazione del cliente e le vendite complessive.

Il sistema è stato implementato in due tasks caricate nella piattaforma Oribea, una per l'analisi delle vendite e l'altra per il sistema di raccomandazione. Sono state inoltre fornite due corrispettive interfacce utente contenenti i form che permettono di accedere alle due funzionalità senza passare attraverso la piattaforma aziendale. Il sistema è stato testato con successo su 10 dataset di vendita reali, dimostrando la sua efficacia e utilità.

È importante sottolineare, tuttavia, che il funzionamento del sistema è strettamente legato alla qualità e completezza del dataset fornito, che infatti deve possedere esattamente le colonne richieste per l'analisi delle vendite e deve possedere un numero sufficiente interazioni cliente-prodotto per generare raccomandazioni significative. Senza le colonne corrette, il sistema restituisce un errore e non può essere utilizzato. Inoltre, se il dataset è troppo scarso, le raccomandazioni generate potrebbero non essere utili o addirittura fuorvianti.


\section{Raggiungimento degli obiettivi}

Tutti i requisiti funzionali e non funzionali identificati nell'analisi iniziale e riportati nel capitolo \S\ref{cap:analisi-requisiti} sono stati raggiunti con successo.

Tra gli obiettivi definiti nel \emph{Piano di Lavoro} e riportati alla sezione \S\ref{sec:obiettivi-stage}, l'unico obiettivo non soddisfatto è stato l'obiettivo opzionale OZ1 relativo all'implementazione di un chatbot che si potesse collegare ad entrambe le \gls{cloudfunctions} e potesse così fornire entrambi i servizi in modo conversazionale, che si è rivelato troppo complesso da realizzare nei tempi disponibili nel tirocinio.


\section{Conoscenze acquisite}

Il tirocinio svolto ha soddisfatto appieno le aspettative: nonostante il progetto sia risultato piuttosto impegnativo, è stato molto utile per approfondire diverse tecnologie ed algoritmi, per acquisire competenze nell’ambito dei sistemi di analisi dati e di raccomandazione e per sperimentare alcune reali applicazioni di strumenti di intelligenza artificiale.

In particolare, le principali nuove conoscenze e competenze maturate durante il
periodo di stage sono le seguenti:

\begin{itemize}
    \item Approfondimento del linguaggio di programmazione Python e delle sue librerie per data science;
    \item Competenze nella preparazione e analisi di grandi volumi di dati;
    \item Competenze nella stesura di analisi delle vendite e reportistica;
    \item Competenze nello sviluppo di sistemi di raccomandazione nell'ambito e-commerce;
    \item Competenze nel collegamento e utilizzo di Google Cloud Platform;
    \item Competenze nell'integrazione con software aziendali pre-esistenti;
    \item Competenze nello sviluppo frontend di interfacce utente;
    \item Competenze di ottimizzazione delle performance di sistemi di analisi dati;
    \item Competenze nella stesura della documentazione tecnica di progetti software.
\end{itemize}


\section{Possibili miglioramenti futuri}

Nonostante il successo generale del sistema sviluppato, alcune funzionalità interessanti non sono state implementate per questioni di tempo e complessità. I possibili miglioramenti futuri includono:

\begin{itemize}
    \item Implementazione di un chatbot che possa interagire con gli utenti e fornire raccomandazioni personalizzate in modo conversazionale, e possa altrettanto comunicare l'analisi delle vendite ai proprietari dell'e-commerce, integrando così le due tasks sviluppate;
    \item Implementazione di un sistema di logging avanzato per monitorare l'esecuzione e le performance delle due tasks;
    \item Implementazione di algoritmi più sofisticati per migliorare la serendipità delle raccomandazioni;
    \item Implementazione di tecniche di \gls{data-reduction} per ridurre la dimensione delle matrici di raccomandazione senza compromettere la qualità dei risultati.
\end{itemize}

\section{Valutazione personale}

Questo progetto di tirocinio ha rappresentato un'importante tappa nel completamento del percorso universitario, permettendo di mettere in pratica le conoscenze teoriche acquisite, in particolare nell'ambito dell'Ingegneria del Software e dell'analisi dati.

L'esperienza è stata particolarmente utile per la crescita professionale a livello tecnico, fornendo competenze pratiche direttamente applicabili nel mondo del lavoro. Tuttavia, lavorando principalmente da remoto e in autonomia, non è stato purtroppo possibile sviluppare competenze di lavoro di squadra o di dinamiche aziendali.

Nonostante questa limitazione, sono complessivamente soddisfatto del risultato finale: gli obiettivi prefissati sono stati raggiunti e le competenze acquisite costituiscono una solida base per il proseguimento degli studi con la laurea magistrale e per future opportunità professionali nel campo dell'informatica e dell'analisi dati.


    \appendix
    \input{appendix/appendice-a}

    \backmatter
    \printglossary[type=\acronymtype, title=Acronimi e abbreviazioni, toctitle=Acronimi e abbreviazioni]
    \printglossary[type=main, title=Glossario, toctitle=Glossario]

    \cleardoublepage
\chapter{Bibliografia}

\nocite{*}

% Print book bibliography
\begin{refsegment}
\section*{Opere}
%\printbibliography[heading=subbibliography,title={Riferimenti bibliografici},type=book,resetnumbers=true]
\printbibliography[heading=none,category=opere,resetnumbers=true]
\end{refsegment}

% Print site bibliography
\begin{refsegment}
\section*{Siti web consultati}
%\printbibliography[heading=subbibliography,title={Siti web consultati},type=online,,resetnumbers=true]
\printbibliography[heading=none,category=web,resetnumbers=true]
\end{refsegment}

\end{document}
