\chapter{Introduzione}
\label{cap:introduzione}

\intro{In questo capitolo verrà descritta l’azienda proponente del tirocinio, il way of working, l’organizzazione del testo e delle convenzioni tipografiche impostate.}\\

\noindent Introduzione al contesto applicativo.\\

\noindent Esempio di utilizzo di un termine nel glossario \\
\gls{api}. \\

\noindent Esempio di citazione in linea \\
\cite{site:agile-manifesto}. \\

\noindent Esempio di citazione nel pie' di pagina \\
citazione\footcite{womak:lean-thinking} \\

\section{L'azienda}

Oribea AI S.r.l. è una startup innovativa fondata nel 2024 nella Repubblica di San Marino, in seguito alla separazione dall'azienda di e-commerce ITTweb. La missione di Oribea è fornire soluzioni avanzate di intelligenza artificiale per migliorare l'efficienza e la produttività delle aziende, con un focus particolare sull'implementazione di Large Language Models (LLM) e agenti intelligenti.
Tra i principali prodotti sviluppati da Oribea vi è l'AI Agent Builder, uno strumento che consente alle imprese di creare e integrare agenti intelligenti personalizzati nei propri processi aziendali. Questi agenti sono progettati per automatizzare attività ripetitive, migliorare la comunicazione interna ed esterna e supportare la presa di decisioni attraverso l'analisi avanzata dei dati. L'AI Agent Builder si distingue per la sua capacità di adattarsi alle specifiche esigenze di ciascuna azienda, offrendo soluzioni su misura che sfruttano le potenzialità degli LLM.
Inoltre, Oribea sta sviluppando un Sistema Intelligente, concepito per fungere da piattaforma centrale nell'orchestrazione delle attività aziendali. Questo sistema mira a integrare diverse applicazioni e servizi, facilitando la gestione dei processi e migliorando la coerenza e l'efficienza operativa. Attraverso l'uso di tecnologie avanzate di intelligenza artificiale, il Sistema Intelligente di Oribea promette di trasformare il modo in cui le aziende operano, rendendo i processi più fluidi e reattivi alle esigenze del mercato.
La scelta di stabilire la sede a San Marino non è casuale: la Repubblica si sta posizionando come un hub per l'innovazione tecnologica, offrendo un ambiente favorevole allo sviluppo e alla sperimentazione di nuove tecnologie. In questo contesto, Oribea beneficia di un ecosistema dinamico e di una rete di collaborazioni che favoriscono la crescita e l'innovazione.
In sintesi, l'aziemda rappresenta un esempio di come le startup possano contribuire significativamente all'evoluzione del panorama tecnologico, offrendo soluzioni innovative che rispondono alle sfide contemporanee delle aziende. La sua focalizzazione sull'intelligenza artificiale applicata ai processi aziendali la rende un attore rilevante nel contesto della trasformazione digitale.
\dots

\begin{figure}
    \centering
    \includegraphics[width=0.5\textwidth]{oribea-logo.png}
    \caption{Logo di Oribea AI S.r.l.}
    \label{fig:oribea-logo}
\end{figure}

\section{L'idea}

Introduzione all'idea dello stage.

\section{Organizzazione del testo}

\subsection{Struttura del documento}
\label{sec:organizzazione-testo}

\begin{description}
    \item[{\hyperref[cap:processi-metodologie]{Il secondo capitolo}}] descrive ...
    
    \item[{\hyperref[cap:descrizione-stage]{Il terzo capitolo}}] approfondisce ...
    
    \item[{\hyperref[cap:analisi-requisiti]{Il quarto capitolo}}] approfondisce ...
    
    \item[{\hyperref[cap:progettazione-codifica]{Il quinto capitolo}}] approfondisce ...
    
    \item[{\hyperref[cap:verifica-validazione]{Il sesto capitolo}}] approfondisce ...
    
    \item[{\hyperref[cap:conclusioni]{Nel settimo capitolo}}] descrive ...
\end{description}

\subsection{Convenzioni tipografiche}
\label{sec:convenzioni-tipografiche}

Riguardo la stesura del testo, relativamente al documento sono state adottate le seguenti convenzioni tipografiche:
\begin{itemize}
	\item gli acronimi, le abbreviazioni e i termini ambigui o di uso non comune menzionati vengono definiti nel glossario, situato alla fine del presente documento;
	\item per la prima occorrenza dei termini riportati nel glossario viene utilizzata la seguente nomenclatura: \emph{parola}\glsfirstoccur;
	\item i termini in lingua straniera o facenti parti del gergo tecnico sono evidenziati con il carattere \emph{corsivo}.
\end{itemize}
