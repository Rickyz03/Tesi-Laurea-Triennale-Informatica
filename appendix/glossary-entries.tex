% Acronyms
\newacronym[description={\glslink{apig}{Application Program Interface}}]
    {api}{API}{Application Program Interface}

\newacronym[description={\glslink{umlg}{Unified Modeling Language}}]
    {uml}{UML}{Unified Modeling Language}

\newacronym[description={\glslink{llm}{Large Language Model}}]
    {llm}{LLM}{Large Language Model}

\newacronym[description={\glslink{rrf}{Reciprocal Rank Fusion}}]
    {rrf}{RRF}{Reciprocal Rank Fusion}

% Glossary entries
\newglossaryentry{apig} {
    name=\glslink{api}{API},
    text=Application Program Interface,
    sort=api,
    description={in informatica con il termine \emph{Application Programming Interface API} (ing. interfaccia di programmazione di un'applicazione) si indica ogni insieme di procedure disponibili al programmatore, di solito raggruppate a formare un set di strumenti specifici per l'espletamento di un determinato compito all'interno di un certo programma. La finalità è ottenere un'astrazione, di solito tra l'hardware e il programmatore o tra software a basso e quello ad alto livello semplificando così il lavoro di programmazione}
}

\newglossaryentry{umlg} {
    name=\glslink{uml}{UML},
    text=UML,
    sort=uml,
    description={in ingegneria del software \emph{UML, Unified Modeling Language} (ing. linguaggio di modellazione unificato) è un linguaggio di modellazione e specifica basato sul paradigma object-oriented. L'\emph{UML} svolge un'importantissima funzione di ``lingua franca'' nella comunità della progettazione e programmazione a oggetti. Gran parte della letteratura di settore usa tale linguaggio per descrivere soluzioni analitiche e progettuali in modo sintetico e comprensibile a un vasto pubblico}
}

\newglossaryentry{llm} {
    name=LLM,
    text=LLM,
    sort=llm,
    description={\emph{Large Language Model}, modello di linguaggio di grandi dimensioni addestrato su grandi quantità di dati testuali per comprendere e generare testo in linguaggio naturale}
}

\newglossaryentry{agile} {
    name=Agile,
    text=Agile,
    sort=agile,
    description={insieme di metodologie di sviluppo software basate su iterazioni brevi, collaborazione continua e adattamento ai cambiamenti, secondo i principi del Manifesto Agile}
}

\newglossaryentry{scrum} {
    name=Scrum,
    text=Scrum,
    sort=scrum,
    description={framework Agile per la gestione e lo sviluppo di progetti complessi, che suddivide il lavoro in iterazioni chiamate sprint e promuove la collaborazione tra team cross-funzionali}
}

\newglossaryentry{cloudfunction} {
    name=Cloud Function,
    text=Cloud Function,
    sort=cloudfunction,
    description={servizio di cloud computing che consente di eseguire codice in risposta a eventi senza dover gestire server, tipico dell'architettura serverless}
}

\newglossaryentry{cloudstorage} {
    name=Cloud Storage,
    text=Cloud Storage,
    sort=cloudstorage,
    description={servizio che permette di archiviare e gestire dati su server remoti accessibili tramite Internet, fornendo scalabilità e affidabilità}
}

\newglossaryentry{cloudcomputing} {
    name=Cloud Computing,
    text=Cloud Computing,
    sort=cloudcomputing,
    description={modello di erogazione di risorse informatiche (come server, storage, database, networking, software) tramite Internet, su richiesta e con pagamento a consumo}
}

\newglossaryentry{similarita} {
    name=Similarità,
    text=similarità,
    sort=similarita,
    description={misura della somiglianza tra due o più oggetti, spesso utilizzata in informatica per confrontare testi, immagini o altri dati}
}

\newglossaryentry{rrf} {
    name=Reciprocal Rank Fusion (RRF),
    text=RRF,
    sort=rrf,
    description={metodo di combinazione dei risultati di più sistemi di recupero dell'informazione, che assegna un punteggio a ciascun elemento in base alla posizione nelle classifiche parziali, favorendo i risultati che appaiono in più classifiche}
}
