\chapter{Descrizione dello stage}
\label{cap:descrizione-stage}

\intro{Breve introduzione al capitolo}\\

\section{Introduzione al progetto}

% Lavoro da casa eccetera

\section{Way of working e strumenti utilizzati}
\label{sec:way-of-working}

Il modo di lavorare di Oribea è caratterizzato da un approccio agile e collaborativo, che incoraggia la comunicazione aperta e il lavoro di squadra.\\

Gli strumenti principali utilizzati includono:
\begin{itemize}
    \item \textbf{Visual Studio Code}: per la scrittura e la modifica del codice sorgente;
    \item \textbf{Git}: per il versionamento del codice e la gestione delle modifiche;
    \item \textbf{GitHub}: per la gestione del codice sorgente e la collaborazione tra sviluppatori;
    \item \textbf{Slack}: per la comunicazione interna e la gestione dei progetti;
    \item \textbf{Notion}: per la documentazione del progetto;
    \item \textbf{draw.io}: per la creazione di diagrammi e modelli UML; 
\end{itemize}

\section{Analisi preventiva dei rischi}

Durante la fase di analisi iniziale sono stati individuati alcuni possibili rischi a cui si potrà andare incontro.
Si è quindi proceduto a elaborare delle possibili soluzioni per far fronte a tali rischi.\\

\begin{risk}{Assenza di dataset di addestramento}
    \riskdescription{s}
    \risksolution{a}
    \label{risk:dataset-absence} 
\end{risk}

\begin{risk}{Risposta dell'LLM insoddisfacente}
    \riskdescription{s}
    \risksolution{a}
    \label{risk:bad-llm-response} 
\end{risk}
