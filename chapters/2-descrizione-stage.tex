\chapter{Descrizione dello stage}
\label{cap:descrizione-stage}

\intro{Breve introduzione al capitolo}\\

\section{Introduzione al progetto}

% Lavoro da casa eccetera

\section{Way of working e strumenti utilizzati}
\label{sec:way-of-working}

L’azienda Oribea adotta un modello di sviluppo Agile, con l’obiettivo di monitorare e controllare il progetto in modo flessibile e continuo, suddividendo le attività in piccoli incrementi e con una collaborazione asincrona e distribuita.\\
In particolare, il modello di sviluppo adottato è Scrum, che prevede la suddivisione del progetto in sprint, ovvero periodi di tempo di durata fissa, in cui vengono pianificate le attività da svolgere e i relativi obiettivi da raggiungere. All'inizio del progetto è stato scelto di comune accordo la lunghezza per gli sprint di una settimana. Al termine di ogni sprint, è stato effettuato un incontro approfondito con il tutor aziendale, per discutere lo stato di avanzamento del progetto e le attività da svolgere per il successivo sprint.\\
L’obiettivo del modello è dare maggiore importanza al ciclo di vita del software e dei processi correlati, piuttosto che al prodotto finale, con il fine ultimo di migliorare la qualità del prodotto stesso.\\

Gli strumenti principali utilizzati includono:
\begin{itemize}
    \item \textbf{Visual Studio Code}: per la scrittura e la modifica del codice sorgente;
    \item \textbf{Git}: per il versionamento del codice e la gestione delle modifiche;
    \item \textbf{GitHub}: per la gestione del codice sorgente e la collaborazione tra sviluppatori;
    \item \textbf{Google Cloud Functions}: per l'hosting delle functions serverless da collegare ai task di Oribea;
    \item \textbf{Google Cloud Storage}: per l'archiviazione dei file e delle matrice di raccomandazione;
    \item \textbf{Slack}: per la comunicazione interna e la gestione dei progetti;
    \item \textbf{Notion}: per la documentazione del progetto;
    \item \textbf{StarUML}: per la creazione dei diagrammi UML.
\end{itemize}

\section{Analisi preventiva dei rischi}

Durante la fase di analisi iniziale sono stati individuati alcuni possibili rischi a cui si potrà andare incontro.
Si è quindi proceduto a elaborare delle possibili soluzioni per far fronte a tali rischi.\\

\begin{risk}{Assenza di dataset di addestramento}
    \riskdescription{s}
    \risksolution{a}
    \label{risk:dataset-absence} 
\end{risk}

\begin{risk}{Risposta dell'LLM imprecisa}
    \riskdescription{s}
    \risksolution{a}
    \label{risk:bad-llm-response} 
\end{risk}
