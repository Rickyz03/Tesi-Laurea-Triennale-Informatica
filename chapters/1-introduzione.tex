\chapter{Introduzione}
\label{cap:introduzione}

\intro{In questo capitolo verrà descritta l’azienda proponente del tirocinio, il way of working, l’organizzazione del testo e delle convenzioni tipografiche impostate.}\\

\noindent Introduzione al contesto applicativo.\\

\noindent Esempio di utilizzo di un termine nel glossario \\
\gls{api}. \\

\noindent Esempio di citazione in linea \\
\cite{site:agile-manifesto}. \\

\noindent Esempio di citazione nel pie' di pagina \\
citazione\footcite{womak:lean-thinking} \\

\section{L'azienda}

Oribea AI S.r.l. è una startup innovativa fondata nel 2024 nella Repubblica di San Marino, in seguito alla separazione dall'azienda di e-commerce ITTweb. La missione di Oribea è fornire soluzioni avanzate di intelligenza artificiale per migliorare l'efficienza e la produttività delle aziende, con un focus particolare sull'implementazione di \gls{llm}\glsfirstoccur{} e agenti intelligenti.
Tra i principali prodotti sviluppati da Oribea vi è l'AI Agent Builder, uno strumento che consente alle imprese di creare e integrare agenti intelligenti personalizzati nei propri processi aziendali. Questi agenti sono progettati per automatizzare attività ripetitive, migliorare la comunicazione interna ed esterna e supportare la presa di decisioni attraverso l'analisi avanzata dei dati. L'AI Agent Builder si distingue per la sua capacità di adattarsi alle specifiche esigenze di ciascuna azienda, offrendo soluzioni su misura che sfruttano le potenzialità degli \gls{llm}.
Inoltre, Oribea sta sviluppando un Sistema Intelligente, concepito per fungere da piattaforma centrale nell'orchestrazione delle attività aziendali. Questo sistema mira a integrare diverse applicazioni e servizi, facilitando la gestione dei processi e migliorando la coerenza e l'efficienza operativa. Attraverso l'uso di tecnologie avanzate di intelligenza artificiale, il Sistema Intelligente di Oribea promette di trasformare il modo in cui le aziende operano, rendendo i processi più fluidi e reattivi alle esigenze del mercato.
La scelta di stabilire la sede a San Marino non è casuale: la Repubblica si sta posizionando come un hub per l'innovazione tecnologica, offrendo un ambiente favorevole allo sviluppo e alla sperimentazione di nuove tecnologie. In questo contesto, Oribea beneficia di un ecosistema dinamico e di una rete di collaborazioni che favoriscono la crescita e l'innovazione.
In sintesi, l'aziemda rappresenta un esempio di come le startup possano contribuire significativamente all'evoluzione del panorama tecnologico, offrendo soluzioni innovative che rispondono alle sfide contemporanee delle aziende. La sua focalizzazione sull'intelligenza artificiale applicata ai processi aziendali la rende un attore rilevante nel contesto della trasformazione digitale.

\begin{figure}
    \centering
    \includegraphics[width=0.5\textwidth]{oribea-logo.png}
    \caption{Logo di Oribea AI S.r.l.}
    \label{fig:oribea-logo}
\end{figure}

\section{L'idea}

L'idea dello stage è nata dalla necessità di sviluppare un sistema che consenta di generare automaticamente un report di analisi delle vendite per un'azienda di e-commerce. Questo report deve essere generato in modo autonomo, senza la necessità di intervento umano, e deve essere in grado di analizzare i dati delle vendite, identificare tendenze e fornire raccomandazioni per migliorare le performance aziendali.

In aggiunta, dallo stesso dataset delle vendite, il sistema deve essere in grado di generare un sistema di raccomandazioni per i clienti, suggerendo prodotti in base alle loro preferenze e comportamenti di acquisto. Questo approccio mira a migliorare l'esperienza del cliente e aumentare le vendite attraverso raccomandazioni personalizzate. Il sistema deve anche suggerire possibili clienti a cui proporre i prodotti, in modo da ottimizzare le strategie di marketing e vendita.
L'obiettivo finale è quello di creare un sistema integrato che possa automatizzare e ottimizzare i processi di analisi delle vendite e raccomandazione dei prodotti, contribuendo così a migliorare l'efficienza operativa dell'azienda e a massimizzare le opportunità di vendita.

Ho scelto questo progetto di stage con il desiderio di approfondire le mie conoscenze nel campo dell'intelligenza artificiale e del machine learning, gli argomenti di cui sono più interessato, in particolare nell'ambito dell'analisi dei dati e delle raccomandazioni personalizzate.

\section{Organizzazione del testo}

\subsection{Struttura del documento}
\label{sec:organizzazione-testo}
Il presente documento è suddiviso in otto capitoli il cui contenuto è brevemente riassunto in seguito:

\begin{description}
    \item[{\hyperref[cap:descrizione-stage]{Il secondo capitolo}}] descrive nel dettaglio il progetto di stage, le tecnologie utilizzate e il modo di lavorare dell'azienda; inoltre, viene fornita un'analisi dei rischi e delle soluzioni adottate per affrontarli;
    
    \item[{\hyperref[cap:analisi-requisiti]{Il terzo capitolo}}] approfondisce l'analisi dei requisiti del sistema, con particolare attenzione alla definizione dei casi d'uso e dei requisiti funzionali e non funzionali, con apposito tracciamento;
    
    \item[{\hyperref[cap:report-vendite]{Il quarto capitolo}}] approfondisce la teoria che sta ala base del report di analisi delle vendite, con particolare attenzione allo studio dei dati, delle tecniche di analisi utilizzate e dei grafici generati;
    
    \item[{\hyperref[cap:sistema-raccomandazione]{Il quinto capitolo}}] approfondisce la teoria che sta alla base del sistema di raccomandazione, con particolare attenzione allo studio di sistemi reali e delle tecniche di raccomandazione e valutazione dei risultati utilizzate;
    
    \item[{\hyperref[cap:progettazione-implementazione]{Il sesto capitolo}}] approfondisce la progettazione e l'implementazione del sistema, con particolare attenzione alla scelta delle tecnologie utilizzate e al loro utilizzo;
    
    \item[{\hyperref[cap:verifica-validazione]{Il settimo capitolo}}] approfondisce le attività di verifica e validazione del sistema, con particolare attenzione ai test di unità sviluppati e all'approccio adottato per i test riguardanti l'LLM;
    
    \item[{\hyperref[cap:conclusioni]{L'ottavo capitolo}}] rappresenta una sintesi finale del lavoro svolto durante il periodo di tirocinio, descrivendo eventuali successi e difficoltà incontrate durante il percorso. Vengono inoltre analizzati i risultati ottenuti rispetto agli obiettivi iniziali così come l’insieme di competenze teoriche e pratiche acquisite nel corso del progetto. Il documento si conclude con una riflessione critica sull’operato e sulla crescita personale e professionale del laureando Riccardo Stefani durante il tirocinio. Infine, viene fornita una panoramica delle prospettive future per il progetto e per l'azienda, evidenziando le opportunità di sviluppo e miglioramento.

\end{description}

\subsection{Convenzioni tipografiche}
\label{sec:convenzioni-tipografiche}

In merito alla redazione del presente documento, sono state adottate le seguenti convenzioni tipografiche:
\begin{itemize}
	\item Gli acronimi, le abbreviazioni e i termini ambigui o di uso non comune menzionati vengono definiti nel glossario, situato alla fine del presente documento;

	\item Per la prima occorrenza dei termini riportati nel glossario viene utilizzato il seguente stile: \gls{api}\glsfirstoccur{};

	\item I termini particolarmente rilevanti in una sezione e quelli in lingua straniera non di uso comune o facenti parti del gergo tecnico sono evidenziati con il carattere \emph{corsivo}, fatta eccezione per le occorrenze presenti nei titoli delle sezioni o nelle didascalie;

	\item I comandi di terminale, i frammenti di codice sorgente e i nomi di file o directory sono evidenziati con il carattere \texttt{monospaziato}.

\end{itemize}
