\chapter{Introduzione}
\label{cap:introduzione}

\intro{In questo capitolo verrà descritta l’azienda proponente del tirocinio, il way of working, l’organizzazione del testo e delle convenzioni tipografiche impostate.}\\

\noindent Introduzione al contesto applicativo.\\

\noindent Esempio di utilizzo di un termine nel glossario \\
\gls{api}. \\

\noindent Esempio di citazione in linea \\
\cite{site:agile-manifesto}. \\

\noindent Esempio di citazione nel pie' di pagina \\
citazione\footcite{womak:lean-thinking} \\

\section{L'azienda}

Oribea AI S.r.l. è una startup innovativa che si occupa di intelligenza artificiale, nata a San Marino nel 2024 dopo la separazione dall'azienda di e-commerce ITTWEB. La missione di Oribea è quella di fornire soluzioni innovative per migliorare l'efficienza e la produttività delle aziende attraverso l'uso di tecnologie avanzate, come i Large Language Models (LLM).
\dots

\begin{figure}
    \centering
    \includegraphics[width=0.5\textwidth]{oribea-logo.png}
    \caption{Logo di Oribea AI S.r.l.}
    \label{fig:oribea-logo}
\end{figure}

\section{L'idea}

Introduzione all'idea dello stage.

\section{Organizzazione del testo}

\subsection{Struttura del documento}
\label{sec:organizzazione-testo}

\begin{description}
    \item[{\hyperref[cap:processi-metodologie]{Il secondo capitolo}}] descrive ...
    
    \item[{\hyperref[cap:descrizione-stage]{Il terzo capitolo}}] approfondisce ...
    
    \item[{\hyperref[cap:analisi-requisiti]{Il quarto capitolo}}] approfondisce ...
    
    \item[{\hyperref[cap:progettazione-codifica]{Il quinto capitolo}}] approfondisce ...
    
    \item[{\hyperref[cap:verifica-validazione]{Il sesto capitolo}}] approfondisce ...
    
    \item[{\hyperref[cap:conclusioni]{Nel settimo capitolo}}] descrive ...
\end{description}

\subsection{Convenzioni tipografiche}
\label{sec:convenzioni-tipografiche}

Riguardo la stesura del testo, relativamente al documento sono state adottate le seguenti convenzioni tipografiche:
\begin{itemize}
	\item gli acronimi, le abbreviazioni e i termini ambigui o di uso non comune menzionati vengono definiti nel glossario, situato alla fine del presente documento;
	\item per la prima occorrenza dei termini riportati nel glossario viene utilizzata la seguente nomenclatura: \emph{parola}\glsfirstoccur;
	\item i termini in lingua straniera o facenti parti del gergo tecnico sono evidenziati con il carattere \emph{corsivo}.
\end{itemize}
